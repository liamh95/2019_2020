\documentclass[11pt,letterpaper]{report}
\usepackage{amssymb,amsfonts,color,graphicx,amsmath,enumerate}
\usepackage{amsthm}

\newcommand{\naturals}{\mathbb{N}}
\newcommand{\integers}{\mathbb{Z}}
\newcommand{\complex}{\mathbb{C}}
\newcommand{\reals}{\mathbb{R}}
\newcommand{\mcal}[1]{\mathcal{#1}}
\newcommand{\rationals}{\mathbb{Q}}
\newcommand{\field}{\mathbb{F}}
\newcommand{\Var}{\text{Var}}
\newcommand{\ind}{\mathbbm{1}}
\newcommand{\Cov}{\text{Cov}}

\newenvironment{solution}
{\begin{proof}[Solution]}
{\end{proof}}

\voffset=-3cm
\hoffset=-2.25cm
\textheight=24cm
\textwidth=17.25cm
\addtolength{\jot}{8pt}
\linespread{1.3}

\begin{document}
\begin{center}
{\bf \Large Math 130B - More on Joint Random Variables}
\vspace{0.2cm}
\hrule
\end{center}

\begin{enumerate}
	\item Let $X$ and $Y$ be independent variables having the exponential distribution with parameters $\lambda$ and $\mu$, respectively. Find the density function of $X+Y$.

	\vfill

	\item Let $X$ and $Y$ be independent random variables, $X$ being equally likely to take any value in $\{0, 1, \ldots, m\}$ and $Y$ similarly in $\{0, 1, \ldots, n\}$. Find the mass function of $Z = X+Y$.

	\vfill

	\item Suppose that $n$ points are independently chosen at random on the circumference of a circle, and we want the probability that they all lie in a semicircle. That is, we want the probability that there is a line passing through the center of the circle such that all the points are on one side of that line.\\

	\noindent Let $P_1, \ldots, P_n$ denote the $n$ points. Let $A^{(n)}$ denote the event that all the points are contained in some semicircle, and let $A^{(n)}_i$ be the event that all the points lie in the semicircle beginning at the point $P_i$ and going clockwise for $180^\circ$, $i = 1, \ldots, n$.
	\begin{enumerate}
		\item Express $A^{(n)}$ in terms of the $A^{(n)}_i$.
		\item Find $\Pr[A^{(n)}]$ and show that it is $o(1)$.
	\end{enumerate}

	\vfill


\end{enumerate}

\end{document}