\documentclass[11pt,letterpaper]{report}
\usepackage{amssymb,amsfonts,color,graphicx,amsmath,enumerate}
\usepackage{tikz}
\usepackage{amsthm}
\usepackage{bbm}

\newcommand{\naturals}{\mathbb{N}}
\newcommand{\integers}{\mathbb{Z}}
\newcommand{\complex}{\mathbb{C}}
\newcommand{\reals}{\mathbb{R}}
\newcommand{\exreals}{\overline{\mathbb{R}}}
\newcommand{\mcal}[1]{\mathcal{#1}}
\newcommand{\mable}{measurable}
\newcommand{\quats}{\mathbb{H}}
\newcommand{\rationals}{\mathbb{Q}}
\newcommand{\norm}{\trianglelefteq}
\newcommand{\Aut}{\text{Aut}}
\newcommand{\disk}{\mathbb{D}}
\newcommand{\halfplane}{\mathbb{H}}
\newcommand{\Lp}[2]{\left\|{#1}\right\|_{L^{#2}}}
\newcommand{\supp}[1]{\text{supp}({#1})}
\newcommand{\Hom}[2]{\text{Hom}_{{#1}}({#2})}
\newcommand{\tr}{\text{tr}}
\newcommand{\field}{\mathbb{F}}
\newcommand{\Gal}[1]{\text{Gal}\left({#1}\right)}
\newcommand{\esssup}{\text{ess sup }}
\newcommand{\essinf}{\text{ess inf }}
\newcommand{\affine}{\mathbb{A}}
\newcommand{\E}{\mathbb{E}}
\newcommand{\Prob}{\mathbb{P}}
\newcommand{\Var}{\text{Var}}
\newcommand{\ind}{\mathbbm{1}}
\newcommand{\Cov}{\text{Cov}}

\newenvironment{solution}
{\begin{proof}[Solution]}
{\end{proof}}

\voffset=-3cm
\hoffset=-2.25cm
\textheight=24cm
\textwidth=17.25cm
\addtolength{\jot}{8pt}
\linespread{1.3}

\begin{document}
\noindent{\em Liam Hardiman\hfill{December 9, 2019} }
\begin{center}
{\bf \Large 270A - Homework 4}
\vspace{0.2cm}
\hrule
\end{center}

\noindent\textbf{Problem 1. }
Let $E_1, E_2, \ldots$ be events on the same probability space. Assume that
\[
\Prob[E_n]\to 0\quad\text{and}\quad \sum_n\Prob[E_n\cap E_{n-1}^c]<\infty.
\]
Show that
\[
\Prob[E_n\text{ occur infinitely often}] = 0.
\]
\begin{proof}
	By the continuity of measure we have
	\begin{align*}
		\Prob[\limsup E_n^c] &= \lim_{N\to \infty}\Prob\left[\bigcup_{n=N}^\infty E_n^c\right]\\
		&\geq \lim_{N\to \infty}\Prob[E_n^c]\\
		&= 1,
	\end{align*}
	where the last equality follows from the hypothesis that $\Prob[E_n]\to 0$. Now by Borel-Cantelli, $\Prob[E_n\cap E_{n-1}^c\text{ i.o.}] = 0$. Since $\Prob[E_{n-1}^c\text{ i.o.}] = 1$ by the above discussion, we have
	\begin{align*}
		\Prob[E_n\text{ i.o.}] = \Prob[E_n\cap E_{n-1}^c\text{ i.o.}]= 0.
	\end{align*}
\end{proof}


\noindent\textbf{Problem 2. }
Let $X_1, X_2, \ldots$ be iid random variables with the standard exponential distribution,
\[
\Prob[X_i>x] = e^{-x},\quad x\geq 0.
\]
\begin{enumerate}[(a)]
	\item Show that
	\[
	\limsup_n \frac{X_n}{\log n} = 1\text{ a.s.}
	\]
	\begin{proof}
		For any positive $t$ we have
		\[
		\sum_{n=1}^\infty \Prob[X_n > t\log n] = \sum_{n=1}^\infty \frac{1}{n^t} = \begin{cases}
			C_t<\infty,&\text{if }t\leq 1\\
			\infty,&\text{if }t>1
		\end{cases}.
		\]
		By Borel-Cantelli, we then have
		\[
		\Prob\left[\frac{X_n}{\log n}>t\text{ infinitely often}\right] = \begin{cases}
			1,&\text{if }t\leq 1\\
			0,&\text{if }t>1
		\end{cases}.
		\]
		In particular, we have that $\frac{X_n}{\log n} > 1$ infinitely often almost surely, but $\frac{X_n}{\log n}>t$ only finitely often almost surely for any $t>1$. We conclude that $\limsup \frac{X_n}{\log n} = 1$ almost surely.
	\end{proof}

	\item Let $M_n = \max_{1\leq k\leq n}X_k$. Show that
	\[
	\limsup_n \frac{M_n}{\log n} = 1\text{ a.s.}
	\]
	\begin{proof}
		Fix $t>0$ and let $E_n$ be the event given by $E_n= \{M_n > t\log n\}$. By L'H\^opital's rule, we have
		\begin{align*}
			\lim_{n\to \infty}\Prob[E_n] &= \lim_{n\to \infty}\Prob[X_k > t\log n\text{ for at least one }1\leq k\leq n]\\
			&= \lim_{n\to \infty}1 - \Prob[X_k \leq t\log n\text{ for each }1\leq k\leq n]\\
			&= \lim_{n\to \infty}1 - (1 - \Prob[X_1 >t\log n])^n\quad \text{(since the $X_k$'s are independent)}\\
			&= \lim_{n\to \infty}1 - \left(1 - \frac{1}{n^t}\right)^n\\
			&= 0 \text{ if and only if } t> 1.
		\end{align*}
		Now let's compute $\Prob[E_n \setminus E_{n-1}]$.
		\begin{align*}
			\Prob[E_n\setminus E_{n-1}] &= \Prob[X_k > t\log n\text{ for at least one }1\leq k \leq n\\
			&\text{ AND }X_k \leq t\log (n-1)\text{ for each }1\leq k\leq n-1].
		\end{align*}
		The only way $X_k \leq t\log (n-1)$ can hold for each $1\leq k\leq n-1$ while still having at least one of $1\leq k\leq n$ satisfy $X_k > t\log n$ is for $X_n >t\log n$ to hold. We then have
		\begin{align*}
			\sum \Prob[E_n \setminus E_{n-1}] &= \sum \Prob[X_n > t\log n]= \sum \frac{1}{n^t}<\infty \text{ if and only if }t > 1.
		\end{align*}
		By problem 1, we then have $\Prob[M_n > t\log n\text{ infinitely often}] = 0$ if and only if $t > 1$. By the same reasoning we used in part (a), we have that $\limsup \frac{M_n}{\log n} = 1$ almost surely.
	\end{proof}
\end{enumerate}

\noindent\textbf{Problem 3. }
Let
\[
\psi(x) = \begin{cases}
	x^2&\text{if }|x|\leq 1\\
	|x|&\text{if }|x|\geq 1
\end{cases}.
\]
Let $X_1, X_2, \ldots$ be independent mean zero random variables. Show that if $\sum \E[\psi(X_n)]<\infty$, then $\sum X_n$ converges almost surely.
\begin{proof}
	% By the linearity of expectation, the monotone convergence theorem, and our hypothesis, we have
	% \[
	% \sum \E[\psi(X_n)] = \E\left[\sum X_n^2\cdot \ind_{|X_n|\leq 1}\right] + \E\left[\sum |X_n|\cdot \ind_{|X_n|\geq 1}\right] <\infty.
	% \]
	% Since $\sum |X_n|\cdot \ind_{|X_n|\geq 1}$ has finite expectation, it must converge almost surely. In particular, we have $\sum X_n \cdot \ind_{|X_n|\geq 1}$ converges almost surely. It remains to show that $\sum X_n \cdot \ind_{|X_n| \leq 1}$ converges a.s. too.\\

	By Markov's inequality we have
	\[
	\sum \Prob[|X_n| > 1] \leq \sum \E[|X_n| \cdot \ind_{|X_n|> 1}] < \infty.
	\]
	By Borel-Cantelli, $|X_n| \leq 1$ eventually almost surely. In other words, $X_n = X_n\cdot \ind_{|X_n| \leq 1}$ eventually almost surely. In particular, we have that $\sum X_n$ converges a.s. if and only if $\sum X_n \cdot \ind_{|X_n|\leq 1}$ converges a.s.\\

	\noindent Now let's look at $\sum \E[X_n \cdot \ind_{|X_n|\leq 1}]$. Since $\E[X_n] = 0$, we have $\E[X_n \cdot \ind_{|X_n|\leq 1}] = -\E[X_n \cdot \ind_{|X_n|>1}]$. This gives
	\begin{align*}
		\sum |\E[X_n\cdot \ind_{|X_n|\leq 1}]| = \sum |\E[X_n \cdot \ind_{|X_n| \geq 1}]| \leq \sum \E[|X_n|\cdot \ind_{|X_n|\geq 1}] \leq \sum \E[\psi(X_n)] < \infty.
	\end{align*}
	The variances of $X_n \cdot \ind_{|X_n|\leq 1}$ are also summable:
	\begin{align*}
		\sum \Var[X_n \cdot \ind_{|X_n|\leq 1}] \leq \sum \E[X_n^2 \cdot \ind_{|X_n| \leq 1}] \leq \sum \E[\psi(X_n)] < \infty.
	\end{align*}
	By Kolmogorov's two-series theorem, we have that $\sum X_n \cdot \ind_{|X_n|\leq 1}$ converges almost surely, which shows that $\sum X_n$ converges almost surely by our earlier discussion.
\end{proof}

\noindent\textbf{Problem 4. }
Construct a sequence of independent mean zero random variables $X_1, X_2, \ldots$ such that
\[
\frac{1}{n}\sum_{k=1}^nX_k\to \infty\text{ a.s.}
\]
Why does this example not contradict the strong law of large numbers?
\begin{solution}
	Consider the probability triplet $(\Omega, \mcal{F}, \Prob)$ where $\Omega$ is the unit interval $[0,1]$, $\mcal{F}$ is the Borel $\sigma$-algebra, and $\Prob$ is the Lebesgue measure. Define the sequence of random variables $X_k$ by
	\[
	X_k = \begin{cases}
		k(1-k),&\text{if }0\leq x\leq \frac{1}{k}\\
		k,&\text{if }\frac{1}{k}<x\leq 1.
	\end{cases}
	\]
	Each $X_k$ has zero mean since
	\[
	\E[X_k] = \frac{1}{k}\cdot k(1-k) + \left(1 - \frac{1}{k}\right)\cdot k = 0.
	\]
	Now for any $\omega\in [0,1]$ we have $\frac{1}{n}< \omega \leq 1$ for all $n$ larger than some $N_\omega$. We then have for $n$ larger than $N_\omega$,
	\[
	\frac{1}{n}\sum_{k=1}^n X_k(\omega) = \frac{1}{n}\left(\sum_{k=1}^{N_\omega}k(k-1) + \sum_{k=N_\omega+1}^nk\right) = \frac{1}{n}(C_\omega + \Omega(n^2)) \to \infty,
	\]
	for some constant $C_\omega$. This doesn't contradict the strong law of large numbers, since the strong law requires that the variables $X_1, X_2, \ldots$ be iid, whereas the variables $X_k$ defined above are not identically distributed.
\end{solution}

\noindent\textbf{Problem 5. }
Suppose disasters occur at random times $X_i$ apart from each other. Precisely, the $k$-th disaster occurs at time $T_k = X_1 + \cdots +X_k$, where the $X_i$ are iid random variables taking positive values with finite mean $\mu$. Let
\[
N(t) = \max\{n: T_n\leq t\}
\]
be the number of disasters that have occurred by time $t$. Prove that
\[
N(t)\to \infty\quad\text{and}\quad \frac{N(t)}{t}\to \frac{1}{\mu}
\]
almost surely as $t\to \infty$.
\begin{proof}
	First, we claim that $N(t)<n$ if and only if $t<T_n$. This is true since
	\begin{align*}
		N(t)<n = \max(m: T_m\leq t)<n \iff T_n > t.
	\end{align*}
	From this, we can deduce that $t<T_{N(t)+1}$ since $N(t)<N(t)+1$. Similarly, since $N(t)\leq N(t)$, we have $T_{N(t)}\leq t$. Putting these together gives
	\begin{equation}\label{compound}
		T_{N(t)}\leq t< T_{N(t)+1}.
	\end{equation}
	Now by the strong law of large numbers we have that $\frac{T_n}{n}\to \mu$ almost surely. Fix $\epsilon>0$. For $n$ sufficiently large, we have that $|\frac{T_n}{n}-\mu| \leq \epsilon$ a.s. From this we deduce that $T_n \leq n(\mu+\epsilon)$ a.s. Since $T_n\leq t$ if and only if $N(t)\geq n$, we have
	\[
	N(n(\mu+\epsilon)) \geq n
	\]
	for $n$ large. Taking $n$ to infinity and using the fact that $N(t)$ is nondecreasing, we have that $N(t)\to \infty$ as $t\to \infty$. Dividing (\ref{compound}) through by $N(t)$ gives
	\[
	\frac{T_{N(t)}}{N(t)} \leq \frac{t}{N(t)} \leq \frac{T_{N(t)+1}}{N(t)} = \frac{T_{N(t)}+X_{N(t)+1}}{N(t)}.
	\]
	By the strong law of large numbers and the fact that $N(t)\to \infty$, we have that $T_{N(t)}/N(t)\to \mu$ a.s. as $t\to \infty$. Since the $X_k$'s are identically distributed with finite mean, we have that $X_{N(t)+1}/N(t)\to 0$ a.s. Both sides of the above inequality then tend to $\mu$ a.s., so $\frac{N(t)}{t}\to \frac{1}{\mu}$ a.s.
\end{proof}

\noindent\textbf{Problem 6. }
Let $X_1, X_2, \ldots$ be independent random variables. Show that $\sum X_n$ converges in probability if and only if $\sum X_n$ converges almost surely.
\begin{proof}
	Almost sure convergence always implies convergence in probability, so it just remains to show the converse. To this end, let $S_n = \sum_{k=1}^n X_k$ be the $n$-th partial sum. Let's show that $S_n$ is Cauchy a.s. Fix $n$ and some $N$ and apply Etemadi's inequality to the variables $X_{n+1}, X_{n+1}, \ldots, X_N$:
	\begin{align*}
		\Prob\left[\max_{n+1\leq m\leq N}|X_{n+1} + \cdots + X_m|>3\epsilon\right] \leq 3\cdot \max_{n+1\leq m\leq N}\Prob[|X_{n+1} + \cdots + X_m| > \epsilon].
	\end{align*}
	Letting $N\to \infty$, we have
	\[
	\Prob\left[\sup_{m>n}|X_{n+1} + \cdots + X_m|>3\epsilon\right] \leq 3\cdot \sup_{m>n}\Prob[|X_{n+1} + \cdots + X_m| > \epsilon].
	\]
	Now since $\sum X_n$ converges in probability, its partial sums are Cauchy in probability. Consequently, as we take $n$ to infinity, the right-hand side of the above inequality tends to zero as $n\to \infty$. We have then shown that $\sup_{m>n}|S_m - S_n| \to 0$ in probability. Since this quantity is decreasing in $n$, it must converge a.s. as well. Since the partial sums are a.s. Cauchy, we have that $\sum X_n$ converges a.s.
\end{proof}

\noindent\textbf{Problem 7. }
Let $X_1, X_2, \ldots$ be iid random variables taking non-negative values, such that $\Prob[X_i>0]>0$. Prove that
\[
\sum X_n  = \infty \text{ a.s.}
\]
\begin{proof}
	Since $\Prob[X_i>0]>0$, we can find $\delta$ and $\epsilon$ both positive such that $\Prob[X_i> \delta]>\epsilon$. We then have
	\[
	\sum \Prob[X_i > \delta] > \sum \epsilon = +\infty.
	\]
	By Borel-Cantelli, we then have that $\Prob[X_i > \delta\text{ infinitely often}] = 1$. Let $A = \{X_i > \delta \text{ infinitely often}\}$. For any $\omega \in A$, we have that $\sum X_n(\omega)$ is a sum that contains infinitely many terms of size at least $\delta$. Since each $X_n$ takes only nonnegative values, this sum must diverge at $\omega$. Since $\Prob[A] = 1$, we have that $\sum X_n = \infty$ almost surely.
\end{proof}


\noindent\textbf{Problem 8. }
Call a number $x\in [0,1]$ badly approximable by rationals if there exists $c(x)>0$ and $\epsilon(x)>0$ such that for any $p,q\in \naturals$ we have
\[
\left|x - \frac{p}{q}\right| >\frac{c}{q^{2+\epsilon}}.
\]
Prove that almost all numbers in $[0,1]$ are badly approximable.
\begin{proof}
	Fix any positive $\epsilon$ and $c$. Let $E_q$ be the set of rationals that are \textit{not} badly approximable by a rational with denominator $q$:
	\[
	E_q = \left\{x\in [0,1]: \left|x - \frac{p}{q}\right| \leq \frac{c}{q^{2+\epsilon}}\text{ for some }0\leq p\leq q\right\} = \bigcup_{p = 0}^q\left\{x\in [0,1]: \left|x - \frac{p}{q}\right| \leq \frac{c}{q^{2+\epsilon}}\right\}.
	\]
	The set of not badly approximable numbers is the union of the $E_q$'s. Let's compute the measure of this set, $E_{c, \epsilon} = \cup_q E_q$
	\begin{align*}
		m(E_{c, \epsilon}) &= \sum_q\sum_{p=0}^qm\left\{x\in [0,1]: \left|x - \frac{p}{q}\right|\leq \frac{c}{q^{2+\epsilon}}\right\}\\
		&\leq \sum_q\sum_{p=0}^q\frac{2c}{q^{2+\epsilon}}\\
		&= 2c\sum_q \frac{q+1}{q^{2+\epsilon}}\\
		&<\infty.
	\end{align*}
	By Borel-Cantelli, we have that for any fixed $\epsilon, c > 0$, almost every $x\in [0,1]$ satisfies $|x - p/q| \leq C/q^{2+\epsilon}$ for only finitely many $p$ and $q$.
\end{proof}


\noindent\textbf{Problem 9. }
Let $X_1, X_2, \ldots$ be iid random variables with finite mean $\mu$. Prove that
\[
\frac{1}{\ln n}\sum_{k=1}^n\frac{X_k}{k}\to \mu\text{ a.s.}
\]
\begin{proof}
	When we proved the strong law of large numbers in class, we crucially used the fact that if a sequence of numbers $x_n$ converges to $x$, then it converges in mean: $\frac{1}{n}\sum_{k=1}^n x_k$ converges to $x$ as well. In this problem, we'll show that if $x_n\to x$, then $\frac{1}{\ln n}\sum_{k=1}^n \frac{x_k}{k}\to x$. This fact along with minor modifications to our proof from class will establish the claim.\\

	\noindent From the monotonicity of $f(x) = \frac{1}{x}$, we have that
	\[
	\ln n + \frac{1}{n} \leq \sum_{k=1}^n \frac{1}{k}\leq \ln n + 1.
	\]
	Dividing through by $\ln n$ and taking the limit establishes
	\[
	\frac{1}{\ln n}\sum_{k=1}^n \frac{1}{k} \to 1.
	\]
	\noindent Now suppose a sequence of real numbers $x_n$ converges to $x$. By the above reasoning, we have
	\[
	\frac{1}{\ln n}\sum_{k=1}^n \frac{x}{k} = x\left(\frac{1}{\ln n}\sum_{k=1}^n\frac{1}{k}\right)\to x.
	\]
	Fix $\epsilon>0$ and choose $N$ such that $|x_n - x|<\epsilon$ for all $n > N$. If $n> N$ we have
	\begin{align*}
		\left|\frac{1}{\ln n}\sum_{k=1}^n\frac{x_k}{k} - \frac{1}{\ln n}\sum_{k=1}^n\frac{x}{k}\right| &\leq \frac{1}{\ln n}\sum_{k = 1}^N\frac{|x_k - x|}{k} + \frac{1}{\ln n}\sum_{k=N+1}^n\frac{|x_k-x|}{k}\\
		&\leq \frac{1}{\ln n}\sum_{k=1}^N\frac{|x_k-x|}{k} + \epsilon \cdot \frac{1}{\ln n}\sum_{k=1}^n\frac{1}{k}.
	\end{align*}
	The first sum has only finitely many terms in it, so as $n\to \infty$ it vanishes. The second sum approaches $\epsilon$ as $n$ tends to infinity. We've then shown that the terms of $\frac{1}{\ln n}\sum_{k=1}^n\frac{x_k}{k}$ and $\frac{1}{\ln n}\sum_{k=1}^n\frac{x}{k}$ become arbitrarily close to one another. Since the latter sequence limits to $x$, we must then have
	\begin{equation}\label{harm_avg}
	\frac{1}{\ln n }\sum_{k=1}^n\frac{x_k}{k}\to x.
	\end{equation}

	\noindent Returning to the problem at hand, by splitting $X = X^+ - X^-$, we can assume without loss of generality that $X_k \geq 0$. Let's start by truncating the $X_k$'s and define $Y_k = X_k\cdot \ind_{\{|X_k|\leq k\}}$ and $T_n = \sum_{k=1}^n Y_k/k$. We claim that it suffices to prove that $T_n/\ln n \to \mu $ a.s. The idea is that $X_k = Y_k$ eventually almost surely. To see this, consider the sum
	\[
	\sum \Prob[|X_k| > k] \leq \int_0^\infty \Prob[|X_1|>t]\ dt = \E[|X_1|]<\infty.
	\]
	By Borel-Cantelli, we have that $|X_k|\leq k$ eventually almost surely. If $|X_k|\leq k$, then by definition we have $X_k = Y_k$. Consequently, the difference $\sum_{k=1}^n \frac{1}{k}|X_k(\omega) - Y_k(\omega)|<C_\omega<\infty$ for all $n$.\\

	\noindent Let $\epsilon>0$ be arbitrary and let $k(n) = \lceil 2^{(1+\epsilon)^n}\rceil$. Let $\delta>0$. By Chebyshev's inequality we have
	\begin{align*}
		\sum_{n = 1}^\infty \Prob[|T_{k(n)} - \E[T_{k(n)}]| > \delta \ln k(n)] &\leq \frac{1}{\delta^2}\sum_n \frac{\Var[T_{k(n)}]}{(1+\epsilon)^{2n}}\\
		&= \frac{1}{\delta^2}\sum_n (1+\epsilon)^{-2n}\sum_{m = 1}^{k(n)}\Var\left[\frac{Y_m}{m}\right]\\
		&= \frac{1}{\delta^2}\sum_{m=1}^\infty \frac{1}{m^2}\Var[Y_m]\sum_{n: k(n)>m}(1+\epsilon)^{-2n}\\
		&\leq \frac{C_\epsilon}{\delta^2}\sum_m\frac{1}{m^2}\Var[Y_m].
	\end{align*}
	Now we showed in class that $\sum \Var[Y_m]/m^2<\infty$, so the above sum is finite. Since $\delta$ was arbitrary, we have that
	\begin{equation}\label{diff}
	\frac{T_{k(n)} - \E[T_{k(n)}]}{\ln k(n)}\to 0\text{ a.s.}
	\end{equation}
	By dominated convergence we have that $\E[Y_k] \to \E[X_1] = \mu$. By our earlier discussion and (\ref{harm_avg}), we have
	\[
	\frac{\E[T_{k(n)}]}{\ln k(n)} = \frac{1}{\ln k(n)}\sum_{k = 1}^{k(n)}\frac{\E[Y_k]}{k} \to \mu.
	\]
	By (\ref{diff}), we must have that $\frac{T_{k(n)}}{\ln k(n)}\to \mu$ a.s. Now we interpolate: suppose $k(n)\leq m< k(n+1)$. Since we're assuming that $Y_k\geq 0$ for all $k$, we have
	\begin{align}
	&\frac{T_{k(n)}}{\ln k(n+1)} \leq \frac{T_m}{\ln m}\leq \frac{T_{k(n+1)}}{\ln k(n)}.
	\end{align}
	Now $\ln k(n+1)/ \ln k(n) \to (1+\epsilon)$, so after taking limits we have
	\[
	\frac{1}{1+\epsilon}\mu \leq \liminf_{m\to \infty}\frac{T_m}{\ln m}\leq \limsup_{m\to \infty}\frac{T_m}{\ln m}\leq (1+\epsilon)\mu.
	\]
	Taking $\epsilon\to 0$ establishes $\frac{T_m}{\ln m}\to \mu$. By our previous discussion, this implies that $\frac{1}{\ln n}\sum_{k=1}^n\frac{X_k}{k}\to \mu$ a.s.
\end{proof}


\end{document}