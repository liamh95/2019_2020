\documentclass[11pt,letterpaper]{report}
\usepackage{amssymb,amsfonts,color,graphicx,amsmath,enumerate}
\usepackage{amsthm}

\newcommand{\naturals}{\mathbb{N}}
\newcommand{\integers}{\mathbb{Z}}
\newcommand{\complex}{\mathbb{C}}
\newcommand{\reals}{\mathbb{R}}
\newcommand{\mcal}[1]{\mathcal{#1}}
\newcommand{\rationals}{\mathbb{Q}}
\newcommand{\field}{\mathbb{F}}
\newcommand{\Var}{\text{Var}}
\newcommand{\ind}{\mathbbm{1}}
\newcommand{\Cov}{\text{Cov}}

\newenvironment{solution}
{\begin{proof}[Solution]}
{\end{proof}}

\voffset=-3cm
\hoffset=-2.25cm
\textheight=24cm
\textwidth=17.25cm
\addtolength{\jot}{8pt}
\linespread{1.3}

\begin{document}
\begin{center}
{\bf \Large Math 130B - Concentration Inequalities}
\vspace{0.2cm}
\hrule
\end{center}

\begin{enumerate}
	\item A biased coin lands heads with probability $1/10$. The coin is flipped 200 times. Use Markov's inequality to give an upper bound on the probability that the coin shows heads at least 120 times. Improve this bound using Chebyshev's inequality. Improve it even further with Chernoff's inequality.

	\vfill

	\item The average height of an anteater is 10 inches.
	\begin{enumerate}
		\item Give an upper bound on the probability that a certain raccoon is at least 15 inches tall.
		\vfill
		\item The standard deviation for this height distribution is 2 inches. Find a lower bound on the probability that a certain raccoon is between 5 and 15 inches tall.
		\vfill
		\item Now assume that this distribution is normal. Use a normal table to repeat the calculation from part (b). How close was your bound to the true probability?
	\end{enumerate}
	\vfill

	\item Consider the random graph $G \sim \mcal{G}(n, 1/2)$, which has $n$ vertices and any two vertices are connected by an edge with probability $n=1/2$.
	\begin{enumerate}
		\item Show that the average degree of a vertex is $\frac{n-1}{2}\approx \frac{n}{2}$.
		\vfill
		\item Let $d$ be the degree of a vertex in $G$. Use Chebyshev's inequality to bound the probability
		\[
		\Pr\big[|d - n/2| \geq 3\sqrt{n}\log n\big].
		\]
		\vfill
		\item Do part (b) with Chernoff instead. Conclude that vertices in $\mcal{G}(n, 1/2)$ have degree $n/2$ with high probability.
		\vfill
	\end{enumerate}
\end{enumerate}

\end{document}