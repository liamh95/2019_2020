\documentclass[12pt]{article}  
\pagestyle{empty}

\usepackage{graphics}
\usepackage{amsmath,amssymb,amsthm, multicol}
\usepackage[pdftex]{graphicx}
\usepackage{epsf}

\newcommand{\naturals}{\mathbb{N}}
\newcommand{\integers}{\mathbb{Z}}
\newcommand{\complex}{\mathbb{C}}
\newcommand{\reals}{\mathbb{R}}
\newcommand{\mcal}[1]{\mathcal{#1}}
\newcommand{\rationals}{\mathbb{Q}}
\newcommand{\Aut}{\text{Aut}}
\newcommand{\Lp}[2]{\left\|{#1}\right\|_{L^{#2}}}
\newcommand{\tr}{\text{tr}}
\newcommand{\field}{\mathbb{F}}

\addtolength{\oddsidemargin}{-.75in}
\addtolength{\evensidemargin}{-.75in}
\addtolength{\textwidth}{1.5in}
\addtolength{\topmargin}{-1in}
\addtolength{\textheight}{2.25in}

\begin{document}
\begin{flushleft} 
\centerline{\LARGE{Quiz 4}} 
\vspace{5 mm}
{Student ID Number:}\hfill  
{Name \rule {2 in}{0.01in}}\\
Math 180B, 3PM
\\
{Please justify all your answers}\hfill {May 22 - May 24}
\\
{Please also write your full name on the back} 

\medskip
\end{flushleft}

\begin{enumerate}
	\item Let $E$ be the elliptic curve
	\[
	E: y^2 = x^3 + x + 1.
	\]
	Compute the number of points on $E$ over $\integers/7\integers$. Calculate the defect $a_5 = p - N_5$.

	\vfill

	\item Let $E$ be the elliptic curve $E: y^2 = x^3 - 2x + 4$ and let $P = (0, 2)$ and $Q = (3, -5)$ be two points on $E$. The line $\ell$ through $P$ and $Q$ intersects $E$ in a third point. Find this third point.

	\vfill
\end{enumerate}

\end{document}