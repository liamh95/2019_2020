\documentclass[11pt,letterpaper]{report}
\usepackage{amssymb,amsfonts,graphicx,amsmath,enumerate}
\usepackage{amsthm}

\newcommand{\naturals}{\mathbb{N}}
\newcommand{\integers}{\mathbb{Z}}
\newcommand{\complex}{\mathbb{C}}
\newcommand{\reals}{\mathbb{R}}
\newcommand{\mcal}[1]{\mathcal{#1}}
\newcommand{\rationals}{\mathbb{Q}}
\newcommand{\field}{\mathbb{F}}
\newcommand{\Var}{\text{Var}}
\newcommand{\Cov}{\text{Cov}}

\newenvironment{solution}
{\begin{proof}[Solution]}
{\end{proof}}

\theoremstyle{definition}
\newtheorem{theorem}{Theorem}

\voffset=-3cm
\hoffset=-2.25cm
\textheight=24cm
\textwidth=17.25cm
\addtolength{\jot}{8pt}
\linespread{1.3}

\begin{document}
\noindent{\em Name\hfill{Date} }
\begin{center}
{\bf \Large Class - Homework N}
\vspace{0.2cm}
\hrule
\end{center}

\section*{Problem 1}
Prove the following theorem.
\begin{theorem}[Markov's Inequality]
	If $X:\Omega\to \reals$ is a random variable and $E[|X|]<\infty$, then for all $a\geq 0$
	\begin{equation}\label{Markov}
		\Pr[|X|\geq a] \leq \frac{E[|X|]}{a}.
	\end{equation}
\end{theorem}

\begin{solution}
	Define the random variable $Y$ by
	\[
	Y(\omega) = \begin{cases}
		0,&\text{if }|X(\omega)|<a\\
		a,&\text{if }|X(\omega)|\geq a
	\end{cases}.
	\]
	Note that $Y(\omega)\leq |X(\omega)|$ for all $\omega\in \Omega$. This gives
	\begin{align*}
		E[|X|] &\geq E[Y]\\
		&= a\cdot \Pr[|X|\geq a].
	\end{align*}
	Dividing through by $a$ establishes (\ref{Markov}).
\end{solution}


\section*{Problem 2}
Prove that
\begin{equation}\label{Gauss}
	1 + 2 + \cdots + n = \frac{n(n+1)}{2}.
\end{equation}

\begin{solution}
	You could prove this by induction, but that's pretty boring. Check out this argument allegedly due to Gauss. Call the left-hand side of (\ref{Gauss}) $S$ then write it forward and backwards.
	\begin{align*}
		S &= 1 + 2 + 3 + \cdots + n\\
		S &= n + (n-1) + (n-2) + \cdots +1
	\end{align*}
	The first terms in both rows sum to $n+1$, the second terms in both rows also sum to $n+1$, and so on. There are $n$ terms in each row, so adding these rows together gives $n(n+1)$ on the right-hand side and $2S$ on the left. Dividing by 2 establishes (\ref{Gauss}).
\end{solution}

\end{document}