\documentclass[11pt,letterpaper]{report}
\usepackage{amssymb,amsfonts,color,graphicx,amsmath,enumerate}
\usepackage{amsthm}

\newcommand{\naturals}{\mathbb{N}}
\newcommand{\integers}{\mathbb{Z}}
\newcommand{\complex}{\mathbb{C}}
\newcommand{\reals}{\mathbb{R}}
\newcommand{\mcal}[1]{\mathcal{#1}}
\newcommand{\rationals}{\mathbb{Q}}
\newcommand{\field}{\mathbb{F}}
\newcommand{\Var}{\text{Var}}
\newcommand{\ind}{\mathbbm{1}}
\newcommand{\Cov}{\text{Cov}}

\newenvironment{solution}
{\begin{proof}[Solution]}
{\end{proof}}

\voffset=-3cm
\hoffset=-2.25cm
\textheight=24cm
\textwidth=17.25cm
\addtolength{\jot}{8pt}
\linespread{1.3}

\begin{document}
\begin{center}
{\bf \Large Math 130B - Homework 1}
\vspace{0.2cm}
\hrule
\end{center}

\begin{enumerate}



	\item Find an example of a discrete random variable with finite expectation and infinite variance.

	\item \begin{enumerate}
		\item Prove from the definition that if $\lim_{x\to \infty}\frac{|f(x)|}{|g(x)|}<\infty$ then $f(x) = O(g(x))$.
		\item Prove that
		\[
		\lim_{x\to \infty} \frac{(x+2)\cos^2(x)}{x}
		\]
		does not exist but $(x+2)\cos^2(x) = O(x)$. Deduce that the converse of part (a) is false.
	\end{enumerate}

	\item Prove or give a counterexample. If $E_1$ and $E_2$ are independent, then they are conditionally independent given $F$.

	\item Suppose you are given a polynomial $P(x)$ and its degree is unknown. There is an oracle that can give you the value of the polynomial at any value that you wish. In addition, suppose that all the coefficients of the polynomial are non-negative integers. Show that you can recover the polynomial with only two queries to the oracle.

	\item Suppose that the sphere $S^2 = \{x\in \reals^3: \|x\| = 1\}$ is colored red and blue such that 90\% of its surface area is blue and 10\% is red. Show that one can inscribe a cube in $S$ all of whose vertices are blue.

	\item Consider the vector $v^{(1)} = (1, 2, \ldots, n)$. Now start the following process. Pick a random position $i_1\in \{1, 2, \ldots, n\}$ and swap $v^{(1)}_{i_1}$ and $v^{(1)}_n$ to obtain the new vector $v^{(2)}$. Now pick another random $i_2\in \{1, 2, \ldots, n-1\}$ and swap $v^{(2)}_{i_2}$ and $v^{(2)}_{n-1}$ to obtain $v^{(3)}$. Repeat this until we obtain the vector $v^{(n)}$.\\

	\noindent Show that the vector $v^{(n)}$ is equally likely to be any of the $n!$ permutations of $\{1, 2, \ldots, n\}$. (This example will help us explain card shuffling arguments later.)

\end{enumerate}

\end{document}