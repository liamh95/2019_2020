\documentclass[11pt,letterpaper]{report}
\usepackage{amssymb,amsfonts,color,graphicx,amsmath,enumerate}
\usepackage{tikz}
\usepackage{amsthm}
\usepackage{bbm}

\newcommand{\naturals}{\mathbb{N}}
\newcommand{\integers}{\mathbb{Z}}
\newcommand{\complex}{\mathbb{C}}
\newcommand{\reals}{\mathbb{R}}
\newcommand{\exreals}{\overline{\mathbb{R}}}
\newcommand{\mcal}[1]{\mathcal{#1}}
\newcommand{\mable}{measurable}
\newcommand{\quats}{\mathbb{H}}
\newcommand{\rationals}{\mathbb{Q}}
\newcommand{\norm}{\trianglelefteq}
\newcommand{\Aut}{\text{Aut}}
\newcommand{\disk}{\mathbb{D}}
\newcommand{\halfplane}{\mathbb{H}}
\newcommand{\Lp}[2]{\left\|{#1}\right\|_{L^{#2}}}
\newcommand{\supp}[1]{\text{supp}({#1})}
\newcommand{\Hom}[2]{\text{Hom}_{{#1}}({#2})}
\newcommand{\tr}{\text{tr}}
\newcommand{\field}{\mathbb{F}}
\newcommand{\Gal}[1]{\text{Gal}\left({#1}\right)}
\newcommand{\esssup}{\text{ess sup }}
\newcommand{\essinf}{\text{ess inf }}
\newcommand{\affine}{\mathbb{A}}
\newcommand{\E}{\mathbb{E}}
\newcommand{\Prob}{\mathbb{P}}
\newcommand{\Var}{\text{Var}}
\newcommand{\ind}{\mathbbm{1}}
\newcommand{\Cov}{\text{Cov}}

\newenvironment{solution}
{\begin{proof}[Solution]}
{\end{proof}}

\voffset=-3cm
\hoffset=-2.25cm
\textheight=24cm
\textwidth=17.25cm
\addtolength{\jot}{8pt}
\linespread{1.3}

\begin{document}
\noindent{\em Liam Hardiman\hfill{November 27, 2019} }
\begin{center}
{\bf \Large 271A - Homework 5}
\vspace{0.2cm}
\hrule
\end{center}

\noindent\textbf{Problem 1. }
Consider the space $\reals^d$ and the usual $\|\cdot \|_2$ metric. Show explicitly that a probability measure $\Prob$ on the measurable space $(\reals^d, \mcal{B}(\reals^d))$ is uniquely determined by
\[
F(x_1, \ldots, x_d) = \Prob[y: y_1\leq x_1, \ldots, y_d\leq x_d].
\]
\begin{proof}
	Our strategy is to use the $\pi$-$\lambda$ theorem. Suppose $\Prob_1$ and $\Prob_2$ are two probability measures that agree on sets of the form $\{y_1 \leq x_1, \ldots, y_d\leq x_d)$. Define the collection $\Pi$ by
	\[
	\Pi = \big\{ \{y: y_1\leq x_1, \ldots, y_d\leq x_d\}: x\in \reals^d\big\}.
	\]
	That is, $\Pi$ consists of all products of rays. As our notation suggests, $\Pi$ is a $\pi$-system since it is clearly nonempty and the intersection of any two products of rays is again a product of rays. We also define the collection $\Lambda$ to be the sets in $\sigma(\Pi)$, the $\sigma$-algebra generated by $\Pi$, on which $\Prob_1$ and $\Prob_2$ agree:
	\[
	\Lambda = \{E \in \sigma(\Pi): \Prob_1(E) = \Prob_2(E)\}.
	\]
	This collection is indeed well-defined since every set in $\Pi$ is a Borel set, so $\sigma(\Pi)\subseteq \mcal{B}(\reals^d)$ and each $E\in \sigma(\Pi)$ is $\Prob_1$ and $\Prob_2$ measurable. We again claim that our notation makes sense and that $\Lambda$ is a $\lambda$-system. Let's verify this claim.
	\begin{itemize}
		\item $\reals^d\in \Lambda$: We can write $\reals^d$ as a union of ray-products, $\reals^d = \bigcup_{n=1}^\infty (-\infty, n]^d$, so $\reals^d$ is indeed in $\sigma(\Pi)$. That $\Prob_1[\reals^d] = \Prob_2[\reals^d]$ follows from the fact that $\Prob_1$ and $\Prob_2$ are probability measures.

		\item Closure under complements: Let $E\in \Lambda$. Since $\Prob_1$ an d
	\end{itemize}
\end{proof}

\end{document}