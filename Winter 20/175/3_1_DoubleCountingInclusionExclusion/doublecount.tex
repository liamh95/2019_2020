\documentclass[12pt]{article}  

\pagestyle{empty}

\usepackage{graphics}
\usepackage{amsmath,amssymb,amsthm, multicol}
\usepackage[pdftex]{graphicx}
\usepackage{epsf}

\newcommand{\naturals}{\mathbb{N}}
\newcommand{\integers}{\mathbb{Z}}
\newcommand{\complex}{\mathbb{C}}
\newcommand{\reals}{\mathbb{R}}
\newcommand{\mcal}[1]{\mathcal{#1}}
\newcommand{\rationals}{\mathbb{Q}}
\newcommand{\Aut}{\text{Aut}}
\newcommand{\Lp}[2]{\left\|{#1}\right\|_{L^{#2}}}
\newcommand{\tr}{\text{tr}}
\newcommand{\field}{\mathbb{F}}

\addtolength{\oddsidemargin}{-.75in}
\addtolength{\evensidemargin}{-.75in}
\addtolength{\textwidth}{1.5in}
\addtolength{\topmargin}{-1in}
\addtolength{\textheight}{2.25in}

\begin{document}
\begin{center}
{\bf \Large Double Counting and Inclusion-Exclusion}
\vspace{0.2cm}
\hrule
\end{center}

\begin{enumerate}
	\item Use the following steps to show that the sequence
	\[
	\gamma_n = 1 + \frac{1}{2} + \frac{1}{3} + \cdots + \frac{1}{n} - \log n
	\]
	has a limit.
	\begin{enumerate}
		\item Sketch a graph for $f(x) = \frac{1}{x}$ and interpret $\gamma_n$ as an area to show that $\gamma_n > 0$ for all $n$.
		\vfill
		\item Interpret
		\[
		\gamma_n - \gamma_{n+1} = [\log(n+1) - \log n] - \frac{1}{n+1}
		\]
		as a difference of areas to show that $\gamma_n$ is a decreasing sequence.
		\vfill
		\item Deduce that $\lim_{n\to \infty}\gamma_n$ converges.
	\end{enumerate}
	\vfill
	\item Use your proof in exercise 1 to finish the proof from lecture about the divisor counting function: if we let $t(n)$ be the number of divisors of $n$ and define $\tau(n) = \frac{1}{n}\sum_{k=1}^nt(k)$, then
	\[
	|\tau(n)-\log n| \leq 1.
	\]
	\vfill

	\item Prove that $(x+y)^p \equiv x^p+y^p \pmod{p}$, where $p$ is a prime.
	\vfill

	\item Let $n$ and $k$ be positive integers and let $S$ be a set of $n$ points in the plane such that no three points of $S$ are colinear and for every point $P$ of $S$ there are at least $k$ points of $S$ equidistant from $P$. Use these steps to prove that
	\[
	k < \frac{1}{2} + \sqrt{2n}.
	\]
	\begin{enumerate}
		\item Fix a point $P$ in $S$. At \textit{least} how many isosceles triangles in $S$ have $P$ as their apex?
		\item Fix two points $P$ and $Q$ in $S$. At \textit{most} how many isosceles triangles in $S$ contain points $P$ and $Q$? \textit{Hint: Consider the perpendicular bisector of $A$ and $B$.}
		\item Use your answers to (a) and (b) to do a double-counting argument.
	\end{enumerate}
	\vfill\null\pagebreak

	\item How many positive integers below 100 are divisible by 2, 3, or 5?
	\vfill
	\item \begin{enumerate}
		\item A class has 20 students. How many ways are there to form a study group, assuming a study group must have at least two members?
		\vfill
		\item How many ways are there to form a study group that contains at least one of Alice, Bob, and Carol?
	\end{enumerate}
	\vfill

	\item At a 2020 Tokyo Olympics press conference, there are 200 journalists. 175 of them speak Japanese, 150 of them speak German, 180 of them speak English, and 160 of them speak Mandarin. What is the minimum number of journalists that can speak all four languages?
	\vfill

	\item There are 50 students in a class who are given a test with three questions on it. All of the students answer at least one question. If 12 students did not answer question 1, 14 didn't answer question 2, 10 didn't answer question 3, and 25 answered all three questions, then how many students answered exactly 1 question?
	\vfill
\end{enumerate}

\end{document}