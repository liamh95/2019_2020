\documentclass[11pt,letterpaper]{report}
\usepackage{amssymb,amsfonts,color,graphicx,amsmath,enumerate}
\usepackage{tikz}
\usepackage{amsthm}
\usepackage{bbm}
\usepackage{hyperref}

\newcommand{\naturals}{\mathbb{N}}
\newcommand{\integers}{\mathbb{Z}}
\newcommand{\complex}{\mathbb{C}}
\newcommand{\reals}{\mathbb{R}}
\newcommand{\mcal}[1]{\mathcal{#1}}
\newcommand{\rationals}{\mathbb{Q}}
\newcommand{\Lp}[2]{\left\|{#1}\right\|_{L^{#2}}}
\newcommand{\field}{\mathbb{F}}
\newcommand{\affine}{\mathbb{A}}
\newcommand{\E}{\mathbb{E}}
\newcommand{\Prob}{\mathbb{P}}
\newcommand{\Var}{\text{Var}}
\newcommand{\ind}{\mathbbm{1}}
\newcommand{\Cov}{\text{Cov}}

\newenvironment{solution}
{\begin{proof}[Solution]}
{\end{proof}}

\voffset=-3cm
\hoffset=-2.25cm
\textheight=24cm
\textwidth=17.25cm
\addtolength{\jot}{8pt}
\linespread{1.3}

\begin{document}
\begin{center}
{\bf \Large Math 175 - Homework 6}
\vspace{0.2cm}
\hrule
\end{center}

\begin{enumerate}
	\item \begin{enumerate}[(a)]
		\item Is there a graph on six vertices with degrees 2, 3, 3, 3, 3, 3?
		\item Is there a graph on seven vertices with degrees 6, 6, 5, 5, 4, 2, 1?
		\item Is there a graph on five vertices with degrees 0, 1, 2, 3, 4?
	\end{enumerate}

	\vfill

	\item \begin{enumerate}
		\item We delete an edge from a connected graph $G$. Show by an example that the remaining graph might not be connected.
		\item Prove that if we assume that the deleted edge belongs to a cycle in $G$, then the remaining graph is connected.
	\end{enumerate}
	\vfill

	\item Let $G$ be a graph with 100 vertices and minimal degree 6. Prove that $G$ must contain a $C_3$, $C_4$, or $C_5$. 
	\vfill

	\item Let $G$ be a connected graph in which ever pair of edges has an endpoint in common. Show that $G$ is either a star or a triangle. A \textit{star} on $n$ vertices is formed by taking the empty graph on $n$ vertices and connecting one vertex to all of the others.
	\vfill

	\item \begin{enumerate}
		\item Let $a_m$ be the number of cycle graphs on the vertex set $[m]$. Show that $a_0 = a_1 = a_2 = 0$ and for $m\geq 3$, $a_m = \frac{1}{2}(m-1)!$.
		\item Find the function whose power series is given by $\sum_{n=0}^\infty \frac{a_n}{n!}x^n$. This is called an \textit{exponential generating function}. 
	\end{enumerate}

	\vfill
\end{enumerate}

\end{document}