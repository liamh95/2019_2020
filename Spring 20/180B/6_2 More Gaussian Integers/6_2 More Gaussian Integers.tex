\documentclass[11pt,letterpaper]{report}
\usepackage{amssymb,amsfonts,color,graphicx,amsmath,enumerate}
\usepackage{amsthm}

\newcommand{\naturals}{\mathbb{N}}
\newcommand{\integers}{\mathbb{Z}}
\newcommand{\complex}{\mathbb{C}}
\newcommand{\reals}{\mathbb{R}}
\newcommand{\mcal}[1]{\mathcal{#1}}
\newcommand{\rationals}{\mathbb{Q}}
\newcommand{\field}{\mathbb{F}}
\newcommand{\Var}{\text{Var}}
\newcommand{\ind}{\mathbbm{1}}
\newcommand{\Cov}{\text{Cov}}

\newenvironment{solution}
{\begin{proof}[Solution]}
{\end{proof}}

\voffset=-3cm
\hoffset=-2.25cm
\textheight=24cm
\textwidth=17.25cm
\addtolength{\jot}{8pt}
\linespread{1.3}

\begin{document}
\begin{center}
{\bf \Large Math 180B - More on Gaussian Integers}
\vspace{0.2cm}
\hrule
\end{center}

\begin{enumerate}
	\item Find a greatest common divisor for each of the following pairs of Gaussian integers.
	\begin{enumerate}
		\item $\alpha = 8+38i$ and $\beta = 9+59i$
		\item $\alpha = -9+19i$ and $\beta= -19+4i$
	\end{enumerate}

	\vfill

	\item Let $R$ be the following set of complex numbers:
	\[
	R = \{a+bi\sqrt{5}: a,b\in \integers\}.
	\]
	\begin{enumerate}
		\item Verify that $R$ is a ring.
		\vfill
		\item Show that the only units in $R$ are $1$ and $-1$.
		\vfill
		\item Recall that an element $\alpha$ of $R$ is irreducible if and only if its only divisors in $R$ are units and unit multiples of $\alpha$. Prove that 2 is an irreducible element of $R$.
		\vfill
		\item Define the norm of an element $\alpha = a+bi\sqrt{5}$ to be $N(\alpha) = \alpha\bar{\alpha} = a^2+5b^2$. Let $\alpha = 11+2i\sqrt{5}$ and $\beta = 1+i\sqrt{5}$. Show that it is not possible to find elements $\gamma$ and $\rho$ in $R$ satisfying
		\[
		\alpha = \beta\gamma + \rho\quad\text{and}\quad N(\rho)<N(\beta).
		\]
		Thus, $R$ does not have the division with remainder property.

		\vfill

		\item Show that 2 does not divide either factor in the product.
		\[
		(1+i\sqrt{5})(1-i\sqrt{5}) = 6.
		\]
		Conclude that 2 is not prime in $R$.

		\vfill

		\item Conclude that $R$ does not have the unique factorization property
		\vfill
	\end{enumerate}
\end{enumerate}

\end{document}