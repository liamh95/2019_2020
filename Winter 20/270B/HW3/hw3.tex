\documentclass[11pt,letterpaper]{report}
\usepackage{amssymb,amsfonts,color,graphicx,amsmath,enumerate}
\usepackage{tikz}
\usepackage{amsthm}
\usepackage{bbm}
\usepackage{hyperref}

\newcommand{\naturals}{\mathbb{N}}
\newcommand{\integers}{\mathbb{Z}}
\newcommand{\complex}{\mathbb{C}}
\newcommand{\reals}{\mathbb{R}}
\newcommand{\mcal}[1]{\mathcal{#1}}
\newcommand{\rationals}{\mathbb{Q}}
\newcommand{\Lp}[2]{\left\|{#1}\right\|_{L^{#2}}}
\newcommand{\field}{\mathbb{F}}
\newcommand{\affine}{\mathbb{A}}
\newcommand{\E}{\mathbb{E}}
\newcommand{\Prob}{\mathbb{P}}
\newcommand{\Var}{\text{Var}}
\newcommand{\ind}{\mathbbm{1}}
\newcommand{\Cov}{\text{Cov}}

\newenvironment{solution}
{\begin{proof}[Solution]}
{\end{proof}}

\voffset=-3cm
\hoffset=-2.25cm
\textheight=24cm
\textwidth=17.25cm
\addtolength{\jot}{8pt}
\linespread{1.3}

\begin{document}
\noindent{\em Liam Hardiman\hfill{February 24, 2020} }
\begin{center}
{\bf \Large 270B - Homework 3}
\vspace{0.2cm}
\hrule
\end{center}

\noindent\textbf{Problem 1. }
Let $X_1, X_2, \ldots$ be independent random variables with means $\mu_i$ and finite variances $\sigma_i^2$. Consider the sums $S_n = X_1 + \cdots + X_n$. Find sequences of real numbers $(b_i)$ and $(c_i)$ such that $S_n^2 + b_nS_n + c_n$ is a martingale with respect to the $\sigma$-algebras generated by $X_1, \ldots, X_n$.
\begin{solution}
	Let's start by centering the sum: define the random variable $M_n = S_n - \sum_{i=1}^n \mu_i$. Since the $X_i$'s are independent, we have $\Var[M_n] = \sum_{i=1}^n \sigma_i^2$. We claim that
	\[
	V_n = M_n^2 - \sum_{i=1}^n\sigma_i^2 = \left(S_n - \sum_{i=1}^n\mu_i\right)^2 - \sum_{i=1}^n\sigma_i^2
	\]
	is a martingale with respect to the filtration $\mcal{F}_n = \sigma(X_1, \ldots, X_n)$. Let's start the computation.
	\begin{align*}
		\E[V_{n+1}|\mcal{F}_n] &= \E[S_{n+1}^2] - 2\left(\sum_{i=1}^{n+1}\mu_i\right)\E[S_{n+1}|\mcal{F}_n] + \left(\sum_{i=1}^{n+1}\mu_i\right)^2 - \sum_{i=1}^{n+1}\sigma_i^2\\
		&= S_n^2 + 2S_n\mu_{n+1} + \E[X_{n+1}^2] - 2\left(\sum_{i=1}^{n+1}\mu_i\right)(S_n + \mu_{n+1}) + \left(\sum_{i=1}^{n+1}\mu_i\right)^2 - \sum_{i=1}^{n+1}\sigma_i^2\\
		&= S_n^2 -2\left(\sum_{i=1}^n\mu_i\right)S_n + \E[X_{n+1}^2] - 2\mu_{n+1}^2 + \left(\sum_{i=1}^n\mu_i\right)^2 + \mu_{n+1}^2 - \sum_{i=1}^{n+1}\sigma_i^2\\
		&= S_n^2 - 2\left(\sum_{i=1}^n\mu_i\right)S_n + \left(\sum_{i=1}^n\mu_i\right)^2 - \sum_{i=1}^n\sigma_i^2\\
		&= V_n.
	\end{align*}
	Here we've used the fact that $S_n$ is $\mcal{F}_n$-measurable and $X_{n+1}$ is independent of $\mcal{F}_n$. The sequences we want are then
	\[
	b_n = -2\sum_{i=1}^n\mu_i,\qquad c_n = \left(\sum_{i=1}^n\mu_i\right)^2 - \sum_{i=1}^n\sigma_i^2.
	\]
\end{solution}

\noindent\textbf{Problem 2. }
\begin{enumerate}[(a)]
	\item Show that if $(X_n)$ and $(Y_n)$ are martingales with respect to the same filtration, then $X_n\lor Y_n$ is a submartingale.
	\begin{proof}
		We use the trusty identity
		\[
		X_n\lor Y_n = \frac{1}{2}[(X_n+Y_n) + |X_n - Y_n|].
		\]
		Since the sum of martingales is a martingale and conditional Jensen says the absolute value of a martingale is a submartingale, we have
		\begin{align*}
			\E[X_{n+1}\lor Y_{n+1}|\mcal{F}_n] &= \frac{1}{2}(\E[X_{n+1}+Y_{n+1}|\mcal{F}_n] + \E[|X_{n+1} -Y_{n+1}|\ |\ \mcal{F}_n])\\
			&\geq \frac{1}{2}[(X_n + Y_n) + |X_n - Y_n|]\\
			&= X_n\lor Y_n.
		\end{align*}
		Hence, $X_n\lor Y_n$ is a submartingale.
	\end{proof}

	\item Give an example showing that $X_n \lor Y_n$ need not be a martingale.
	\begin{proof}
		
	\end{proof}
\end{enumerate}

\noindent\textbf{Problem 3. }
Give an example of a martingale $(X_n)$ such that $X_n\to -\infty$ a.s.
\begin{solution}
	Durrett gives a hint to let $X_n = \xi_1 + \cdots + \xi_n$ for some independent centered $\xi_i$'s. The idea is to concentrate most of the mass of $\xi_i$ around some negative value and put the rest (some summable amount) around some positive value, then apply Borel-Cantelli.\\

	\noindent Concretely, let $\xi_i$ be given by
	\[
	\xi_i = \begin{cases}
		2^j&\text{with probability }\frac{1}{2^j}\\
		-\frac{1}{1-2^{-j}}&\text{with probability }1-\frac{1}{2^j}
	\end{cases}.
	\]
	Clearly $\xi_i$ is centered, so $X_n = \xi_1 + \cdots + \xi_n$ is a martingale. Note that
	\[
	\sum_{i=1}^\infty \Prob[\xi_i = 2^j] = \sum_{i=1}^\infty \frac{1}{2^j} = 1<\infty.
	\]
	By Borel-Cantelli, we have that $\xi_i = -\frac{1}{1-2^{-j}}$ eventually with probability 1, so $X_n\to -\infty$ a.s.
\end{solution}

\noindent\textbf{Problem 4. }
Let $(X_n)$ be a martingale that is bounded a.s. either above or below by some constant $M$. Show that $\sup_n \E|X_n|<\infty$.
\begin{proof}
	If $X_n$ is bounded below, then $X_n+c$ is a nonnegative martingale for some $c>0$. By the martingale convergence theorem, $X_n+c$ converges almost surely to some limit $Y$ with $\E|Y|<\infty$. Consequently, $X_n$ also converges a.s. to an integrable function, so $\sup_n \E|X_n|<\infty$. If $X_n$ is bounded above, then $-X_n+c$ is a nonnegative martingale for some $c$ and the same argument works.
\end{proof}

\end{document}