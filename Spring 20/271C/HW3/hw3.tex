\documentclass[11pt,letterpaper]{report}
\usepackage{amssymb,amsfonts,color,graphicx,amsmath,enumerate,hyperref}
\usepackage{amsthm}

\newcommand{\naturals}{\mathbb{N}}
\newcommand{\integers}{\mathbb{Z}}
\newcommand{\complex}{\mathbb{C}}
\newcommand{\reals}{\mathbb{R}}
\newcommand{\mcal}[1]{\mathcal{#1}}
\newcommand{\rationals}{\mathbb{Q}}
\newcommand{\field}{\mathbb{F}}
\newcommand{\Var}{\text{Var}}
\newcommand{\ind}{\mathbbm{1}}
\newcommand{\Cov}{\text{Cov}}

\theoremstyle{definition}
\newtheorem{claim}{Claim}

\newenvironment{solution}
{\begin{proof}[Solution]}
{\end{proof}}

\voffset=-3cm
\hoffset=-2.25cm
\textheight=24cm
\textwidth=17.25cm
\addtolength{\jot}{8pt}
\linespread{1.3}

\begin{document}
\noindent{\em Liam Hardiman\hfill{May 17, 2020} }
\begin{center}
{\bf \Large 271C - Homework 3}
\vspace{0.2cm}
\hrule
\end{center}

\section*{\O ksendal 7.6}
Let $F(x, t, r, \omega) = X_r^{t, x}(\omega)$, where
\[
X_r^{t, x} = x + \int_t^rb(X_u^{t, x})\ du + \int_t^r\sigma(X_u^{t, r})\ dB_u,
\]
and $b: \reals^n\to \reals^n$ and $\sigma: \reals^n\to \reals^{n\times m}$ are Lipschitz continuous. Let $g(x, \omega) = f\circ F(x, t, t+h, \omega)$ where $f:\reals\to \reals$ is continuous.
\begin{enumerate}[(a)]
	\item Prove that the map $x\mapsto g(x, \cdot)$ is continuous from $\reals^n$ into $L^2(P)$.
	\begin{proof}
		Take $x,y\in \reals^n$ and let $v(t) = E[|X_t^x - X_t^y|^2]$ for all $t\leq T$. In the proof of theorem 5.2.1 in \O ksendal, it is shown that
		\[
		v(t) \leq F + A\int_0^tv(s)\ ds,
		\]
		where $F= 3E[|x-y|^2]$ and $A = 3(1+T)D^2$. Here, $D$ is the Lipschitz constant of $b$ and $\sigma$. Since $x$ and $y$ are deterministic, we have $E[|x-y|^2] = |x-y|^2$. By Gronwall we have
		\[
		E[|X_t^x - X_t^y|^2] = v(t) \leq Fe^{At} = 3|x-y|^2e^{At}.
		\]
		In particular, we have that $x\mapsto X_t^x$ is continuous. Consequently, the function $F(x, t, r, \omega)$ is continuous in its first argument. The map $x\mapsto g(x, \cdot) = f\circ F(x, t, t+h, \omega)$ is then a composition of continuous function, and thus continuous.
	\end{proof}

	\item For simplicity assume that $n=1$. Prove that $(x, \omega)\mapsto g(x, \omega)$ is measurable.
	\begin{proof}
		For each $m = 1, 2, \ldots$, set $\xi_k = \xi_k^{(m)} = k\cdot 2^{-m}$, $k = 1, 2, \ldots$. Let
		\[
		g^{(m)}(x, \cdot) = \sum_kg(\xi_k, \cdot)\cdot \chi_{[\xi_k, \xi_{k+1})}(x).
		\]
		We break our strategy up into a couple of claims.
		\begin{claim}\label{each_x}
			$g^{(m)}(x, \cdot)$ converges to $g(x, \cdot)$ in $L^2(P)$ for each $x$.
		\end{claim}
		From this we deduce the following.
		\begin{claim}\label{product}
			$g^{(m)}\to g$ in the product space $L^2(dm_R\times dP)$ for all $R$, where $dm_R$ is the Lebesgue measure on $\{|x|\leq R\}$.
		\end{claim}
		From this claim, we deduce that there is a subsequence of $g^{(m)}(x, \omega)$ that converges to $g(x, \omega)$ for almost all $(x, \omega)$.
	\end{proof}
\end{enumerate}










\section*{\O ksendal 7.9}
Let $X_t$ be a geometric Brownian motion, i.e.
\[
dX_t = rX_t\ dt + \alpha X_t\ dB_t,\quad X_0 = x>0,
\]
where $B_t\in \reals$ and $r,\alpha$ are constants.
\begin{enumerate}[(a)]
	\item Find the generator $A$ of $X_t$ and compute $Af(x)$ when $f(x) = x^\gamma$, $x>0$, $\gamma$ a constant.
	\begin{solution}
		By \O ksendal theorem 7.3.3, the generator of an It\^o diffusion
		\[
		dX_t = b(X_t)\ dt + \sigma(X_t)\ dB_t
		\]
		for $f\in C_0^2(\reals^n)$ is given by
		\[
		Af(x) = \left[ \sum_i b_i(x)\partial_{x_i}+ \frac{1}{2}\sum_{i,j}(\sigma\sigma^T)_{i,j}(x)\partial_{x_i}\partial_{x_j}\right]f(x).
		\]
		In our case, we have only one spatial variable, so we have
		\[
		Af(x) = \left[rx\partial_x + \frac{1}{2}\alpha^2x^2\partial^2_{x}\right]f(x).
		\]
		Setting $f(x) = x^\gamma$ gives
		\begin{align*}
			Af(x) &= rx\cdot \gamma x^{\gamma-1} + \frac{1}{2}\alpha^2x^2\cdot \gamma(\gamma-1)x^{\gamma-2}\\
			&= \left(r\gamma + \frac{1}{2}\alpha^2\gamma(\gamma-1)\right)x^\gamma.
		\end{align*}
		In particular, we have that $x^\gamma$ is an eigenfunction of the generator.
	\end{solution}

	\item Suppose $r<\frac{1}{2}\alpha^2$. What is the probability $p$ that $X_t$, when starting from $x<R$, ever hits the value $R$? Use Dynkin's formula with $f(x) = x^{\gamma_1}$, $\gamma_1 = 1-\frac{2r}{\alpha^2}$ to prove that
	\[
	p = \left(\frac{x}{R}\right)^{\gamma_1}.
	\]
	\begin{proof}
		We more or less follow example 7.4.2, which is the same question but for a regular Brownian motion instead of a geometric one. Fix any $r$ with $0<r<X<R$ and let $\tau$ be the stopping time $\tau = \inf\{t: X_t\notin (r, R)\}$. Set $\nu_k = \min(k , \tau)$. By Dynkin and part (a) we have by our choice of $\gamma_1$
		\begin{align*}
			E^x[f(X_{\nu_k})] &= f(x) + E^x\left[\int_0^{\nu_k}Af(X_s)\ ds\right]\\
			&= f(x).
		\end{align*}
		Since $r<\frac{1}{2}\alpha^2$, $X_t\to 0$ a.s., so $\nu_k< \infty$ a.s. Taking the limit on both sides of the above equation and applying the bounded convergence theorem gives
		\begin{align*}
			f(x) &= E[f(X_\tau)]\\
			&= f(r)[1-p_r] + f(R)p_r,
		\end{align*}
		where $p_r$ is the probability that $X_t$ exits $(r, R)$ by hitting $R$ first. Solving for $p_r$ gives
		\begin{align*}
			p_r &= \frac{f(x)-f(r)}{f(R)-f(r)}\\
			&= \frac{x^{\gamma_1}-r^{\gamma_1}}{R^{\gamma_1} - r^{\gamma_1}}.
		\end{align*}
		Taking the limit $r\to 0$ establishes the desired result.
	\end{proof}


	\item If $r>\frac{1}{2}\alpha^2$ then $X_t\to \infty$ a.s. $Q^x$. Put
	\[
	\tau = \inf\{t>0: X_t\geq R\}.
	\]
	Use Dynkin's formula with $f(x)  = \log x$, $x>0$ to prove that
	\[
	E^x[\tau] = \frac{\log(R/x)}{r - \alpha^2/2}.
	\]
\end{enumerate}


\section*{\O ksendal 7.14}
Let $B_t$ be an $n$-dimensional Brownian motion, $D\subseteq \reals^n$ a bounded open set and $h>0$ a harmonic function on $D$. Let $X_t$ be the solution of the SDE
\[
dX_t = \nabla (\log h)(X_t)\ dt+dB_t.
\]
More precisely, choose an increasing sequence $\{D_k\}$ of open subsets of $D$ such that $\overline{D}_k\subset D$ and $\cup_{k=1}^\infty D_k = D$. Then for each $k$ the equation above can be solved (strongly) for $t<\tau_{D_k}$. This gives in a natural way a solution for $t<\tau := \lim_{k\to \infty}\tau_{D_k}$.
\begin{enumerate}[(a)]
	\item Show that the generator $A$ of $X_t$ satisfies
	\[
	Af= \frac{\Delta (hf)}{2h},
	\]
	for $f\in C_0^2(D)$. In particular, if $f= \frac{1}{h}$, then $Af = 0$.
	\begin{proof}
		Let's dig into both sides of the desired inequality and meet in the middle. We have that $X_t$ is an It\^o diffusion with drift $b(x) = \nabla(\log h)(x)$ and diffusion $\sigma(x) = I_n$. Now the generator is given by
		\begin{align*}
			Af(x) &= (b\cdot \nabla f)(x) + \frac{1}{2}\sum_{i,j}(\sigma\sigma^T)_{i,j}(x)\partial_{x_i}\partial_{x_j}f\\
			&= \frac{1}{h}\sum_i(\partial_{x_i}h)(\partial_{x_i}f) + \frac{1}{2}\Delta f.
		\end{align*}
		On the other hand, we have
		\begin{align*}
			\Delta(hf) &= \sum_i\partial^2_{x_i}(hf)\\
			&= \sum_i(\partial^2_{x_i}h)f + 2(\partial_{x_i}h)(\partial_{x_i}f) + h(\partial^2_{x_i}f)\\
			&= 2 \nabla h\cdot \nabla f + h\Delta f.
		\end{align*}
		Dividing this last equation through by $2h$ establishes the claim.
	\end{proof}

	\item Use (a) to show that if there exists $x_0\in \partial D$ such that
	\[
	\lim_{x\to y\in \partial D}h(x) = \begin{cases}
		0&\text{if }y\neq x_0\\
		\infty&\text{if }y = x_0
	\end{cases}
	\]
	then
	\[
	\lim_{t\to \tau}X_t = x_0\text{ a.s.}
	\]
\end{enumerate}










\section*{\O ksendal 7.20}
Consider the equation
\[
dX_t = rX_t(K-X_t)\ dt + \alpha X_t(K-X_t)\ db_t;\quad X_0 = x\geq 0.
\]
This equation does not satisfy the conditions for existence and uniqueness in Theorem 5.2.1. However, we can still prove that a unique strong solution exists by proceeding as follows.
\begin{enumerate}[(a)]
	\item For $n = 1, 2, \ldots$ define
	\[
	b_n(y) = \begin{cases}
		y(K-y)&\text{if }0\leq y\leq n\\
		n(K-n)&\text{if }y>n
	\end{cases}
	\]
	and
	\[
	\sigma_n(y) = \begin{cases}
		\alpha y(K-y)&\text{if }0\leq y\leq n\\
		\alpha n(K-n)&\text{if }y>n
	\end{cases}
	\]
	and let $X_t = X^{(n)}_t$ be the unique solution of
	\[
	dX_t = b_n(X_t)\ dt + \sigma_n(X_t)\ dB_t;\quad X_0 = x.
	\]
	Define
	\[
	\tau_n = \inf\{t>0: X_t^{(n)} = n\}.
	\]
	Show that
	\[
	X^{(n)}_t = X^{(n+1)}_t\quad\text{for all }t<\tau_n
	\]
	and use this to find a unique strong solution $X_t$ for $t<\tau_\infty:= \lim_{n\to \infty} \tau_n$.
	\begin{proof}
		Even though the coefficient functions in $dX_t$ do not satisfy our theorem for existence and uniqueness of strong solutions, the truncated coefficients $b_n$ and $\sigma_n$ do. Consequently, there is a unique strong solution for $X_t^{(n)}$ for all $t$. Since $X^{(n)}$ and $X^{(n+1}$ satisfy the same SDE for $X_t^{(n)}, X_t^{(n+1)}\leq n$, which holds for all $t<\tau_n$, strong uniqueness implies that $X_t^{(n)} = X_t^{(n+1)}$ for all $t<\tau_n$. Loosely speaking, $X^{(n)}$ and $X^{(n+1)}$ agree with each other for $t<\tau_n$, so they can be ``glued'' together here.\\

		\noindent Now since $X_t$ must hit $n$ before it hits $n+1$ by continuity, the compatibility discussed above implies that
		\[
		\tau_n = \inf\{t>0: X_t^{(n)}=n\} \leq \inf\{t>0: X_t^{(n+1)} = n+1\} = \tau_{n+1}.
		\]
		Consequently, $\tau_n$ is an increasing sequence of stopping times, so it has an a.s. limit by the monotone convergence theorem. The compatibility of $X^{(n)}$ and $X^{(n+1)}$ then allows us to glue these truncated solutions together in the limit, so there is a unique strong solution $X_t$ for $t<\tau_\infty$.
	\end{proof}


	\item Prove that $\tau_\infty =\infty$ a.s.
	\begin{proof}
		We assume that the claims put forth in part (c) hold. Part (i) of (c) says that if $X$ starts at zero, it stays there. Consequently, $\tau_n = \infty$ a.s. for all $n\geq 1$ since $X$ will never reach $n$ a.s. in this case. We then have $\tau_\infty = \infty$ a.s. here. The same reasoning applies if $X$ starts at $K$: it gets stuck there so $\tau_\infty = \infty$ a.s.\\

		\noindent If $X_0$ starts in $(0, K)$, then part (iii) of (c) says that $X$ will stay there for all time. We then have $\tau_n = \infty$ a.s. for all $n>K$, so $\tau_\infty = \infty$ a.s.\\

		\noindent The only unintuitive case is when $X_0>K$ (I'm stealing this argument from the 1997 \href{https://reader.elsevier.com/reader/sd/pii/S0025556497000291?token=351C47C91AE312A241ADEE389E279E76737C9DA9ECED317F8963BDBCD3F22534CBC4612DEB95F032348FF7F1641792FA}{paper} by Lungu and \O ksendal). We take the expectation of both sides of the integral form of our SDE and use the strong Markov property on the stopping time $t\land \tau_n$:
		\[
		E^x[X_{t\land \tau_n}] = x + rE^x\left[\int_0^{t\land \tau_n}X_s(K-X_s)\ ds\right]\leq x.
		\]
		We've established that $\lim_{n\to \infty}\tau_n$ exists, so we can compute the pointwise limit
		\[
		\lim_{n\to \infty}X_{t\land \tau_n} = \begin{cases}
			X_t&\text{if }t<\tau_\infty\\
			\infty&\text{if }t\geq \tau_\infty.
		\end{cases}
		\]
		By Fatou's lemma we then have
		\[
		E^x\left[\lim_{n\to \infty}X_{t\land\tau_n}\right] \leq \liminf_{n\to \infty}E^x[X_{t\land \tau_n}].
		\]
		Since the right-hand side was just shown to be bounded by $x$, we must have that $\Pr[t\geq \tau_\infty] = 0$. Since this holds for all $t$, we must have $\tau_\infty = \infty$.
	\end{proof}

	\item Prove that
	\begin{enumerate}[(i)]
		\item $X_0 = 0 \implies X_t = 0$ for all $t$.
		\begin{proof}
			If $X_0 = 0$, then $dX_0 = 0$. By strong uniqueness, since $X_t = 0$ for all $t$ solves the SDE with this initial condition, it must be \textit{the} solution.
		\end{proof}

		\item $X_0 = K\implies X_t= K$ for all $t$.
		\begin{proof}
			The same reasoning used in (i) applies here. Since $dX_0 = 0$ and solves the SDE with this condition, it must be the solution by strong uniqueness.
		\end{proof}

		\item $0<X_0<K\implies 0<X_t<K$ for all $t$.
		\begin{proof}
			Intuitively, if $X_t$ ever hits 0 or $K$, since the governing SDE is autonomous, the shift $t-\tau_K$, say, puts us in the setting of part (ii). Strong uniqueness then lets us extend the $X_{t-\tau_K} = K$ solution back in time to $X_t = K$ for all $t$. But if this were the case, then $X$ couldn't have started in $(0, K)$. We conclude that if $X$ starts in $(0, K)$, it must stay there.
		\end{proof}

		\item $X_0>K\implies X_t>K$ for all $t$.
		\begin{proof}
			We use the same reasoning as in part (iii). If $X$ starts above $K$ and somehow hits the ``floor'' that is $K$, autonomy of the SDE lets us shift the time origin to the instant we hit the floor. Part (ii) says that if $X_{t-\tau_K}$ starts at $K$, it stays there. Strong uniqueness lets us extend this solution back in time to say that $X_t$ started at $K$, which contradicts the assumption that $X$ started above $K$.
		\end{proof}
	\end{enumerate}
\end{enumerate}










\section*{\O ksendal 8.2}
Show that the solution $u(t, x)$ of the IVP
\begin{align*}
\partial_t u &= \frac{1}{2}\beta^2x^2\partial^2_xu + \alpha x\partial_x u;\quad t>0, x\in \reals\\
u(0, x) &= f(x)\quad (f\in C_0^2(\reals)\text{ given})
\end{align*}
can be expressed as follows:
\begin{align*}
	u(t,x) &= E[f(x\cdot \exp\{\beta B_t + (\alpha - \frac{1}{2}\beta^2)t\})]\\
	&= \frac{1}{\sqrt{2\pi t}}\int_\reals f(x\cdot \exp\{\beta y + (\alpha - \frac{1}{2}\beta^2)t\})\exp\left(-\frac{y^2}{2t}\right)\ dy;\quad t>0.
\end{align*}
\begin{proof}
	We apply Kolmogorov's backward equation, which states that if $u(t, x)$ satisfies
	\begin{align*}
		\partial_t u &= Au,\quad t>0, x\in \reals^n\\
		u(0, x) &= f(x),\quad x\in \reals^n,
	\end{align*}
	where $A$ is the generator of some diffusion process, $X_t$, then
	\[
	u(t, x) = E^x[f(X_t)].
	\]
	Basically, we have to identify that RHS of the given PDE as the generator of a diffusion process and the integral as the expectation with respect to an appropriate probability measure.\\

	\noindent Some combing through \O ksendal shows that the RHS of the PDE is indeed the generator of a geometric Brownian motion:
	\[
	dX_t = \alpha X_t\ dt + \beta X_t\ dB_t,\quad X_0 = x>0.
	\]
	We've seen that the solution to this SDE is
	\[
	X_t = xe^{(\alpha -\frac{\beta^2}{2})t + \beta B_t}.
	\]
	That definitely explains the first line in the desired form of $u(t, x)$, but I'm not sure why the relevant probability measure implicit in the expectation is Gaussian. Do you get this by looking at Kolmogorov's forward equation?
\end{proof}










\section*{Karatzas and Shreve 1.2.23}
Show that if $\{T_n\}_{n=1}^\infty$ is a sequence of optional times and $T = \inf_{n\geq 1}T_n$, then $\mcal{F}_{T^+} = \cap_{n=1}^\infty \mcal{F}_{T_n^+}$. Besides, if each $T_n$ is a positive stopping time and $T<T_n$ on $\{T<\infty\}$, then we have $\mcal{F}_{T^+} = \cap_{n=1}^\infty \mcal{F}_{T_n}$.
\begin{proof}
	By Lemma 2.11 in the text, $T$ is an optional time. Consequently, we have that $A\in \mcal{F}_{T^+}$ if and only if $A\cap \{T\leq t\}\in \mcal{F}_{t^+}$. By the definition of the infimum we have that $A\cap \{T_n\leq t\}\in \mcal{F}_{t^+}$ as well for all $n$, so $A\in \mcal{F}_{T_n}$ too. We then have that $\mcal{F}_{T^+}\subseteq \cap_{n=1}^\infty \mcal{F}_{T_n^+}$. Conversely, suppose $A\cap \{T_n<t\}\in \mcal{F}_t$ for all $n$ and $t$. Again by the definition of the infimum, we have
	\[
	A\cap \{T<t\} = A\cap \left(\bigcup_{n=1}^\infty\{T_n<t\}\right) = \bigcup_{n=1}^\infty(A\cap \{T_n<t\})\in \mcal{F}_t,
	\]
	so $\cap_{n=1}^\infty \mcal{F}_{T_n^+}\subseteq \mcal{F}_{T^+}$.\\

	\noindent As for the second claim, by Problem 2.22, since $T_n$ is positive and $T<T_n$ on $\{T<\infty\}$, then $\mcal{F}_{T^+} \subseteq \mcal{F}_{T_n}$ for each $n$, so $\mcal{F}_{T^+}\subseteq \cap_{n=1}^\infty \mcal{F}_{T_n}$. The other inclusion follows from the above discussion.
\end{proof}










\section*{Karatzas and Shreve 1.5.24(i)}
Let $M = \{M_t, \mcal{F}_t: 0\leq t<\infty\}$ be a process in $\mcal{M}_2\cup \mcal{M}^{c, loc}$ and assume that its quadratic variation process $\langle M\rangle$ is integrable: $E\langle M\rangle_\infty<\infty$. Show that $M$ is a martingale, and that $M$ and the submartingale $M^2$ are both uniformly integrable. In particular, show that $M_\infty = \lim_{t\to \infty}M_t$ exists a.s. $P$ and $EM_\infty^2 = E\langle M\rangle_\infty$.
\begin{proof}
	If $M\in \mcal{M}_2$, then it's clearly a martingale. Since it's square integrable, it is uniformly integrable by Cauch-Schwartz. We also have
	\[
	E[M_t^2] = E\langle M\rangle_t \leq E\langle M\rangle_\infty,
	\]
	so the submartingale $M^2$ is also uniformly integrable. The martingale convergence theorem for uniformly integrable martingales implies that $M$ converges a.s.
\end{proof}

\end{document}