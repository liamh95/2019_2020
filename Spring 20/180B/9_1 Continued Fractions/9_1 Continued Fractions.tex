\documentclass[11pt,letterpaper]{report}
\usepackage{amssymb,amsfonts,color,graphicx,amsmath,enumerate}
\usepackage{amsthm}

\newcommand{\naturals}{\mathbb{N}}
\newcommand{\integers}{\mathbb{Z}}
\newcommand{\complex}{\mathbb{C}}
\newcommand{\reals}{\mathbb{R}}
\newcommand{\mcal}[1]{\mathcal{#1}}
\newcommand{\rationals}{\mathbb{Q}}
\newcommand{\field}{\mathbb{F}}
\newcommand{\Var}{\text{Var}}
\newcommand{\ind}{\mathbbm{1}}
\newcommand{\Cov}{\text{Cov}}

\newenvironment{solution}
{\begin{proof}[Solution]}
{\end{proof}}

\voffset=-3cm
\hoffset=-2.25cm
\textheight=24cm
\textwidth=17.25cm
\addtolength{\jot}{8pt}
\linespread{1.3}

\begin{document}
\begin{center}
{\bf \Large Math 180B - Continued Fractions}
\vspace{0.2cm}
\hrule
\end{center}

\begin{enumerate}
	\item Find the continued fraction expansion for the following numbers.
	\begin{enumerate}
		\item $61/14$
		\vfill
		\item $\frac{1+\sqrt{5}}{2}$.
		\vfill
		\item $\sqrt{13}$.
		\vfill
	\end{enumerate}

	\item Let $c_1, \ldots, c_n$ be integers such that the continued fraction $[c_1; \ldots, c_n]$ exists. Show that we can describe the continued fraction in terms of matrix multiplication
	\[
	\begin{pmatrix}
		c_1 & 1\\
		1 & 0
	\end{pmatrix}
	\begin{pmatrix}
		c_2 & 1\\
		1 & 0
	\end{pmatrix}\cdots
	\begin{pmatrix}
		c_n & 1\\
		1 & 0
	\end{pmatrix}=
	\begin{pmatrix}
		A & B\\
		C & D
	\end{pmatrix}\implies [c_1;\ldots, c_n] = \frac{A}{C}.
	\]
	\vfill
\end{enumerate}

\end{document}