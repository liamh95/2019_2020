\documentclass[12pt]{article}  
%%Read the manual for other options. 

\pagestyle{empty} %%Eliminates page numbers
%%\input rmb_macros
%%Collect your favorite macros in a 
%%separate file

%\input amssym.def
%\input amssym
%\input mssymb
%%Defines additional symbols



\usepackage{graphics}
\usepackage{amsmath,amssymb,amsthm, multicol}
\usepackage[pdftex]{graphicx}
\usepackage{hyperref}
%%Use to include pictures. 

%\newcommand{\comment}[1]{}
%\newcommand{\sobolev}[2]{W^{#1,#2}}
%\newcommand{\sobolev}[2]{L^#2_#1}
%%Some examples of macros or new commands.
\newtheorem{theorem}{Theorem}
\newtheorem*{theorem*}{Theorem}
\newtheorem{corollary}{Corollary}
\newtheorem{lemma}[theorem]{Lemma}

\addtolength{\oddsidemargin}{-.75in}
\addtolength{\evensidemargin}{-.75in}
\addtolength{\textwidth}{1.5in}
\addtolength{\topmargin}{-1in}
\addtolength{\textheight}{2.25in}
%%Set margins, defaults are ok. 

\begin{document}
\begin{flushleft} 
%%Paragraphs will not be indented in this 
%%environment
\centerline{\LARGE{Quiz 2}} 
\vspace{5 mm}
{Student ID Number:}\hfill  
%%\hfill forces following text 
%%to right margin
{Name \rule {2 in}{0.01in}}\\
Math 175, 12PM
\\
%%gives a line of length 2in and 
%%thickness 0.01in
{Please justify all your answers}\hfill {January 23, 2020}
\\
{Please also write your full name on the back} 

\medskip
\end{flushleft}

\begin{enumerate}
	\item True or false?
	\begin{enumerate}
		\item If $A$ and $B$ are finite sets, then $|A\cup B| = |A| + |B|$.
		\item There are $\binom{52}{5}$ ways to draw a hand of five cards from a deck of 52 cards.
	\end{enumerate}
	\item Briefly explain what is wrong with this inductive proof.
	\begin{theorem*}
		$\frac{d}{dx}x^n = 0$ for all $n\geq 0$.
	\end{theorem*}
	\begin{proof}
		For the base case, suppose $n = 0$. Then
		\[
		\frac{d}{dx}x^0 = \frac{d}{dx}1 = 0.
		\]
		Suppose the claim holds for all $k\leq n$. By the product rule we have
		\[
		\frac{d}{dx}x^{n+1} = \frac{d}{dx}(x^n\cdot x^1) = x^n\cdot \frac{d}{dx}x^1 + x^1\cdot \frac{d}{dx}x^n = x^n\cdot 0 + x^1\cdot 0 = 0.
		\]
		The claim follows by induction.
	\end{proof}
	\vfill

	\item How many positive integers below 100 are divisible by 3, 5, or 7?
	\vfill
\end{enumerate}

%\vfill will divide page evenly
%use \begin{enumerate} environment for ordered lists
\end{document}