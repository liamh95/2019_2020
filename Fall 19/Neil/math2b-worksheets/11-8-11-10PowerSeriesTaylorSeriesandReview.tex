\documentclass[12pt,fleqn]{article}  
\usepackage{amsmath}
\usepackage{amssymb}

\pagestyle{empty}

\textheight 22cm
\textwidth 17cm
\oddsidemargin -0.5cm
\evensidemargin -0.5cm
\topmargin -1.5cm
\topskip 0cm
\headheight 0.5cm
\headsep 1cm


\begin{document}

\begin{center}
	\textbf{Math 2B Worksheet: 11.8/10 Power Series and Taylor Series}
\end{center}

\emph{Write your names and Student ID numbers at the top of the page}


\begin{enumerate}
\item Find the radius and interval of convergence.
\[\sum_{n=1}^\infty n!(2x-1)^n\]

%\[\sum_{n=1}^\infty\frac{x^n}{n^4 4^n}\]
\vfill

\item Find a power series representation for the function and determine its interval of convergence.
\[f(x)=\frac{x}{2x^2+1}\]

%\[f(x)=\frac{x^2}{x^4+16}\]
\vfill

%\item Use differentiation to find a power series representation for $\displaystyle f(x)=\frac{1}{(1+x)^2}$.\vfill

\newpage
\item Evaluate $\displaystyle\int\frac{t}{1-t^8}dt$ as a power series then find the radius of convergence.\vfill



\item Use the definition of Taylor series to find the first 4 terms of the Taylor series for $f(x)=\ln x$ centered at $a=1$.  Do the same for the Maclaurin series of $g(x)=xe^x$ (the first 4 nonzero terms).\vfill

%\item Use a known Maclaurin series to obtain the Maclaurin series for the following. (Hint for the second one:  find the Maclaurin series for the derivative first.  Then integrate.)\\
%$f(x)=x\cos\big(\frac{x^2}{2}\big)$

%$f(x)=\arctan(x^2)\;$



\end{enumerate}

\end{document}

