\documentclass[11pt,letterpaper]{report}
\usepackage{amssymb,amsfonts,color,graphicx,amsmath,enumerate}
\usepackage{tikz}
\usepackage{amsthm}

\newcommand{\naturals}{\mathbb{N}}
\newcommand{\integers}{\mathbb{Z}}
\newcommand{\complex}{\mathbb{C}}
\newcommand{\reals}{\mathbb{R}}
\newcommand{\exreals}{\overline{\mathbb{R}}}
\newcommand{\mcal}[1]{\mathcal{#1}}
\newcommand{\mable}{measurable}
\newcommand{\quats}{\mathbb{H}}
\newcommand{\rationals}{\mathbb{Q}}
\newcommand{\norm}{\trianglelefteq}
\newcommand{\Aut}{\text{Aut}}
\newcommand{\disk}{\mathbb{D}}
\newcommand{\halfplane}{\mathbb{H}}
\newcommand{\Lp}[2]{\left\|{#1}\right\|_{L^{#2}}}
\newcommand{\supp}[1]{\text{supp}({#1})}
\newcommand{\Hom}[2]{\text{Hom}_{{#1}}({#2})}
\newcommand{\tr}{\text{tr}}
\newcommand{\field}[1]{\mathbb{F}_{{#1}}}
\newcommand{\Gal}[1]{\text{Gal}\left({#1}\right)}
\newcommand{\esssup}{\text{ess sup }}
\newcommand{\essinf}{\text{ess inf }}
\newcommand{\affine}{\mathbb{A}}
\newcommand{\E}{\mathbb{E}}

\newenvironment{solution}
{\begin{proof}[Solution]}
{\end{proof}}

\voffset=-3cm
\hoffset=-2.25cm
\textheight=24cm
\textwidth=17.25cm
\addtolength{\jot}{8pt}
\linespread{1.3}

\begin{document}
\noindent{\em Liam Hardiman\hfill{October 28, 2019} }
\begin{center}
{\bf \Large 270A - Homework 1}
\vspace{0.2cm}
\hrule
\end{center}

\begin{enumerate}
	\item \begin{enumerate}
		\item Let $\mcal{F}$ be the family of all finite subsets of $\Omega$ and their complements. Is $\mcal{F}$ a $\sigma$-algebra?
		\begin{solution}
			If $\Omega$ is finite then $\mcal{F}$ is simply the power set of $\Omega$, which is definitely a $\sigma$-algebra. However, if $\mcal{F}$ is infinite, then $\mcal{F}$ is never a $\sigma$-algebra. To see this, let $(x_n)$ be a countable sequence of distinct elements in $\Omega$ and consider the set of even-indexed terms
			\[
				F = \{x_n: n = 2k,\ k\in \naturals\}.
			\]
			This set is a countable union of singletons and all singletons belong to $\mcal{F}$. $F$ is clearly infinite, but so is its complement, which contains the (infinite) set of odd-indexed terms. We conclude that $F$ is neither finite nor co-finite, so $\mcal{F}$ is not closed under countable unions when $\Omega$ is an infinite set.
		\end{solution}

		\item Let $\mcal{F}$ be the family of all countable subsets of $\Omega$ and their complements. Is $\mcal{F}$ a $\sigma$-algebra?
		\begin{solution}
			$\mcal{F}$ is indeed a $\sigma$-algebra. The empty set is clearly countable, and $\Omega^C = \emptyset$. Let $F_n$ be a countable collection of sets in $\mcal{F}$ and consider their union, $F = \cup_{n=1}^\infty F_n$. If each $F_n$ is countable, then $F$ is just a countable union of countable sets: countable. If one of the $F_n$'s, say $F_k$, were co-countable, then $F^C \subseteq F_k^C$, which is countable, so $F$ is co-countable. Since $\mcal{F}$ contains the empty set and $\Omega$ and is closed under countable unions and complements, it is a $\sigma$-algebra.
		\end{solution}

		\item Let $\mcal{F}$ and $\mcal{G}$ be two $\sigma$-algebras of subsets of $\Omega$. Is $\mcal{F}\cap \mcal{G}$ always a $\sigma$-algebra?
		\begin{solution}
			$\mcal{F}$ is a $\sigma$-algebra. Since $\mcal{F}$ and $\mcal{G}$ both contain $\emptyset$ and $\Omega$, so does their intersection. Let $E_n$ be a countable collection of sets in $\mcal{F}\cap \mcal{G}$. Since $\mcal{F}$ and $\mcal{G}$ are both $\sigma$-algebras, the union $E = \cup_{n=1}^\infty E_n$ is in both $\mcal{F}$ and $\mcal{G}$ and each $E_n^C$ is in both $\mcal{F}$ and $\mcal{G}$ as well.
		\end{solution}

		\item Let $\mcal{F}$ and $\mcal{G}$ be two $\sigma$-algebras of subsets of $\Omega$. Is $\mcal{F}\cup \mcal{G}$ always a $\sigma$-algebra?
		\begin{solution}
			The union need not be a $\sigma$-algebra. Let $\Omega = \{1, 2, 3, 4\}$, $\mcal{F} = \{\emptyset, \Omega, \{1\}, \{2, 3, 4\}\}$, and $\mcal{G} = \{\emptyset, \Omega, \{2\}, \{1, 3, 4\}\}$. $\mcal{F}$ and $\mcal{G}$ are $\sigma$-algebras, but the set $\{1\}\cup \{2\} = \{1,2\}$ is not in their union.
		\end{solution}
	\end{enumerate}

	\item A subset $A\subset \naturals$ is said to have asymptotic density if
	\[
	\lim_{n\to \infty}\frac{|A\cap \{1, \ldots, n\}|}{n}
	\]
	exists. Let $\mcal{F}$ be the collection of subsets of $\naturals$ for which the asymptotic density exists. Is $\mcal{F}$ a $\sigma$-algebra?
	\begin{solution}
		
	\end{solution}
\end{enumerate}

\end{document}