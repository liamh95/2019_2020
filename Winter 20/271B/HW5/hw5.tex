\documentclass[11pt,letterpaper]{report}
\usepackage{amssymb,amsfonts,color,graphicx,amsmath,enumerate}
\usepackage{tikz}
\usepackage{amsthm}
\usepackage{bbm}
\usepackage{hyperref}

\newcommand{\naturals}{\mathbb{N}}
\newcommand{\integers}{\mathbb{Z}}
\newcommand{\complex}{\mathbb{C}}
\newcommand{\reals}{\mathbb{R}}
\newcommand{\mcal}[1]{\mathcal{#1}}
\newcommand{\rationals}{\mathbb{Q}}
\newcommand{\Lp}[2]{\left\|{#1}\right\|_{L^{#2}}}
\newcommand{\field}{\mathbb{F}}
\newcommand{\affine}{\mathbb{A}}
\newcommand{\E}{\mathbb{E}}
\newcommand{\Prob}{\mathbb{P}}
\newcommand{\Var}{\text{Var}}
\newcommand{\ind}{\mathbbm{1}}
\newcommand{\Cov}{\text{Cov}}

\newenvironment{solution}
{\begin{proof}[Solution]}
{\end{proof}}

\voffset=-3cm
\hoffset=-2.25cm
\textheight=24cm
\textwidth=17.25cm
\addtolength{\jot}{8pt}
\linespread{1.3}

\begin{document}
\noindent{\em Liam Hardiman\hfill{March 11, 2020} }
\begin{center}
{\bf \Large 271B - Homework 5}
\vspace{0.2cm}
\hrule
\end{center}

\noindent\textbf{Problem 1. }
Consider
\begin{equation}\label{sde}
dX_t = \mu(X_t)dt + \sigma(X_t)dB_t,\quad X_0 = 1,
\end{equation}
with $\mu(x) = x+a$, $\sigma(x) = 4x$. Assuming $X_t > 0$, find $dY_t$ when $Y_t = \sqrt{X_t}$. Can you find $Y_t$?
\begin{solution}
	Let $g(t, x) = \sqrt{x}$ so that $Y_t = g(t, X_t)$. By It\^o's lemma we have
	\begin{align*}
		dY_t &= \frac{1}{2}X_t^{-1/2}dX_t - \frac{1}{8}X_{t}^{-3/2}(dX_t)^2\\
		&= \frac{1}{2}X_t^{-1/2}[(X_t+a)dt] - \frac{1}{8}X_t^{-3/2}(16X_t^2dt)\\
		&= \frac{a-3Y_t^2}{2Y_t}dt + 2Y_tdB_t.
	\end{align*}
	Using $dY_t$ to find $Y_t$ proved difficult. Finding $X_t$ and taking the square root worked however. We use the stochastic version of integrating factors. Define the function
	\[
	F_t = \exp\left(\frac{1}{2}\int_0^t16\ ds -\int_0^t4\ dB_s \right) = \exp(8t - 4B_t).
	\]
	From this we get
	\[
	dF_t = 16F_tdt - 4F_tdB_t.
	\]
	We apply the multivariable It\^o lemma to obtain (after some tedious algebra)
	\begin{align*}
	d(X_tF_t) &= X_tdF_t + F_tdX_t + d\langle X_t, F_t\rangle\\
	&= F_t(a+X_t)dt.
	\end{align*}
	Setting $Z_t = X_tF_t$, we obtain the linear DE
	\[
	\frac{dZ_t}{dt}-Z_t = aF_t.
	\]
	We multiply through by $e^{-t}$ to obtain
	\[
	\frac{d}{dt}(Z_te^{-t}) = ae^{7t-4B_t}.
	\]
	Integrating through and substituting $X_t$ back in gives
	\[
	X_t = \frac{1 + \int_0^t\exp(7s-4B_s)ds}{\exp(7t-4B_t)}.
	\]
	Taking the square root gives $Y_t$.
	\[
	Y_t = \left(\frac{1 + \int_0^t\exp(7s-4B_s)ds}{\exp(7t-4B_t)} \right)^{1/2}.
	\]
\end{solution}

\noindent\textbf{Problem 2. }
Let $X_t$ be as in (\ref{sde}) but with $\mu(x) = 2x$ and $\sigma(x) = x^a$ and and $Y_t = X_t^b$. Find $b$ so that $\langle Y\rangle_t$ is linear in $t$.
\begin{solution}
	By It\^o's lemma we have
	\begin{align*}
		dY_t &= bX^{b-1}dX_t + \frac{1}{2}b(b-1)X^{b-2}(dX_t)^2\\
		&= f(X_t)dt + bX_t^{a+b-1}dB_t,
	\end{align*}
	for some function $f$. Consequently we have
	\[
	d\langle Y\rangle_t = (bX_t^{a+b-1})^2dt.
	\]
	From this we compute the quadratic variation:
	\[
	\langle Y\rangle_t = \int_0^td\langle Y\rangle_t = b^2\int_0^tX_s^{2a+2b-2}ds.
	\]
	Setting $b = 1-a$ makes the exponent in the integrand zero, which makes $\langle Y\rangle_t$ linear in $t$.
\end{solution}

\noindent\textbf{Problem 3. }
Let
\[
dX_t = \sqrt{1+X_t}\ dB_t,\quad X_0 = 0.
\]
Find $\E[X_t]$ and $\E[X_t^2]$.
\begin{solution}
	Since $\sqrt{1+x}$ is sublinear and Lipschitz on $[0, \infty)$, the SDE solution existence and uniqueness theorem says that $\sqrt{1+X_t}$ is in class I$^*$. The given SDE then describes an It\^o process which has only a martingale component. The expectation is then given by
	\[
	\E[X_t] = X_0 = 0.
	\]
	As for the second moment, we use the It\^o isometry and the fact that $X_0 = 0$
	\begin{align*}
		\E[X_t^2] &= \E\left[\left(X_0 + \int_0^t\sqrt{1+X_s}dB_s\right)^2 \right]\\
		&= \E\left[\int_0^t(1+X_s)ds \right]\\
		&= \int_0^t\E[1+X_s]ds\\
		&= t.
	\end{align*}
\end{solution}

\noindent\textbf{Problem 4. }
Let $Y$ be an $\mcal{F}_T$-measurable random variable such that $\E[|Y|^2]<\infty$ and consider Doob's martingale
\[
M_t = \E[Y|\mcal{F}_t],\quad 0\leq t\leq T
\]
with respect to the filtration $\{\mcal{F}_t\}_{0\leq t\leq T}$.
\begin{enumerate}[(a)]
	\item Show that $\E[M_t^2]<\infty$ for all $t\in [0, T]$.
	\begin{proof}
		This follows from the conditional form of Jensen's inequality.
		\begin{align*}
			\E[M_t^2]  = \E[\E[Y|\mcal{F}_t]^2] \leq \E[\E[Y^2|\mcal{F}_t]] = \E[Y^2]<\infty.
		\end{align*}
	\end{proof}

	\item By the martingale representation theorem, there exists a unique process $g(t, \omega)$ in class I$^*$ such that
	\[
	M_t = \E[M_0] + \int_0^tg(s, \omega)dB_s,\quad t\in [0,T].
	\]
	Find $g$ in the following cases.
	\begin{enumerate}[(i)]
		\item $Y(\omega) = B^2_T(\omega)$.
		\begin{solution}
			\begin{align*}
				M_t &= \E[(B_T-B_t+B_t)^2|\mcal{F}_t]\\
				&= \E[(B_T-B_t)^2|\mcal{F}_t] + 2\E[B_t(B_T-B_t)|\mcal{F}_t] + \E[B_t^2|\mcal{F}_t]\\
				&= (T-t) + B_t^2\\
				&= T + \int_0^tB_s\ dB_s.
			\end{align*}
			So $g(s, \omega) = B_s(\omega)$.
		\end{solution}

		\item $Y(\omega) = B^3_T(\omega)$.
		\begin{solution}
			\begin{align*}
				M_t &= \E[(B_T-B_t+B_t)^3|\mcal{F}_t]\\
				&= \E[(B_T-B_t)^3|\mcal{F}_t] + 3\E[B_t(B_T-B_t)^2|\mcal{F}_t] + 3\E[B_t^2(B_T-B_t)|\mcal{F}_t] + \E[B_t^3|\mcal{F}_t]\\
				&= 3B_t(T-t) + B_t^3.
			\end{align*}
			Let $g(t, X)= 3x(T-t)+x^3$ so that $M_t = g(t, B_t)$. By It\^o's lemma
			\begin{align*}
				dM_t &= -3B_tdt + [3(T-t)+3B_t^2]dB_t+3B_tdt = 3[(T-t)+B_t^2]dB_t.
			\end{align*}
			Consequently, our $g(s, \omega) = 3[(T-t)+B_t^2]$.
			% We've seen that
			% \begin{align*}
			% B_t^3 &= 3\int_0^tB_s\ ds + 3\int_0^tB_s^2\ dB_s\\
			% &= 3tB_t + 3\int_0^t(B_s^2-s)dB_s\\
			% &= 3\int_0^t(B_s^2 + (t-s))dB_s.
			% \end{align*}
			% Conditioning on $\mcal{F}_t$ gives $M_t$:
			% \[
			% M_t = \E[B_T(\omega)|\mcal{F}_t] = 3\int_0^t(B_s^2 + (t-s))dB_s.
			% \]
			% We then have $g(s, \omega) = B_s^2(\omega) + t-s$.
		\end{solution}

		\item $Y(\omega) = \exp(\sigma B_T)$, $\sigma \in \reals$ is a constant.
		\begin{solution}
			
		\end{solution}
	\end{enumerate}
\end{enumerate}

\end{document}