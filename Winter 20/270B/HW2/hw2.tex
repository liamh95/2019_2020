\documentclass[11pt,letterpaper]{report}
\usepackage{amssymb,amsfonts,color,graphicx,amsmath,enumerate,mathtools,bbm}
\usepackage{tikz}
\usepackage{amsthm}

\newcommand{\naturals}{\mathbb{N}}
\newcommand{\integers}{\mathbb{Z}}
\newcommand{\complex}{\mathbb{C}}
\newcommand{\reals}{\mathbb{R}}
\newcommand{\mcal}[1]{\mathcal{#1}}
\newcommand{\rationals}{\mathbb{Q}}
\newcommand{\Lp}[2]{\left\|{#1}\right\|_{L^{#2}}}
\newcommand{\field}{\mathbb{F}}
\newcommand{\affine}{\mathbb{A}}
\newcommand{\E}{\mathbb{E}}
\newcommand{\Prob}{\mathbb{P}}
\newcommand{\Var}{\text{Var}}
\newcommand{\ind}{\mathbbm{1}}
\newcommand{\Cov}{\text{Cov}}


\newenvironment{solution}
{\begin{proof}[Solution]}
{\end{proof}}

\voffset=-3cm
\hoffset=-2.25cm
\textheight=24cm
\textwidth=17.25cm
\addtolength{\jot}{8pt}
\linespread{1.3}

\begin{document}
\noindent{\em Liam Hardiman\hfill{February 10, 2020} }
\begin{center}
{\bf \Large 270B - Homework 2}
\vspace{0.2cm}
\hrule
\end{center}

\noindent\textbf{Problem 1. }
\begin{enumerate}[(a)]
	\item Prove that if the probability density functions of $X_n$ converge pointwise to the probability density function of $X$, then $X_n$ converges to $X$ weakly.

	\begin{proof}
		Let $f_n$ and $f$ be the density functions of $X_n$ and $X$, respectively, and let $\varphi:\reals\to \reals$ be bounded and continuous. We then have
		\begin{equation}\label{bdd}
		\begin{split}
			|\E_n[\varphi] - \E[\varphi]| &= \left|\int_\reals f_n(x)\varphi(x)\ dx - \int_\reals f(x)\varphi(x)\ dx\right|\\
			&\leq \|\varphi\|_\infty \int_\reals|f_n(x) - f(x)|\ dx.
		\end{split}
		\end{equation}
		Now we claim that $\Lp{f_n - f}{1}\to 0$ (argument taken from 210 notes). Since $\int f_n = \int f = 1$ we have by Fatou
		\begin{align*}
		2\int_\reals f\ dx = \int_\reals\liminf_{n\to \infty}(f + f_n - |f_n - f|)\ dx \leq 2\int_\reals f\ dx - \limsup_{n\to \infty}|f_n - f|\ dx.
		\end{align*}
		In particular, we have $\limsup_{n\to \infty}\int |f_n - f| = 0$, so $\Lp{f_n - f}{1}\to 0$. Consequently, the right-hand side of (\ref{bdd}) goes to zero as $n\to \infty$. We then have that $\E_n[\varphi]\to \E[\varphi]$ for all $\varphi$ bounded continuous, so $X_n\to X$ weakly.
	\end{proof}

	\item Prove that if the probability mass functions of $X_n$ converge to the probability mass function of $X$ pointwise then $X_n$ converges to $X$ weakly.
	\begin{proof}
		We essentially mirror our proof of part (a) but with the counting measure. Let $f_n$ and $f$ be the mass functions of $X_n$ and $X$ respectively and suppose they take values in the discrete set $S\subset \reals$. Suppose $\varphi$ is a bounded function on $S$ (note that any function from a discrete space is continuous). We then have
		\begin{equation}\label{bdd_disc}
			\begin{split}
				|\E_n[\varphi] - \E[\varphi]| &= \left|\sum_{x\in S}f_n(x)\varphi(x) - \sum_{x\in S}f(x)\varphi(x)\right|\\
				&\leq \|\varphi\|_\infty \sum_{x\in S}|f_n(x) -f(x)|.
			\end{split}
		\end{equation}
		Fatou still works with the counting measure, so we have
		\[
		2\sum_{x\in S}f(x) = \sum_{x\in S}\liminf_{n\to \infty}(f(x) + f_n(x) - |f_n(x)-f(x)|)\leq 2\sum_{x\in S}f(x) - \limsup_{n\to \infty}\sum_{x\in S}|f_n(x)-f(x)|.
		\]
		We conclude that $\sum_{x\in S}|f_n(x)-f(x)|\to 0$ as $n\to \infty$. The right-hand side of (\ref{bdd_disc}) vanishes as $n\to \infty$, so $X_n\to X$ weakly.
	\end{proof}

	\item In general, weak convergence does not imply pointwise convergence of probability density functions. Show this by example.
	\begin{solution}
		Let $f_{n,k}(x)$ be a typewriter sequence weighted to have integral 1, that is
		\[
		f_{n,k}(x) = \frac{1}{1-2^{-n}}\ind_{[0,1]\setminus [k\cdot 2^{-n},\ (k+1)\cdot 2^{-n}]}(x),\quad n = 1, 2, \ldots,\ k = 0, 1, \ldots, 2^{n}-1.
		\]
		Intuitively, $f_{n,k}$ is a flat line of height $\frac{1}{1-2^{-n}}$ over $[0,1]$ except for a gap at $[k\cdot 2^{-n}, (k+1)\cdot 2^{-n}]$, where it is zero. Since this gap slides along the unit interval indefinitely, $f_{n,k}$ does not converge pointwise anywhere. Now for any $\varphi$ bounded and continuous we have
		\[
		|\E_{n,k}[\varphi] - \E[\varphi]| = \int_0^1\varphi(x)(f_{n,k}(x) - 1)dx \leq \|\varphi\|_\infty\cdot 2^{-n}\to 0,
		\]
		so $X_n$ converges weakly to the uniform distribution on $[0,1]$ even though the densities don't converge pointwise at all.
\end{solution}
\end{enumerate}

\noindent\textbf{Problem 2. }
Consider normal random variables $X_n\sim \mcal{N}(\mu_n, \sigma^2_n)$. Assume $X_n$ converges weakly to some random variable $X$. Prove that $X\sim \mcal{N}(\mu, \sigma^2)$ where $\mu = \lim_{n\to \infty}\mu_n$ and $\sigma^2 = \lim_{n\to \infty}\sigma_n^2$ and both limits exits.
\begin{proof}
	Since $x\mapsto e^{itx}$ is bounded and continuous, the weak convergence of $X_n$ to $X$ implies that the characteristic functions $\phi_n$ of $X_n$ converge pointwise to the characteristic function $\phi$ of $X$. After completing the square in the exponent, $\phi_n$ is given by
	\begin{align*}
		\phi_n(t) &= \frac{1}{\sqrt{2\pi \sigma_n^2}}\int_\reals e^{-\frac{1}{2}\left(\frac{x-\mu_n}{\sigma_n}\right)^2}\cdot e^{itx}\ dx\\
		&= e^{i\mu_nt - \sigma_n^2t^2/2}.
	\end{align*}
	As $n$ goes to infinity we have
	\[
	\lim_{n\to \infty}\phi_n(t) = \lim_{n\to \infty}e^{i\mu_nt - \sigma_n^2t^2/2} = e^{i\mu t- \sigma^2t^2/2},
	\]
	which is the characteristic function for the random variable $Y\sim \mcal{N}(\mu, \sigma^2)$. Since the characteristic function determines the distribution of the random variable, we conclude that $X\sim \mcal{N}(\mu, \sigma^2)$.
\end{proof}

\noindent\textbf{Problem 3. }
Let $X_1, X_2, \ldots$ be independent Rademacher random variables. Let $S_n = X_1 + \cdots + X_n$.
\begin{enumerate}[(a)]
	\item Prove that the sequence $S_n/\sqrt{n}$ is unbounded almost surely.
	\begin{proof}
		
	\end{proof}
\end{enumerate}

\end{document}