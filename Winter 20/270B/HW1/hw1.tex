\documentclass[11pt,letterpaper]{report}
\usepackage{amssymb,amsfonts,color,graphicx,amsmath,enumerate}
\usepackage{tikz}
\usepackage{amsthm}
\usepackage{bbm}
\usepackage{hyperref}

\newcommand{\naturals}{\mathbb{N}}
\newcommand{\integers}{\mathbb{Z}}
\newcommand{\complex}{\mathbb{C}}
\newcommand{\reals}{\mathbb{R}}
\newcommand{\mcal}[1]{\mathcal{#1}}
\newcommand{\rationals}{\mathbb{Q}}
\newcommand{\Lp}[2]{\left\|{#1}\right\|_{L^{#2}}}
\newcommand{\field}{\mathbb{F}}
\newcommand{\affine}{\mathbb{A}}
\newcommand{\E}{\mathbb{E}}
\newcommand{\Prob}{\mathbb{P}}
\newcommand{\Var}{\text{Var}}
\newcommand{\ind}{\mathbbm{1}}
\newcommand{\Cov}{\text{Cov}}

\newenvironment{solution}
{\begin{proof}[Solution]}
{\end{proof}}

\voffset=-3cm
\hoffset=-2.25cm
\textheight=24cm
\textwidth=17.25cm
\addtolength{\jot}{8pt}
\linespread{1.3}

\begin{document}
\noindent{\em Liam Hardiman\hfill{January 29, 2020 (Late)} }
\begin{center}
{\bf \Large 270B - Homework 1}
\vspace{0.2cm}
\hrule
\end{center}

\noindent\textbf{Problem 1. }Let $X_i$ be i.i.d. random variables each having the Poisson distribution with mean 1, and consider $S_n = X_1 + \cdots + X_n$. Let $x\in \reals$. Show that if $k = k(n)$ is such that $(k-n)/\sqrt{n}\to x$ as $n\to \infty$, we have
\[
\sqrt{2\pi n}\Prob[S_n = k] \to \exp(-x^2/2).
\]
\begin{proof}
	First we claim that $S_n$ has Poisson distribution with mean $n$. To see this, observe that by independence we have
	\begin{equation}\label{chfsum}
		\varphi_{S_n}(t) = \E\left[e^{it(X_1 + \cdots + X_n)}\right] = \E\left[e^{itX_1}\right]\cdots \E\left[e^{itX_n}\right] = \varphi_1(t)\cdots \varphi_n(t),
	\end{equation}
	where $\varphi_j$ is the characteristic function of $X_j$. Now if the random variable $X$ has Poisson distribution with intensity $\lambda$, its characteristic function is given by
	\begin{align*}
		\E\left[e^{itX}\right] = \sum_{k=0}^\infty e^{itk}\cdot \frac{\lambda^ke^{-\lambda}}{k!} = \exp(\lambda(e^{it}-1)).
	\end{align*}
	Using this, we see that
	\[
	\varphi_{S_n}(t) = \exp(e^{it}-1)^n = \exp(n(e^{it} - 1)),
	\]
	which is the characteristic function of the Poisson with intensity $\lambda$. Since a distribution is determined by its characteristic function, we conclude that $S_n$ has Poisson distribution with intensity $\lambda$.
\end{proof}

\end{document}