%% Please change the file name by replacing N with the apporpriate number
%% corresponding to the current homework and XX with your initials.
%% https://www.math.uci.edu/~gpatrick/jsOnline/hw1.html

\documentclass[11pt,letterpaper]{report}
\usepackage{amssymb,amsfonts,color,graphicx,amsmath,enumerate}
\usepackage{tikz} %This package offers the ability to draw pictures
\usepackage{amsthm}

\newcommand{\naturals}{\mathbb{N}}
\newcommand{\integers}{\mathbb{Z}}
\newcommand{\complex}{\mathbb{C}}
\newcommand{\reals}{\mathbb{R}}
\newcommand{\exreals}{\overline{\mathbb{R}}}
\newcommand{\mcal}[1]{\mathcal{#1}}
\newcommand{\mable}{measurable}
\newcommand{\quats}{\mathbb{H}}
\newcommand{\rationals}{\mathbb{Q}}
\newcommand{\norm}{\trianglelefteq}
\newcommand{\Aut}{\text{Aut}}
\newcommand{\disk}{\mathbb{D}}
\newcommand{\halfplane}{\mathbb{H}}
\newcommand{\Lp}[2]{\left\|{#1}\right\|_{L^{#2}}}
\newcommand{\supp}[1]{\text{supp}({#1})}
\newcommand{\Hom}[2]{\text{Hom}_{{#1}}({#2})}
\newcommand{\tr}{\text{tr}}
\newcommand{\field}[1]{\mathbb{F}_{{#1}}}
\newcommand{\Gal}[1]{\text{Gal}\left({#1}\right)}
\newcommand{\esssup}{\text{ess sup }}
\newcommand{\essinf}{\text{ess inf }}
\newcommand{\affine}{\mathbb{A}}

\newenvironment{solution}
{\begin{proof}[Solution]}
{\end{proof}}

\theoremstyle{remark}
\newtheorem*{claim}{Claim}

\voffset=-3cm
\hoffset=-2.25cm
\textheight=24cm
\textwidth=17.25cm
\addtolength{\jot}{8pt}
\linespread{1.3}

\begin{document}
%\noindent{\em Liam Hardiman\hfill{Date} }
% Please give relevant information
\begin{center}
{\bf \Large MATH 2B - Week 1 Discussion Correction} %Replace N with the appropriate number
\vspace{0.2cm}
\hrule
\end{center}

\begin{claim}
	If $f$ is increasing (and let's assume continuous) on $[a,b]$ then using $n$ rectangles with left endpoints to approximate the area under the graph of $f$ will yield an underestimate.
\end{claim}
\begin{proof}
	This actually has nothing to do with concavity like I said in discussion. If we subdivide the interval $[a,b]$ into $n$ subintervals of equal width, each subinterval with have width equal to the length of the whole interval divided by the number of subintervals:  $\frac{b-a}{n}$. Call this width $\Delta x = \frac{b-a}{n}$. We're using the left endpoints of each subinterval to determine the heights of our rectangles, whose areas we call $A_1$, $A_2$, and so on. The area under $f$ is roughly the sum of the areas $A_1$, $A_2$, up to $A_n$:
	% \begin{align*}
	% \text{Area under $f$} &\approx A_1 + A_2 + A_3 + \cdots + A_n\\
	% &= f(a)\Delta x + f(a + \Delta x)\Delta x + f(a + 2\Delta x)\Delta x + \cdots + f(a + (n-1)\Delta x)\Delta x
	% \end{align*}
	\begin{align*}
	\text{Area under $f$} &\approx A_1 &+& A_2 &+& A_3 &+& \cdots &+& A_n\\
	&= f(a)\Delta x &+& f(a + \Delta x)\Delta x &+& f(a + 2\Delta x)\Delta x &+& \cdots &+& f(a + (n-1)\Delta x)\Delta x.
	\end{align*}

	\noindent Why is this an underestimate? Call the true area under $f$ over the first subinterval $\tilde{A}_1$, over the second $\tilde{A}_2$, and so on. We clearly have that the true area under $f$ over the whole interval $[a,b]$ is the sum of these smaller true areas. How do the true areas $\tilde{A}_i$ compare to the left-endpoint areas, $A_i$? Well the true area under any continuous curve (not just the graph of an increasing function) over an interval is at least as big as the smallest value the function takes over the interval times the width of the interval:
	\begin{equation}\label{bound}
	\text{True area} \geq (\text{smallest value the function takes}) \cdot (\text{width of interval}).
	\end{equation}
	If $f$ is increasing on the subinterval $[a, a+ \Delta x]$ then the smallest value $f$ takes here must be $f(a)$. Using (\ref{bound}), we then have
	\begin{align*}
		\text{True area under $f$ on $[a, a+\Delta x]$} &= \tilde{A_1}\\
		&\geq (\text{smallest value $f$ takes on $[a, a+\Delta x]$})\cdot (\text{width of }[a, a+\Delta x])\\
		&= f(a)\Delta x.
	\end{align*}
	The same logic says that $\tilde{A}_2 \geq f(a+\Delta x)\Delta x$ and so on. Putting it all together we have
	\begin{align*}
		\text{True area under $f$ on }[a, a+\Delta x] &= \tilde{A}_1 + \tilde{A}_2 + \cdots + \tilde{A}_n\\
		&\geq f(a)\Delta x + f(a+\Delta x)\Delta x + \cdots + f(a + (n-1)\Delta x)\Delta x\\
		&= \text{Left endpoint approximation}.
	\end{align*}
\end{proof}


\end{document}