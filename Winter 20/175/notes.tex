\documentclass[11pt,letterpaper]{report}
\usepackage{amssymb,amsfonts,color,graphicx,amsmath,enumerate, hyperref}
\usepackage{tikz}
\usepackage{amsthm}

\newcommand{\naturals}{\mathbb{N}}
\newcommand{\integers}{\mathbb{Z}}
\newcommand{\complex}{\mathbb{C}}
\newcommand{\reals}{\mathbb{R}}
\newcommand{\mcal}[1]{\mathcal{#1}}
\newcommand{\rationals}{\mathbb{Q}}
\newcommand{\Aut}{\text{Aut}}
\newcommand{\Lp}[2]{\left\|{#1}\right\|_{L^{#2}}}
\newcommand{\tr}{\text{tr}}
\newcommand{\field}{\mathbb{F}}

\newtheorem{theorem}{Theorem}[section]
\newtheorem{corollary}{Corollary}[theorem]
\newtheorem{lemma}[theorem]{Lemma}
\newtheorem*{claim*}{Claim}

\theoremstyle{definition}
\newtheorem{definition}{Definition}[section]

\theoremstyle{remark}
\newtheorem{remark}{Remark}[section]

\newenvironment{solution}
{\begin{proof}[Solution]}
{\end{proof}}

\voffset=-3cm
\hoffset=-2.25cm
\textheight=24cm
\textwidth=17.25cm
\addtolength{\jot}{8pt}
\linespread{1.3}

\begin{document}
\begin{center}
{\bf \Large Combinatorics Discussion Notes}
\vspace{0.2cm}
\hrule
\end{center}

\section{Week 1, Discussion 1}

\noindent\textbf{Problem 1. }How many ways are there to place $n+2$ distinct balls into $n$ bins?
\begin{solution}
    There are two case. In case 1 a single bin contains three balls, while in case 2 there are two bins that contain two balls. Let's count how many in each case.\\
    
    \noindent In case 1, there are
    $$
    (n+2)(n+1)\cdots ((n+2)-(n-1)) = \frac{1}{2}(n+2)!
    $$
    ways to place one ball into each bin. There are $n$ choices for where to place the remaining two balls, but we have to divide by 3 since there are three ways to choose which ball was first placed into this bin. The total number of configurations in this case is then
    $$\frac{1}{6}n(n+2)!.$$
    
    \noindent In case 2, there are again $\frac{1}{2}(n+2)!$ ways to place the first $n$ balls. There are $n(n-1)$ ways to choose where the other two go. Finally, we divide by 4 since both of the bins with two balls have two options for which ball went in first. The number of configurations here is then
    $$\frac{1}{8}n(n-1)(n-2)!.$$
    
    \noindent In total, the number of configurations is
    $$\frac{1}{6}n(n+2)! + \frac{1}{8}n(n-1)(n-2)!.$$
    We'll talk about this problem again after you learn about binomial coefficients.
\end{solution}

\noindent\textbf{Problem 2. }How many functions $f: [2n]\to [n]$ are there such that each element in $[n]$ has exactly two preimages?

\begin{solution}
    Consider an arbitrary ordering of $[2n]$. Define $f$ to be the function that sends the first pair in our ordering to $1$, the second pair to $2$, and so on. Permuting any pair in this ordering gives the same function and there are $n$ pairs to permute. There are $(2n)!$ orderings of $[2n]$ and $2^n$ ways to permute the pairs, so there are $\frac{(2n)!}{2^n}$ such functions.
\end{solution}

\noindent\textbf{Problem 3. }In how many ways can one partition the set $[2n]$ into pairs? This problem's solution is closely related to the previous one.

\section{Week 1, Discussion 2}
\newcommand{\GL}{\text{GL}}
\newcommand{\SL}{\text{SL}}
\noindent\textbf{Problem. }Fix a prime $p$. How many invertible $n\times n$ matrices are there over $\field_p$? In other words, find $|\GL_n(\field_p)|$.
\begin{solution}
	We use the elementary fact from linear algebra that a matrix is invertible if and only if its rows are linearly independent. We proceed by a counting argument.\\

	\noindent Suppose $A\in \field_p^{n\times n}$ is invertible. How many possibilities are there for the first row of $A$? We need the rows to form an independent set, so the first row can be anything but the zero vector, which gives $p^n - 1$ possible first rows ($p^n$ row vectors in total, minus the one ``bad'' vector).\\

	\noindent How many possible second rows? Whatever the second row is, it must be independent of the first row, so it can be anything but a multiple of the first row. There are $p$ possible multiples of the first row since we're working in $\field_p$, so there are $p^n - p$ possible second rows.\\

	\noindent How many possible $k$-th rows? The $k$-th row can be anything but a linear combination of the $k-1$ rows that came before it. A linear combination of the rows $v_1, \ldots v_{k-1}$ looks like
	\[
	c_1v_1 + c_2v_2 + \cdots +c_{k-1}v_{k-1},
	\]
	where the $c_i$'s come from $\field_p$. There are $k-1$ constants to choose, each of which can take any value in $\field_p$, so there are $p^{k-1}$ linear combinations in total. We want to get rid of these, so there are $p^n - p^{k-1}$ possible $k$-th rows.\\

	\noindent Now we put it all together.
	\begin{align*}
	|\GL_n(\field_p)| &= \#\{\text{possible first rows}\} \cdots \#\{\text{possible $n$-th rows}\}\\
	&= (p^n-1)(p^n-p)\cdots (p^n-p^{n-1})\\
	&= \prod_{j=0}^{n-1}(p^n-p^j).
	\end{align*}

BELOW YOU WORK TOO HARD IN MY OPINION. TAKE ANY MATRIX M AND MULTIPLY THE FIRST ROW BY C. THEN THE NEW MATIRX HAS DETERMINANT = OLD TIMES C. IN PARTICULAR, IF DET IS NOT 0, THEN THERE IS A UNIQUE CONSTANT TO MULTIPLY THE FIRST ROW IN ORDER TO GET DET 1. THEREFORE, THE PROPORTION OF DET=1 MATRICES IS $ 1/(p-1)$.
	\noindent From here you answer other questions like how big is $\SL_n(\field_p)$, the set of $n\times n$ matrices with determinant 1. Consider the map $\det: \GL_n(\field_p)\to \field_p^\times$. Since the determinant is multiplicative, the map $\det$ is a group homomorphism. By the first isomorphism theorem, $\GL_n(\field_p)/\ker \det \cong \field_p^\times$. The kernel of $\det$ is the set of matrices that get sent to $1$, which is exactly $\SL_n(\field_p)$. We then have $|\GL_n(\field_p)|/|\SL_n(\field_p)| = |\field_p^\times| = p-1$. Rearranging gives
	\[
	|\SL_n(\field_p)| = \frac{1}{p-1}|\GL_n(\field_p)|.
	\]
\end{solution}

\section{Week 2, Discussion 1}
\begin{enumerate}

	\item What's wrong with this proof?
	\begin{claim*}
		$n(n+1)$ is an odd number for every natural $n$.
	\end{claim*}
	\begin{proof}
		Suppose this is true for $n$; we prove it for $n+1$. Some quick algebra shows that
		\[
		(n+1)(n+2) = n(n+1) + 2(n+1).
		\]
		By the induction hypothesis, $n(n+1)$ is odd. Since $2(n+1)$ is clearly even, we must have that $(n+1)(n+2)$ is odd.
	\end{proof}
	\vfill

	\item What's wrong with this proof?
	\begin{claim*}
		If we have $n$ lines in the plane, no two of which are parallel, then they all go through one point.
	\end{claim*}
	\begin{proof}
		This is clearly true for one or two lines by definition. Suppose that it is true for any set of $n$ lines and let $S = \{\ell_1, \ell_2, \ell_3, \ell_4,  \ldots, \ell_{n+1}\}$ be a set of $n+1$ lines in the plane, no two of which are parallel. Delete the line $\ell_3$ to obtain a set $S'$ of $n$ lines, no two of which are parallel. By the induction hypothesis, the lines in $S'$ must all pass through some point $P$. In particular, $\ell_1$ and $\ell_2$ pass through $P$.\\

		\noindent Now put $\ell_3$ back and delete $\ell_4$ instead to get another set $S''$ of $n$ lines, no two of which are parallel. Again by the induction hypothesis, they must all pass through some point $Q$. In particular, $\ell_1$ and $\ell_2$ pass through $Q$. But $\ell_1$ and $\ell_2$ pass through $P$. Since two lines can pass through at most one point, we must have $P = Q$. But then $\ell_3$ goes through $P$, so all the lines in $S$ go through $P$.
	\end{proof}

	\vfill

	\item What is the following sum? Prove your guess by induction.
	\[
	\frac{1}{1\cdot 2} + \frac{1}{2\cdot 3} + \frac{1}{3\cdot 4} + \cdots + \frac{1}{(n-1)n}.
	\]

	\vfill

	\item What is the following sum? Give an inductive proof and a combinatorial proof.
	\[
	0\cdot \binom{n}{0} + 1\cdot \binom{n}{1} + 2\cdot \binom{n}{2} + \cdots + (n-1)\cdot\binom{n}{n-1} + n\cdot \binom{n}{n}
	\]
	\vfill\pagebreak

	\item Prove the following identity. Try to give a combinatorial proof.
	\[
	\binom{n}{k} = \binom{n-1}{k} + \binom{n-1}{k-1}.
	\]

	\vfill

	\item Prove the following identity.
	\[
	\binom{2}{2} + \binom{3}{2} + \cdots + \binom{n}{2} = \binom{n+1}{3}.
	\]

	\vfill

	\item We select 38 even positive integers, all less than 1000. Prove that there will be two of them whose difference is at most 26.

	\vfill

	\item Prove that in a group of $n$ people, at least two will have the same number of friends in that group.

	\vfill

	\item Let $S$ be a finite set. Prove that $f: S\to S$ is injective (one-to-one) if and only if it is surjective (onto).\\

	\vfill

	\item (If you've seen 120B) Prove that any finite integral domain is a field.

	\vfill

	\item (Hard) Every element of the infinite grid $\integers\times \integers$ is colored red, green, or blue. Prove that there is a rectangle whose corners are all the same color.
	\vfill

\end{enumerate}
\pagebreak

\begin{center}
{\bf \Large Math 175 - Homework 3}
\vspace{0.2cm}
\hrule
\end{center}

\begin{enumerate}
	\item There are twenty-five beads in a bag, each colored red, blue or green. Prove that there are at least nine beads of the same color.
	\vfill

	\item Let $A = \{a_1, a_2, a_3, a_4, a_5\}$ be a set of five positive integers. Show that for any permutation $a_{i_1}, a_{i_2}, a_{i_3}, a_{i_4}, a_{i_5}$ of $A$, the product
	\[
	(a_{i_1}-a_1)(a_{i_2}-a_2)(a_{i_3}-a_3)(a_{i_4}-a_4)(a_{i_5}-a_5).
	\]
	is always even.
	\vfill
	\item Let $K_n$ be the complete graph on $n$ vertices, i.e. the graph with $n$ vertices where every pair of vertices is connected by an edge. 
	\begin{enumerate}
		\item Suppose we color each edge of $K_6$ either red or blue. Prove that there is at least one monochromatic triangle. A triangle is monochromatic if all of its edges have the same color.
		\item Is part (a) still true for $K_5$?
		\item Prove that in any coloring of $K_6$ we actually have at least \textit{two} monochromatic triangles. \textit{Hint: Count the complement.}
	\end{enumerate}
	\vfill

	\item \begin{enumerate}
		\item Is it possible to cover a chessboard (an 8 by 8 grid where squares alternate between two colors) with dominoes that cover exactly two squares?
		\item What if we delete two white squares?
	\end{enumerate}

	\vfill
	
	\item \begin{enumerate}
		\item Let $A$ be a set of $m$ positive integers where $m\geq 1$. Show that there exists a nonempty subset $B$ of $A$ such that the sum $\sum_{x\in B}x$ is divisible by $m$.
		\item Let $X\subseteq \{1, 2, \ldots, 99\}$ and $|X| = 10$. Show that it is possible to select two disjoint nonempty proper subsets $Y, Z$ of $X$ such that $\sum_{y\in Y}y = \sum_{z\in Z}z$.
	\end{enumerate}
	\vfill
\end{enumerate}

\pagebreak

\begin{center}
{\bf \Large Math 175 - Homework 4 Ideas}
\vspace{0.2cm}
\hrule
\end{center}

\begin{enumerate}

	\item The $n\times n$ determinant $a_n$ is defined for $n\geq 1$ by
    \[
    a_n = \begin{vmatrix}
    	p & p-q & 0 & 0 & \cdots & 0 & 0\\
    	q & p & p-q & 0 & \cdots & 0 & 0\\
    	0 & q & p & p-q & \cdots & 0 & 0\\
    	9 & 9 & q & p & \cdots & 0 & 0\\
    	\vdots & \vdots & \vdots & \vdots & \ddots & \vdots & \vdots\\
    	0 & 0 & 0 & 0 & \cdots & p & p-q\\
    	0 & 0 & 0 & 0 & \cdots & q & p\\   	
    \end{vmatrix},
    \]
    where $p$ and $q$ are distinct nonzero constants. Find a recurrence relation for $a_n$.

    \vfill

    \item Find the number of ways to seat $n$ penguin couples around a table in each of the following cases.
    \begin{enumerate}
    	\item Male and female penguins alternate.
    	\vfill
    	\item Every female penguin is next to her mate.
    \end{enumerate}

    \vfill

    \item Let $d_n$ be the number of bijections $f:[n]\to [n]$ so that $f(k)\neq k$ for all $k\in [n]$. Find a recurrence relation for $d_n$.
    
    \vfill

    \item Let $1\leq r\leq n$ and consider all $r$-element subsets of the set $\{1, 2, \ldots, n\}$. Each of these subsets has a largest element. Find the arithmetic mean of these largest elements.
    \vfill
    
    \item A triangulation of an $n$-gon $P_n$, $n\geq 3$, is a subdivision of $P_n$ into triangles by means of nonintersecting diagonals of $P_n$. Let $a_0 = 1$ and for $n\geq 1$, let $t_n$ be the number of different triangulations of an $(n+2)$-gon. Find a recurrence relation for $t_n$.
    
    \vfill
    
\end{enumerate}

\end{document}