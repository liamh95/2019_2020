\documentclass[11pt,letterpaper]{report}
\usepackage{amssymb,amsfonts,color,graphicx,amsmath,enumerate}
\usepackage{tikz}
\usepackage{amsthm}

\newcommand{\naturals}{\mathbb{N}}
\newcommand{\integers}{\mathbb{Z}}
\newcommand{\complex}{\mathbb{C}}
\newcommand{\reals}{\mathbb{R}}
\newcommand{\mcal}[1]{\mathcal{#1}}
\newcommand{\rationals}{\mathbb{Q}}
\newcommand{\Lp}[2]{\left\|{#1}\right\|_{L^{#2}}}
\newcommand{\field}{\mathbb{F}}
\newcommand{\affine}{\mathbb{A}}
\newcommand{\E}{\mathbb{E}}
\newcommand{\Prob}{\mathbb{P}}
\newcommand{\Var}{\text{Var}}
\newcommand{\ind}{\mathbbm{1}}
\newcommand{\Cov}{\text{Cov}}

\newtheorem{theorem}{Theorem}[section]
\newtheorem{corollary}{Corollary}[theorem]
\newtheorem{lemma}[theorem]{Lemma}
\newtheorem*{claim*}{Claim}

\newenvironment{solution}
{\begin{proof}[Solution]}
{\end{proof}}

\voffset=-3cm
\hoffset=-2.25cm
\textheight=24cm
\textwidth=17.25cm
\addtolength{\jot}{8pt}
\linespread{1.3}

\begin{document}
\begin{center}
{\bf \Large Math 180B - Induction}
\vspace{0.2cm}
\hrule
\end{center}

\begin{enumerate}
	\item Prove the following equations by induction. In each case, $n$ is a positive integer.
	\begin{enumerate}[(a)]
		\item $1+4+7 + \cdots + (3n-2) = \frac{n(3n-1)}{2}$.
		\item $1+x+x^2 + \cdots + x^n = \frac{1-x^{n+1}}{1-x}$. What happens when $x = 1$?
		\item $\lim_{x\to \infty}\frac{x^n}{e^x} = 0$.
		\item $n! = \int_0^\infty x^ne^{-x}\ dx$.
	\end{enumerate}

	\item Prove the following inequalities by induction. $n$ is a positive integer.
	\begin{enumerate}
		\item $1+\frac{1}{2} + \frac{1}{3} + \cdots + \frac{1}{2^n} \geq 1+\frac{n}{2}$.
		\item $\binom{2n}{n}<4^n$.
		\item $n!\leq n^n$.
	\end{enumerate}

	\item Prove that the sum of the angles of a convex $n$-gon $(n\geq 3)$ is $180(n-2)$ degrees or $\pi(n-2)$ radians.

	\item Prove the binomial theorem
	\[
	(x+y)^n= \sum_{k=0}^n\binom{n}{k}x^{n-k}y^k.
	\]

	\item What's wrong with this proof?
	\begin{claim*}
		If we have $n$ lines in the plane, no two of which are parallel, then they all go through one point.
	\end{claim*}
	\begin{proof}
		This is clearly true for one or two lines by definition. Suppose that it is true for any set of $n$ lines and let $S = \{\ell_1, \ell_2, \ell_3, \ell_4,  \ldots, \ell_{n+1}\}$ be a set of $n+1$ lines in the plane, no two of which are parallel. Delete the line $\ell_3$ to obtain a set $S'$ of $n$ lines, no two of which are parallel. By the induction hypothesis, the lines in $S'$ must all pass through some point $P$. In particular, $\ell_1$ and $\ell_2$ pass through $P$.\\

		\noindent Now put $\ell_3$ back and delete $\ell_4$ instead to get another set $S''$ of $n$ lines, no two of which are parallel. Again by the induction hypothesis, they must all pass through some point $Q$. In particular, $\ell_1$ and $\ell_2$ pass through $Q$. But $\ell_1$ and $\ell_2$ pass through $P$. Since two lines can pass through at most one point, we must have $P = Q$. But then $\ell_3$ goes through $P$, so all the lines in $S$ go through $P$.
	\end{proof}
\end{enumerate}



\end{document}