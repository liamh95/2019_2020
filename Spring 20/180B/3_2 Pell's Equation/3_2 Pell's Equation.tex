\documentclass[11pt,letterpaper]{report}
\usepackage{amssymb,amsfonts,color,graphicx,amsmath,enumerate}
\usepackage{amsthm}

\newcommand{\naturals}{\mathbb{N}}
\newcommand{\integers}{\mathbb{Z}}
\newcommand{\complex}{\mathbb{C}}
\newcommand{\reals}{\mathbb{R}}
\newcommand{\mcal}[1]{\mathcal{#1}}
\newcommand{\rationals}{\mathbb{Q}}
\newcommand{\field}{\mathbb{F}}
\newcommand{\Var}{\text{Var}}
\newcommand{\ind}{\mathbbm{1}}
\newcommand{\Cov}{\text{Cov}}

\newenvironment{solution}
{\begin{proof}[Solution]}
{\end{proof}}

\voffset=-3cm
\hoffset=-2.25cm
\textheight=24cm
\textwidth=17.25cm
\addtolength{\jot}{8pt}
\linespread{1.3}

\begin{document}
\begin{center}
{\bf \Large Math 180B - Pell's Equation}
\vspace{0.2cm}
\hrule
\end{center}

\begin{enumerate}
	\item Let $d$ be a positive integer that is not a perfect square. If $k$ is any positive integer, prove that there are infinitely many solutions in integers of $x^2-dy^2 = 1$ with $k\mid y$.

	\item Find the smallest positive solution of $x^2-dy^2 = 1$ by successively substituting $y = 1, 2, 3, \ldots$ when d is (a) 7 and (b) 11.

	\item Find all positive solutions to $x^2 - 2y^2 = 1$ for which $y<250$.

	\item A Pell's equation has the form $x^2 - dy^2 = 1$ where $d$ is a positive integer that is not a perfect square. Why don't we want $d$ to be a perfect square?

	\item Consider a right triangle, the lengths of whose sides are integers. Prove that the area cannot be a perfect square.
\end{enumerate}

\end{document}