\documentclass[11pt,letterpaper]{report}
\usepackage{amssymb,amsfonts,color,graphicx,amsmath,enumerate}
\usepackage{tikz}
\usepackage{amsthm}

\newcommand{\naturals}{\mathbb{N}}
\newcommand{\integers}{\mathbb{Z}}
\newcommand{\complex}{\mathbb{C}}
\newcommand{\reals}{\mathbb{R}}
\newcommand{\mcal}[1]{\mathcal{#1}}
\newcommand{\rationals}{\mathbb{Q}}
\newcommand{\Lp}[2]{\left\|{#1}\right\|_{L^{#2}}}
\newcommand{\field}{\mathbb{F}}
\newcommand{\affine}{\mathbb{A}}
\newcommand{\E}{\mathbb{E}}
\newcommand{\Prob}{\mathbb{P}}
\newcommand{\Var}{\text{Var}}
\newcommand{\ind}{\mathbbm{1}}
\newcommand{\Cov}{\text{Cov}}

\theoremstyle{definition}
\newtheorem{definition}{Definition}
\newtheorem{exercise}{Exercise}
\newtheorem{theorem}{Theorem}

\newenvironment{solution}
{\begin{proof}[Solution]}
{\end{proof}}

\voffset=-3cm
\hoffset=-2.25cm
\textheight=24cm
\textwidth=17.25cm
\addtolength{\jot}{8pt}
\linespread{1.3}

\begin{document}
\begin{center}
{\bf \Large Math 130B - Review and Big $O$ Notation}
\vspace{0.2cm}
\hrule
\end{center}

Here are some review exercises.
\begin{exercise}
	\begin{enumerate}[(a)]
		\item Give a combinatorial explanation of the identity
		\[
		\binom{n}{r} = \binom{n}{n-r}.
		\]

		\item Determine the number of solutions to the inequality 
		\[
		x_1 + x_2 + \cdots + x_n < k,
		\]
		where each $x_i$ is a positive integer and $k> n$.

		\item If a die is rolled four times, what is the probability that 6 comes up at least once?

		\item Consider an experiment whose sample space consists of a countably infinite number of points. Show that not all points can be equally likely. Can all points have a positive probability of occurring?

		\item An urn contains $n$ white and $m$ black balls, where $n$ and $m$ are positive numbers.
		\begin{enumerate}
		 	\item If two balls are randomly withdrawn, what is the probability that they are the same color?
		 	\item If a ball is randomly withdrawn and then replaced before the second one is drawn, what is the probability that the withdrawn balls are the same color?
		 	\item Show that the probability in part (b) is always larger than the one in part (a).
		\end{enumerate} 

		% \item Two fair dice are rolled. What is the conditional probability that at least one lands on 6 given that the dice land on different numbers?

		\item Three cards are randomly selected, without replacement, from an ordinary deck of 52 playing cards. Compute the conditional probability that the first card selected is a spade given that the second and third cards are spades.

		\item 
	\end{enumerate}
\end{exercise}

The following definition gives us a rigorous way of saying one function is ``larger'' than another.
\begin{definition}
	Let $f,g:\reals\to \reals$ be functions.
	\begin{enumerate}[(a)]
		\item We write $f(x) = O(g(x))$ if there exist positive constants $c_1$ and $c_2$ such that $|f(x)| \leq c_1|g(x)|$ for all $x \geq c_2$. That is, $f$ is \textit{eventually} at most a constant multiple of $g$.
		\item We write $f(x) = \Omega(g(x))$ if there exist positive constants $c_1$ and $c_2$ such that $|f(x)| \geq c_1|g(x)|$ for all $x\geq c_2$. That is, $f$ is eventually at least a constant multiple of $g$.
	\end{enumerate}
\end{definition}

\begin{exercise}
	Show that if $f(x) = O(g(x))$ and $g(x) = O(h(x))$ then $f(x) = O(h(x))$ and that the same is true if we replace $O$ with $\Omega$.
\end{exercise}

\begin{exercise}
	\begin{enumerate}[(a)]
		\item Let $f(x) = ax+b$ and $g(x) = cx+d$ where $a,c\neq 0$. Show that $f(x) = O(g(x))$.
		\item Let $f(x) = x^a$ and $g(x) = \log_b(x)$ where $a>0$ and $b>1$. Show that $f(x) = \Omega(g(x))$.
	\end{enumerate}
\end{exercise}

These definitions can be a bit cumbersome to work with sometimes. The following is sometimes easier to check and you'll prove it on your homework.

\begin{theorem}
	\begin{enumerate}[(a)]
		\item If $\lim_{x\to \infty}\frac{|f(x)|}{|g(x)|}<\infty$ then $f(x) = O(g(x))$.
		\item If $\lim_{x\to \infty}\frac{|f(x)|}{|g(x)|} > 0$ then $f(x) = \Omega(g(x))$.
	\end{enumerate}
\end{theorem}

\begin{exercise}
	Prove the following.
	\begin{enumerate}[(a)]
		\item $x^2 + \sqrt{x} = O(x^2)$.
		\item $5+6x^2 - 37x^5 = O(x^5)$.
		\item $k^22^k = O(e^{2k})$.
		\item $N^{10}2^N = O(e^N)$.
	\end{enumerate}
\end{exercise}

We also have notation to express the idea of one function being \textit{strictly} less or greater than another.

\begin{definition}
	Let $f,g:\reals\to \reals$ be functions.
	\begin{enumerate}[(a)]
		\item We write $f(x) = o(g(x))$ if for all $c_1>0$ there exists a $c_2>0$ so that $|f(x)|\leq c_1|g(x)|$ for all $x \geq c_2$. That is, $f$ is eventually smaller than \textit{any} constant multiple of $g$.
		\item We write $f(x)= \omega(g(x))$ if for all $c_1>0$ there exists a $c_2>0$ so that $|f(x)|\geq c_1|g(x)|$ for all $x\geq c_2$. That is, $g$ is eventually greater than any multiple of $g$.
	\end{enumerate}
\end{definition}

Just like with $O$ and $\Omega$, we can take limits to show $f(x) = o(g(x))$ or $\omega(g(x))$.

\begin{theorem}
	\begin{enumerate}[(a)]
		\item $f(x) = o(g(x))$ if and only if $\lim_{x\to \infty}\frac{f(x)}{g(x)} = 0$.
		\item $f(x) = \omega(g(x))$ if and only if $\lim_{x\to \infty}\frac{|f(x)|}{|g(x)|} = \infty$.
	\end{enumerate}
\end{theorem}

\begin{exercise}
	Prove the following.
	\begin{enumerate}[(a)]
		\item If $f(x) = o(g(x))$ then $f(x) = O(g(x))$. If $f(x) = \omega(g(x))$ then $f(x) = \Omega(g(x))$. Give examples to show that the converses to these statements are false.
		\item $k^{300} = o(2^k)$.
		\item $k^{0.001} = \omega((\log k)^{375})$.
	\end{enumerate}
\end{exercise}


\end{document}