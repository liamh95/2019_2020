\documentclass[12pt]{article}  
%%Read the manual for other options. 

\pagestyle{empty} %%Eliminates page numbers
%%\input rmb_macros
%%Collect your favorite macros in a 
%%separate file

%\input amssym.def
%\input amssym
%\input mssymb
%%Defines additional symbols



\usepackage{graphics}
\usepackage{amsmath,amssymb,amsthm, multicol}
\usepackage[pdftex]{graphicx}
\usepackage{hyperref}
%%Use to include pictures. 

%\newcommand{\comment}[1]{}
%\newcommand{\sobolev}[2]{W^{#1,#2}}
\newcommand{\integers}{\mathbb{Z}}
\newcommand{\field}{\mathbb{F}}
%\newcommand{\sobolev}[2]{L^#2_#1}
%%Some examples of macros or new commands.

\addtolength{\oddsidemargin}{-.75in}
\addtolength{\evensidemargin}{-.75in}
\addtolength{\textwidth}{1.5in}
\addtolength{\topmargin}{-1in}
\addtolength{\textheight}{2.25in}
%%Set margins, defaults are ok. 

\begin{document}
\begin{flushleft} 
%%Paragraphs will not be indented in this 
%%environment
\centerline{\LARGE{Quiz 3}} 
\vspace{5 mm}
{Student ID Number:}\hfill  
%%\hfill forces following text 
%%to right margin
{Name \rule {2 in}{0.01in}}\\
Math 173B, 1PM
\\
%%gives a line of length 2in and 
%%thickness 0.01in
{Please justify all your answers}\hfill {January 30, 2020}
\\
{Please also write your full name on the back} 

\medskip
\end{flushleft}

\begin{enumerate}
	\item True or False?
	\begin{enumerate}
		\item The usual formula for the addition law on an elliptic curve works over $\integers/N \integers$ for any integer $N>0$.
		\item If $P = (x, y)$ is a point on the elliptic curve $E$ defined over $\field{2^k}$ for some $k\geq 1$, then $-P = (x, -y)$.
	\end{enumerate}

	\item Let $E$ be an elliptic curve over $\integers/N \integers$ where $N$ is a composite integer with unknown factorization. You (correctly) program a computer to add points on the curve using the usual point addition formula. You tell your program to compute $2P$ for some point $P$ on $E$ and it gives you an error. Briefly explain how you can use this error to factor $N$.
	\vfill
\end{enumerate}

%\vfill will divide page evenly
%use \begin{enumerate} environment for ordered lists
\end{document}