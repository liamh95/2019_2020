\documentclass[11pt,letterpaper]{report}
\usepackage{amssymb,amsfonts,color,graphicx,amsmath,enumerate}
\usepackage{tikz}
\usepackage{amsthm}

\newcommand{\naturals}{\mathbb{N}}
\newcommand{\integers}{\mathbb{Z}}
\newcommand{\complex}{\mathbb{C}}
\newcommand{\reals}{\mathbb{R}}
\newcommand{\mcal}[1]{\mathcal{#1}}
\newcommand{\rationals}{\mathbb{Q}}
\newcommand{\Lp}[2]{\left\|{#1}\right\|_{L^{#2}}}
\newcommand{\field}{\mathbb{F}}
\newcommand{\affine}{\mathbb{A}}
\newcommand{\E}{\mathbb{E}}
\newcommand{\Prob}{\mathbb{P}}
\newcommand{\Var}{\text{Var}}
\newcommand{\ind}{\mathbbm{1}}
\newcommand{\Cov}{\text{Cov}}

\newenvironment{solution}
{\begin{proof}[Solution]}
{\end{proof}}

\voffset=-3cm
\hoffset=-2.25cm
\textheight=24cm
\textwidth=17.25cm
\addtolength{\jot}{8pt}
\linespread{1.3}

\begin{document}
\noindent{\em Liam Hardiman\hfill{March 20, 2020} }
\begin{center}
{\bf \Large 271B - Final}
\vspace{0.2cm}
\hrule
\end{center}

\noindent\textbf{Problem 1. }
Let $B$ be a standard one-dimensional Brownian motion. Consider the SDE
\begin{equation}\label{1_sde}
	dX_t = (t^2\Sigma)dB_t,\quad X_0 = 0
\end{equation}
where $\Sigma$ is an exponentially distributed random variable with parameter $\lambda$ and independent of the Brownian motion.
\begin{enumerate}[(a)]
	\item Find $Y_t$ so that $M_t = \exp(X_t)Y_t$ is a martingale. Specify the filtration.
	\begin{solution}
		Let $\sigma_t(\omega) = t^2\Sigma(\omega)$ and let $Y_t = \exp(-\frac{1}{2}\int_0^t\sigma^2_s\ ds)$. We claim that $M_t = \exp(X_t)Y_t$ is a martingale. By It\^o's lemma we have
		\begin{align*}
			dM_t &= -\frac{1}{2}\sigma_t^2M_t\ dt + M_t\ dX_t + \frac{1}{2}M_t(dX_t)^2\\
			&= \sigma_tM_t\ dB_t\\
			&= t^2\Sigma M_t\ dB_t.
		\end{align*}
		In order for us to conclude that this is a Martingale, have to show that $t^2\Sigma M_t$ is in class I$^*$. To this end, we check the Kazamaki condition (\O ksendal, remark after exercise 4.4. I tried using Novikov's condition, which we covered in class, but the expectation wasn't finite): if the following condition holds, then $M_t$ is a martingale.
		\begin{equation}\label{Novikov}
			\E\left[\exp\left( \frac{1}{2}\int_0^T(s^2\Sigma)dB_s \right) \right]<\infty.
		\end{equation}
		We've shown on a previous homework assignment that for a deterministic function $f(s)$,
		\[
		\int_0^tf(s)dB_s \sim \mcal{N}\left(0, \int_0^tf^2(s)ds \right).
		\]
		From this we conclude that
		\[
		\exp\left(\frac{1}{2}\int_0^Ts^2\Sigma\ dB_s\right)\sim \exp(\Sigma g),
		\]
		where $g\sim \mcal{N}(0, T^5/20)$. Since $\Sigma$ is independent of the Brownian motion, it is also independent of $g$, hence
		\begin{align*}
			\E\left[\exp\left( \frac{1}{2}\int_0^T(s^2\Sigma)dB_s \right) \right] &= \E[e^{\Sigma g}] = \E[e^\Sigma]\cdot \E[e^g].
		\end{align*}
		This quantity is finite if $\lambda>1$. (I needed $\lambda > 1$ for $\E[e^\Sigma]<\infty$. This seems arbitrary, however. Is it still true without this?) Since Kazamaki's condition holds, $M_t$ is indeed a martingale with respect to the filtration generated by the Brownian motion.
	\end{solution}

	\item Compute the variance of $M_t$.
	\begin{solution}
		The variance is given by $\Var[M_t] = \E[M_t^2] - \E[M_t]^2$. Since $X_0 = 0$ a.s. and $Y_0 = 1$ a.s., $M_t = 1$ a.s. To compute $\E[M_t^2]$, we use the It\^o isometry.
		\begin{align*}
			\E[M_t^2] &= \E\left[ \left(1 + \int_0^t(s^2\Sigma)M_s\ dB_s  \right)^2 \right]\\
			&= 1 + \E\left[\int_0^t (s^2\Sigma)^2M_s^2\ ds \right]
		\end{align*}
	\end{solution}
\end{enumerate}

\noindent\textbf{Problem 2. }
Consider the Ornstein-Uhlenbeck process
\begin{equation}\label{O-U}
	dr_t = a(\overline{r}-r_t)dt + \sigma dB_t,
\end{equation}
where $a$, $\overline{r}$, $\sigma$ are constants. This process models an interest rate. The price of a zero-coupon bond at time $t$ when paying 1 at maturity $T$ is
\[
P(t, x, T) = \E\left[\exp\left(-\int_t^Tr_s\ ds\right)\ |\ r_t = x \right].
\]
\begin{enumerate}[(a)]
	\item Derive the Feynman-Kac formula for the bond price:
	\begin{equation}\label{F-K}
		\begin{cases}
			\partial_tP + \frac{1}{2}\sigma^2\partial_x^2P + a(\overline{r}-x)\partial_xP-xP = 0\\
			P(T, x, T) = 1
		\end{cases}.
	\end{equation}
	\begin{solution}
		I think the idea here is to use It\^o and the Kolmogorov backward equation.
	\end{solution}
\end{enumerate}


\noindent\textbf{Problem 3. }
Let $v$ be a continuous scalar valued process satisfying
\[
0\leq v(t)\leq \alpha(t)+\beta\int_0^tv(s)\ ds;\ 0\leq t\leq T,
\]
with $\beta\geq 0$ and $\alpha$ integrable. Show that
\[
v(t)\leq \alpha(t)+\beta\int_0^t\alpha(s)e^{\beta(t-s)}ds,\ 0\leq t\leq T.
\]
Can you relax the assumption about continuity?
\begin{solution}
	Define the function
	\begin{equation}\label{eff}
	F(s) = e^{-\beta s}\cdot \beta\int_0^sv(u)\ du.
	\end{equation}
	We differentiate:
	\[
	F'(s) = \beta e^{-\beta s}\left(v(s)-\beta\int_0^sv(u)\ du\right) \leq \beta\alpha(s)e^{-\beta s}.
	\]
	Integrating from 0 to $t$ gives
	\[
	F(t) \leq \beta\int_0^t\alpha(s)e^{-\beta s}\ ds.
	\]
	Now we have from (\ref{eff}) and the above 
	\[
	\beta \int_0^tv(s)\ ds = e^{\beta t}F(t)\leq \beta\int_0^t\alpha(s)e^{\beta(t-s)}ds.
	\]
	Finally, we know $\beta\int_0^tv(s)\ ds \geq v(t)-\alpha(t)$, so the desired inequality follows.
\end{solution}

\noindent\textbf{Problem 4. }
Let $B_t = B^{(1)}_t + iB^{(2)}_t$ be a complex Brownian motion.
\begin{enumerate}[(a)]
	\item Let $F(z) = u(z)+iv(z)$ be analytic and define
	\[
	Z_t = F(B_t).
	\]
	Prove that
	\[
	dZ_t = F'(B_t)\ dB_t,
	\]
	where $F'$ is the complex derivative of $F$.
	\begin{proof}
		We assume the component Brownian motions are independent. By It\^o we have
		\[
		dB_t = dB^{(1)}_t + idB^{(2)}_t.
		\]
		Write $Z_t$ in terms of the component functions of $F$:
		\[
		Z_t = u(B^{(1)}_t, B^{(2)}_t) + iv(B^{(1)}_t, B^{(2)}_t).
		\]
		By It\^o's lemma we have (suppressing the dependence on $B^{(1)}$ and $B^{(2)}$)
		\begin{align*}
			dZ_t = (u_x+iv_x)dB^{(1)}_t + (u_y+iv_y)dB^{(2)}_t + \frac{1}{2}[(u_{xx}+iv_{xx})dt + (u_{yy}+iv_{yy})dt].
		\end{align*}
		Now the components of an analytic function are harmonic, so the bracketed term vanishes. Applying the Cauchy-Riemann equations gives
		\[
		dZ_t = (u_x+iv_x)dB_t = F'(z)dB_t.
		\]
	\end{proof}

	\item Solve the complex SDE
	\[
	dZ_t = \alpha Z_tdB_t,
	\]
	where $\alpha$ is a constant.
	\begin{solution}
		Pretending that this is a real ODE, we guess that the solution will be exponential. Indeed, by part (a) we have
		\[
		d(e^{\alpha B_t}) = \alpha e^{\alpha B_t}dB_t.
		\]
		Thus, $Z_t = Z_0 + e^{\alpha B_t}$ solves the SDE.
	\end{solution}
\end{enumerate}

\end{document}