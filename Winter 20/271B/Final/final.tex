\documentclass[11pt,letterpaper]{report}
\usepackage{amssymb,amsfonts,color,graphicx,amsmath,enumerate}
\usepackage{tikz}
\usepackage{amsthm}

\newcommand{\naturals}{\mathbb{N}}
\newcommand{\integers}{\mathbb{Z}}
\newcommand{\complex}{\mathbb{C}}
\newcommand{\reals}{\mathbb{R}}
\newcommand{\mcal}[1]{\mathcal{#1}}
\newcommand{\rationals}{\mathbb{Q}}
\newcommand{\Lp}[2]{\left\|{#1}\right\|_{L^{#2}}}
\newcommand{\field}{\mathbb{F}}
\newcommand{\affine}{\mathbb{A}}
\newcommand{\E}{\mathbb{E}}
\newcommand{\Prob}{\mathbb{P}}
\newcommand{\Var}{\text{Var}}
\newcommand{\ind}{\mathbbm{1}}
\newcommand{\Cov}{\text{Cov}}

\newenvironment{solution}
{\begin{proof}[Solution]}
{\end{proof}}

\voffset=-3cm
\hoffset=-2.25cm
\textheight=24cm
\textwidth=17.25cm
\addtolength{\jot}{8pt}
\linespread{1.3}

\begin{document}
\noindent{\em Liam Hardiman\hfill{March 20, 2020} }
\begin{center}
{\bf \Large 271B - Final}
\vspace{0.2cm}
\hrule
\end{center}

\noindent\textbf{Problem 1. }
Let $B$ be a standard one-dimensional Brownian motion. Consider the SDE
\begin{equation}\label{1_sde}
	dX_t = (t^2\Sigma)dB_t,\quad X_0 = 0
\end{equation}
where $\Sigma$ is an exponentially distributed random variable with parameter $\lambda$ and independent of the Brownian motion.
\begin{enumerate}[(a)]
	\item Find $Y_t$ so that $M_t = \exp(X_t)Y_t$ is a martingale. Specify the filtration.
	\begin{solution}
		Let $\sigma_t(\omega) = t^2\Sigma(\omega)$ and let $Y_t = \exp(-\frac{1}{2}\int_0^t\sigma^2_s\ ds)$. We claim that $M_t = \exp(X_t)Y_t$ is a martingale. By It\^o's lemma we have
		\begin{align*}
			dM_t &= -\frac{1}{2}\sigma_t^2M_t\ dt + M_t\ dX_t + \frac{1}{2}M_t(dX_t)^2\\
			&= \sigma_tM_t\ dB_t\\
			&= t^2\Sigma M_t\ dB_t.
		\end{align*}
		In order for us to conclude that this is a Martingale, have to show that $t^2\Sigma M_t$ is in class I$^*$. To this end, we check the Kazamaki condition (\O ksendal, remark after exercise 4.4. I tried using Novikov's condition, which we covered in class, but the expectation wasn't finite): if the following condition holds, then $M_t$ is a martingale.
		\begin{equation}\label{Novikov}
			\E\left[\exp\left( \frac{1}{2}\int_0^T(s^2\Sigma)dB_s \right) \right]<\infty.
		\end{equation}
		We've shown on a previous homework assignment that for a deterministic function $f(s)$,
		\[
		\int_0^tf(s)dB_s \sim \mcal{N}\left(0, \int_0^tf^2(s)ds \right).
		\]
		From this we conclude that
		\[
		\exp\left(\frac{1}{2}\int_0^Ts^2\Sigma\ dB_s\right)\sim \exp(\Sigma g),
		\]
		where $g\sim \mcal{N}(0, T^5/20)$. Since $\Sigma$ is independent of the Brownian motion, it is also independent of $g$, hence
		\begin{align*}
			\E\left[\exp\left( \frac{1}{2}\int_0^T(s^2\Sigma)dB_s \right) \right] &= \E[e^{\Sigma g}] = \E[e^\Sigma]\cdot \E[e^g].
		\end{align*}
		This quantity is finite if $\lambda>1$. (I needed $\lambda > 1$ for $\E[e^\Sigma]<\infty$. This seems arbitrary, however. Is it still true without this?) Since Kazamaki's condition holds, $M_t$ is indeed a martingale with respect to the filtration generated by the Brownian motion.
	\end{solution}

	\item Compute the variance of $M_t$.
	\begin{solution}
		The variance is given by $\Var[M_t] = \E[M_t^2] - \E[M_t]^2$. Since $X_0 = 0$ a.s. and $Y_0 = 1$ a.s., $M_t = 1$ a.s. To compute $\E[M_t^2]$ we use the It\^o isometry:
		% \begin{align*}
		% 	\E[M_t^2] &= \E\left[\left(\int_0^tdM\right)^2 \right]\\
		% 	&= \E\left[\left(\int_0^ts^2\Sigma\ dB_s \right)^2 \right]\\
		% 	&= \E\left[\Sigma^2\int_0^ts^4\ ds \right]\\
		% 	&= 
		% \end{align*}
		\begin{align*}
			\E[M_t^2] &= \E\left[ \left(1 + \int_0^t(s^2\Sigma)M_s\ dB_s  \right)^2 \right]\\
			&= 1 + \E\left[\int_0^t (s^2\Sigma)^2M_s^2\ ds \right]
		\end{align*}
		This gives
		\[
		\Var[M_t] = \E \left[ \int_0^t(s^2\Sigma)^2M_s^2\ ds \right].
		\]
	\end{solution}

	\item Find a bound for $\Prob[\sup_{0<s<t}|M_t|>\epsilon]$.
	\begin{solution}
		We use Doob's martingale inequality:
		\begin{align*}
			\Prob\left[ \sup_{0\leq s\leq t} |M_t| \geq \epsilon\right] &\leq \frac{1}{\epsilon^2}\cdot \E[|M_t|^2]\\
			&= \frac{1}{\epsilon^2}\left(1 + \E\left[\int_0^t(s^2\Sigma)^2M_s^2\ ds \right] \right).
		\end{align*}
	\end{solution}

\end{enumerate}

\noindent\textbf{Problem 2. }
Consider the Ornstein-Uhlenbeck process
\begin{equation}\label{O-U}
	dr_t = a(\overline{r}-r_t)dt + \sigma dB_t,
\end{equation}
where $a$, $\overline{r}$, $\sigma$ are constants. This process models an interest rate. The price of a zero-coupon bond at time $t$ when paying 1 at maturity $T$ is
\[
P(t, x, T) = \E\left[\exp\left(-\int_t^Tr_s\ ds\right)\ |\ r_t = x \right].
\]
\begin{enumerate}[(a)]
	\item Derive the Feynman-Kac formula for the bond price:
	\begin{equation}\label{F-K}
		\begin{cases}
			\partial_tP + \frac{1}{2}\sigma^2\partial_x^2P + a(\overline{r}-x)\partial_xP-xP = 0\\
			P(T, x, T) = 1
		\end{cases}.
	\end{equation}
	\begin{solution}
		We have
		\begin{equation}\label{nocnd}
		d\ \exp\left(-\int_t^Tr_s\ ds \right) = r_t\cdot \exp\left(-\int_t^Tr_s\ ds \right)\ dt.
		\end{equation}
		On the other hand, It\^o tells us that
		\begin{align*}
		dP &= \partial_tP\ dt + \partial_xP\ dr_t + \frac{1}{2}\partial_x^2P\cdot (dr_t)^2\\
		&= \partial_tP\ dt + \partial_xP[a(\overline{r}-r_t)dt + \sigma\ dB_t] + \frac{1}{2}\sigma^2\partial^2_xP\ dt.
		\end{align*}
		We condition (\ref{nocnd}) on $r_t = x$ and equate this to the above quantity to get
		\[
		\partial_tP\ dt + \partial_xP[a(\overline{r}-r_t)dt + \sigma\ dB_t] + \frac{1}{2}\sigma^2\partial^2_xP\ dt = r_tP\ dt.
		\]
		Integrating both sides from $t$ to $T$ and using $r_t = x$, we obtain (I wasn't sure how to get rid of the $\sigma\ dB_t$ term)
		\[
		\partial_tP\ dt + a(\overline{r}-x)\partial_xP + \frac{1}{2}\sigma^2\partial^2_xP - xP = 0.
		\]
	\end{solution}

	\item Solve this PDE for the price.
	\begin{solution}
		We guess a solution of the form $P(t, x) =  A(\tau)\exp[B(\tau)x]$, where $\tau = T-t$. Plugging this into the PDE gives
		\[
			A'(\tau)e^{xB(\tau)}+xA(\tau)B'(\tau)e^{xB(\tau)} = a(\overline{r}-x)A(\tau)B(\tau)e^{xB(\tau)} + \frac{1}{2}\sigma^2A(\tau)B(\tau)^2e^{xB(\tau)} - xA(\tau)e^{xB(\tau)}.
		\]
		We separate the $A$'s and $B$'s:
		\[
		\frac{A'(\tau)}{A(\tau)} = a(\overline{r} - x)B(\tau) + \frac{1}{2}\sigma^2B(\tau)^2 - B'(\tau) - x.
		\]
		Since we're insisting that $A$ does not depend on $x$, the right-hand side cannot depend on $x$. Consequently, we have
		\[
		a(\overline{r}-x)B - x = 0\iff B = -\frac{1}{a}.
		\]
		This gives
		\[
		\frac{A'(\tau)}{A(\tau)} = \frac{\sigma^2}{2a^2} - \overline{r} \implies A(\tau) = Ce^{(\frac{\sigma^2}{2a^2}-\overline{r})\tau}\implies P(t, x) = C\exp\left[\left(\frac{\sigma^2}{2a^2}-\overline{r}\right)\tau - \frac{x}{a} \right],
		\]
		for some constant $C$. Plugging in the final condition $P(T, x) = 1$ gives $C = e^{x/a}$, and hence
		\[
		P(t, x) = e^{(\frac{\sigma^2}{2a^2}-\overline{r})(T-t)}
		\]
		solves the boundary-value problem. 
	\end{solution}
\end{enumerate}


\noindent\textbf{Problem 3. }
Let $v$ be a continuous scalar valued process satisfying
\[
0\leq v(t)\leq \alpha(t)+\beta\int_0^tv(s)\ ds;\ 0\leq t\leq T,
\]
with $\beta\geq 0$ and $\alpha$ integrable. Show that
\[
v(t)\leq \alpha(t)+\beta\int_0^t\alpha(s)e^{\beta(t-s)}ds,\ 0\leq t\leq T.
\]
Can you relax the assumption about continuity?
\begin{solution}
	Define the function
	\begin{equation}\label{eff}
	F(s) = e^{-\beta s}\cdot \beta\int_0^sv(u)\ du.
	\end{equation}
	We differentiate:
	\[
	F'(s) = \beta e^{-\beta s}\left(v(s)-\beta\int_0^sv(u)\ du\right) \leq \beta\alpha(s)e^{-\beta s}.
	\]
	Integrating from 0 to $t$ gives
	\[
	F(t) \leq \beta\int_0^t\alpha(s)e^{-\beta s}\ ds.
	\]
	Now we have from (\ref{eff}) and the above 
	\[
	\beta \int_0^tv(s)\ ds = e^{\beta t}F(t)\leq \beta\int_0^t\alpha(s)e^{\beta(t-s)}ds.
	\]
	Finally, we know $\beta\int_0^tv(s)\ ds \geq v(t)-\alpha(t)$, so the desired inequality follows.
\end{solution}

\noindent\textbf{Problem 4. }
Let $B_t = B^{(1)}_t + iB^{(2)}_t$ be a complex Brownian motion.
\begin{enumerate}[(a)]
	\item Let $F(z) = u(z)+iv(z)$ be analytic and define
	\[
	Z_t = F(B_t).
	\]
	Prove that
	\[
	dZ_t = F'(B_t)\ dB_t,
	\]
	where $F'$ is the complex derivative of $F$.
	\begin{proof}
		We assume the component Brownian motions are independent. By It\^o we have
		\[
		dB_t = dB^{(1)}_t + idB^{(2)}_t.
		\]
		Write $Z_t$ in terms of the component functions of $F$:
		\[
		Z_t = u(B^{(1)}_t, B^{(2)}_t) + iv(B^{(1)}_t, B^{(2)}_t).
		\]
		By It\^o's lemma we have (suppressing the dependence on $B^{(1)}$ and $B^{(2)}$)
		\begin{align*}
			dZ_t = (u_x+iv_x)dB^{(1)}_t + (u_y+iv_y)dB^{(2)}_t + \frac{1}{2}[(u_{xx}+iv_{xx})dt + (u_{yy}+iv_{yy})dt].
		\end{align*}
		Now the components of an analytic function are harmonic, so the bracketed term vanishes. Applying the Cauchy-Riemann equations gives
		\[
		dZ_t = (u_x+iv_x)dB_t = F'(z)dB_t.
		\]
	\end{proof}

	\item Solve the complex SDE
	\[
	dZ_t = \alpha Z_tdB_t,
	\]
	where $\alpha$ is a constant.
	\begin{solution}
		Pretending that this is a real ODE, we guess that the solution will be exponential. Indeed, by part (a) we have
		\[
		d(e^{\alpha B_t}) = \alpha e^{\alpha B_t}dB_t.
		\]
		Thus, $Z_t = Z_0 + e^{\alpha B_t}$ solves the SDE.
	\end{solution}
\end{enumerate}

\noindent\textbf{Problem 5. }
Consider the SDE
\begin{equation}\label{5_sde}
dX_t = f(t, X_t)\ dt + c(t)X_t\ dB_t,\quad X_0 = x
\end{equation}
where $f: \reals\times\reals\to \reals$ and $c:\reals\to \reals$ are given continuous and deterministic functions.
\begin{enumerate}[(a)]
	\item Define the integrating factor
	\[
	F_t = F_t(\omega) = \exp\left(-\int_0^tc(s)\ dB_s + \frac{1}{2}\int_0^tc^2(s)\ ds\right).
	\]
	Show that (\ref{5_sde}) can be written
	\[
	d(F_tX_t) = F_t\cdot f(t, X_t)\ dt.
	\]
	\begin{proof}
		If we define the process $Y_t$ by
		\[
		dY_t = \frac{1}{2}c^2(t)\ dt - c(t)\ dB_t,
		\]
		then by It\^o's lemma we have
		\begin{align*}
		dF_t &= F_t\ dY_t + \frac{1}{2}F_t(dY_t)^2\\
		&= F_tc_t^2\ dt - F_tc_t\ dB_t.
		\end{align*}
		Again by It\^o we have
		\begin{align*}
			d(F_tX_t) &= F_t\ dX_t + X_t\ dF_t + d\langle F_t, X_t\rangle\\
			&= F_t(f(t, X_t)\ dt + c_tX_t\ dB_t) + X_t(F_tc_t^2\ dt - F_tc_t\ dB_t) - c_t^2F_tX_t\ dt\\
			&= F_t\cdot f(t, X_t)\ dt.
		\end{align*}
	\end{proof}

	\item Now define
	\[
	Y_t(\omega) = F_t(\omega)X_t(\omega)
	\]
	so that
	\[
	X_t = F_t^{-1}Y_t.
	\]
	Deduce that (\ref{5_sde}) gets the form
	\[
	\frac{dY_t(\omega)}{dt} = F_t(\omega)\cdot f(t, F_t^{-1}(\omega)Y_t(\omega)),\quad Y_0 = x.
	\]
	\begin{proof}
		By part (a) we have
		\[
		\frac{dY_t(\omega)}{dt} = \frac{d(F_t(\omega)X_t(\omega))}{dt} = F_t(\omega)\cdot f(t, X_t) = F_t(\omega)\cdot f(t, F_t^{-1}(\omega)Y_t(\omega)).
		\]
		Since $F_0 = 1$, we have $Y_0 = x$.
	\end{proof}

	\item Apply this method to solve the SDE
	\[
	dX_t = \frac{1}{X_t}\ dt + \alpha X_t\ dB_t,\quad X_0 = x > 0
	\]
	where $\alpha$ is constant.
	\begin{solution}
		As per part (b), define the integrating factor
		\[
		F_t(\omega) = \exp\left(\frac{1}{2}\alpha^2t - \alpha B_t \right).
		\]
		Letting $Y_t(\omega) = F_t^{-1}(\omega)$, again by part (b) we have
		\[
		\frac{dY_t(\omega)}{dt} = F_t(\omega)\cdot\frac{1}{F_t^{-1}(\omega)Y_t(\omega)} = \frac{F_t(\omega)^2}{Y_t(\omega)}.
		\]
		This is a separable ODE. After separating and integrating we obtain
		\[
		Y_t^2(\omega)-x^2 = 2\int_0^t\exp(\alpha^2s - 2\alpha B_s)\ ds,
		\]
		which gives
		\[
		X_t(\omega) = \exp\left(\frac{1}{2}\alpha^2t - \alpha B_t\right)\left[x^2 + 2\int_0^t\exp(\alpha^2s-2\alpha B_s)\ ds\right]^{1/2}.
		\]
	\end{solution}

	\item Apply the method to study the solutions of the SDE
	\[
	dX_t = X_t^\gamma\ dt + \alpha X_t\ dB_t;\quad X_0 = x>0,
	\]
	where $\alpha$ and $\gamma$ are constants. For what values of $\gamma$ do we get explosion?
	\begin{solution}
		We use the same integrating factor from part (c). The pathwise ODE we get is
		\[
		Y_t^{-\gamma}(\omega)\ dY_t(\omega) = F_t(\omega)^2\ dt.
		\]
		The left-hand side is integrable if and only if $\gamma < 1$.
	\end{solution}
\end{enumerate}

\noindent\textbf{Problem 6. }
Explain the terms
\begin{enumerate}[(a)]
	\item Martingale
	\begin{solution}
		A martingale is an integrable stochastic process $(M_t)_{t\geq 0}$ adapted to a filtration $(\mcal{F}_t)_{t\geq 0}$ whose expected value at a future time is its current value: $\E[X_t\ |\ \mcal{F}_s] = X_s$ a.s. for $s\leq t$.
	\end{solution}

	\item It\^o process
	\begin{solution}
		An It\^o process, $X_t$, is a stochastic process that can be written as a sum of stochastic integrals: one with respect to time and the other with respect to Brownian motion, i.e.
		\[
		X_t(\omega) = X_0(\omega) + \int_0^t\mu_t(\omega)\ dt + \int_0^t\sigma_t(\omega)\ dB_t,
		\]
		where $\mu$ and $\sigma$ are integrable (in $t$) processes for all $\omega$.
	\end{solution}

	\item Stopping time
	\begin{solution}
		A random variable $T(\omega)$ is a stopping time with respect to the filtration $(\mcal{F}_t)_{t\geq 0}$ if $\{T(\omega)\leq t\}\in \mcal{F}_t$. The intuition is that one should know whether or not $T\leq t$ based on the ``information'' $\mcal{F}_t$.
	\end{solution}

	\item Quadratic variation
	\begin{solution}
		The quadratic variation of a process $(X_t)_{t\geq 0}$, in a sense, measures its ``roughness.'' It is given by
		\[
		\langle X\rangle_t = \lim_{\|P\|\to 0}\sum_{k=1}^n(X_{t_k}-X_{t_{k-1}})^2,
		\]
		where $P$ ranges over all partitions of $[0,t]$ and the limit is in probability. All finite variation processes have zero quadratic variation, while Brownian motion has infinite total variation and $\langle B\rangle_t = t$.
	\end{solution}

	\item Kolmogorov backward equation
	\begin{solution}
		Loosely speaking, for a diffusion process $X_t$, the Kolmogorov backward equation is a partial differential equation whose solution, $P(x, t)$, is the probability density function:
		\[
		\int_BP(x, t)\ dx = \Pr[X_t\in B\ |\ X_T = x],
		\]
		where $t\leq T$ and $x$ is fixed. It can be derived from It\^o's lemma.
	\end{solution}
\end{enumerate}

\noindent\textbf{Questionnaire}
The top 3 topics I'd like to see are
\begin{enumerate}
	\item Graph-based models
	\item Large deviations
	\item It\^o calculus for Hilbert space-valued processes
\end{enumerate}
Thank you for asking!

\end{document}