\documentclass[12pt]{article}  
\pagestyle{empty}

\usepackage{graphics}
\usepackage{amsmath,amssymb,amsthm, multicol}
\usepackage[pdftex]{graphicx}
\usepackage{epsf}

\newcommand{\naturals}{\mathbb{N}}
\newcommand{\integers}{\mathbb{Z}}
\newcommand{\complex}{\mathbb{C}}
\newcommand{\reals}{\mathbb{R}}
\newcommand{\mcal}[1]{\mathcal{#1}}
\newcommand{\rationals}{\mathbb{Q}}
\newcommand{\Aut}{\text{Aut}}
\newcommand{\Lp}[2]{\left\|{#1}\right\|_{L^{#2}}}
\newcommand{\tr}{\text{tr}}
\newcommand{\field}{\mathbb{F}}

\addtolength{\oddsidemargin}{-.75in}
\addtolength{\evensidemargin}{-.75in}
\addtolength{\textwidth}{1.5in}
\addtolength{\topmargin}{-1in}
\addtolength{\textheight}{2.25in}

\begin{document}
\begin{flushleft} 
\centerline{\LARGE{Quiz 1}} 
\vspace{5 mm}
{Student ID Number:}\hfill  
{Name \rule {2 in}{0.01in}}\\
Math 180B
\\
{Please justify all your answers}\hfill {April 9, 2020}
\\
{Please also write your full name on the back} 

\medskip
\end{flushleft}

\begin{enumerate}
	\item If $r$ is a primitive root of the odd prime $p$, show that
	\[
	ind_r\ (-1) = \frac{1}{2}(p-1).
	\]
	\vfill

	\item For any integer $n$ let $\sigma(n)$ be the sum of its divisors. Show that for any positive integer $n$,
	\[
	\sum_{d\mid n}\frac{1}{d} = \frac{\sigma(n)}{n}.
	\]
	\vfill
\end{enumerate}

\end{document}