\documentclass[12pt]{article}  
%%Read the manual for other options. 

\pagestyle{empty} %%Eliminates page numbers
%%\input rmb_macros
%%Collect your favorite macros in a 
%%separate file

%\input amssym.def
%\input amssym
%\input mssymb
%%Defines additional symbols



\usepackage{graphics}
\usepackage{amsmath,amssymb,amsthm, multicol}
\usepackage[pdftex]{graphicx}
\usepackage{epsf}
%%Use to include pictures. 

%\newcommand{\comment}[1]{}
%\newcommand{\sobolev}[2]{W^{#1,#2}}
%\newcommand{\sobolev}[2]{L^#2_#1}
\newcommand{\integers}{\mathbb{Z}}
%%Some examples of macros or new commands.

\addtolength{\oddsidemargin}{-.75in}
\addtolength{\evensidemargin}{-.75in}
\addtolength{\textwidth}{1.5in}
\addtolength{\topmargin}{-1in}
\addtolength{\textheight}{2.25in}
%%Set margins, defaults are ok. 

\begin{document}
\begin{flushleft} 
%%Paragraphs will not be indented in this 
%%environment
\centerline{\LARGE{Quiz 2}} 
\vspace{5 mm}
{Student ID Number:}\hfill  
%%\hfill forces following text 
%%to right margin
{Name \rule {2 in}{0.01in}}\\
Math 173A, 3PM
\\
%%gives a line of length 2in and 
%%thickness 0.01in
{Please justify all your answers}\hfill {October 10, 2019}
\\
{Please also write your full name on the back} 

\medskip
\end{flushleft}

\begin{enumerate}
	\item Fill in the blank or answer with ``True'' or ``False''.
	\begin{enumerate}
		\item True or False? Let $n$ be an integer. Then every nonzero element $a$ in $\integers/n\integers$ has a multiplicative inverse.
		\item The highest power of a prime $p$ dividing an integer $n$ is called the \rule{2.5cm}{.15mm} of $p$ in $n$.
		\item True or False? $(\integers/n\integers)^\times$ contains $\phi(n)$ elements, where $\phi$ is the Euler totient function.
	\end{enumerate}
	\vfill

	\item \begin{enumerate}
		\item Solve $7x \equiv 5 \pmod{11}$.
		\vfill
		\item Let $\{p_1, p_2, \ldots, p_r\}$ be a set of prime numbers, and let $N = p_1p_2\cdots p_r + 1$. Prove that $N$ is divisible by some prime not in the original set. Use this fact to deduce that there must be infinitely many prime numbers.
		\vfill
	\end{enumerate}
\end{enumerate}

%\vfill will divide page evenly
%use \begin{enumerate} environment for ordered lists
\end{document}