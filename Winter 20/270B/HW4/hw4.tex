\documentclass[11pt,letterpaper]{report}
\usepackage{amssymb,amsfonts,color,graphicx,amsmath,enumerate}
\usepackage{tikz}
\usepackage{amsthm}
\usepackage{bbm}
\usepackage{hyperref}

\newcommand{\naturals}{\mathbb{N}}
\newcommand{\integers}{\mathbb{Z}}
\newcommand{\complex}{\mathbb{C}}
\newcommand{\reals}{\mathbb{R}}
\newcommand{\mcal}[1]{\mathcal{#1}}
\newcommand{\rationals}{\mathbb{Q}}
\newcommand{\Lp}[2]{\left\|{#1}\right\|_{L^{#2}}}
\newcommand{\field}{\mathbb{F}}
\newcommand{\affine}{\mathbb{A}}
\newcommand{\E}{\mathbb{E}}
\newcommand{\Prob}{\mathbb{P}}
\newcommand{\Var}{\text{Var}}
\newcommand{\ind}{\mathbbm{1}}
\newcommand{\Cov}{\text{Cov}}

\newenvironment{solution}
{\begin{proof}[Solution]}
{\end{proof}}

\voffset=-3cm
\hoffset=-2.25cm
\textheight=24cm
\textwidth=17.25cm
\addtolength{\jot}{8pt}
\linespread{1.3}

\begin{document}
\noindent{\em Liam Hardiman\hfill{March 18, 2020} }
\begin{center}
{\bf \Large 270B - Homework 4}
\vspace{0.2cm}
\hrule
\end{center}

\noindent\textbf{Problem 1. }
Let $(X_n)$ be an irreducible recurrent Markov chain with doubly-infinite transition matrix $P$. Let $\psi: \naturals\to \naturals$ be a bounded function satisfying
\[
\sum_{j=1}^\infty P_{ij}\psi(j) = \psi(i)\quad\text{for all }i\in \naturals.
\]
Show that $\psi$ is a constant function.
\begin{proof}
	First we claim that $\psi(X_n)$ is a martingale. Let $\mcal{F}_n$ be the filtration generated by $X_1, \ldots, X_n$. We then have by hypothesis
	\[
	\E[\psi(X_{n+1})|\mcal{F}_n] = \E\left[\sum_jP_{X_{n+1}, j}\psi(j)\ \big|\ \mcal{F}_n\right] = \sum_{j}P_{X_n, j}\psi(j) = \psi(X_n).
	\]
\end{proof}

\noindent\textbf{Problem 2. }
Let $S$ and $T$ be stopping times with respect to a filtration $(\mcal{F}_n)$. Denote by $(\mcal{F}_T)$ the collection of events $F$ such that $F\cap \{T\leq n\} \in \mcal{F}_n$ for all $n$.
\begin{enumerate}[(a)]
	\item Show that $\mcal{F}_T$ is a $\sigma$-algebra.
	\begin{proof}
		That $\emptyset$ and $\Omega$ are in $\mcal{F}_T$ immediately follows from $T$ being a stopping time. If $F\cap \{T\leq n\}\in \mcal{F}_n$ then
		\[
		F^c\cap \{T\leq n\} = (F\cup \{T > n\})^c\in \mcal{F}_n,
		\]
		since $\mcal{F}_n$ is a $\sigma$-algebra.
	\end{proof}
\end{enumerate}

\end{document}