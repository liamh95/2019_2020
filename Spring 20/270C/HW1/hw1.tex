\documentclass[11pt,letterpaper]{report}
\usepackage{amssymb,amsfonts,color,graphicx,amsmath,enumerate}
\usepackage{amsthm, bbm}

\newcommand{\naturals}{\mathbb{N}}
\newcommand{\integers}{\mathbb{Z}}
\newcommand{\complex}{\mathbb{C}}
\newcommand{\reals}{\mathbb{R}}
\newcommand{\mcal}[1]{\mathcal{#1}}
\newcommand{\rationals}{\mathbb{Q}}
\newcommand{\field}{\mathbb{F}}
\newcommand{\Var}{\text{Var}}
\newcommand{\ind}{\mathbbm{1}}
\newcommand{\Cov}{\text{Cov}}
\newcommand{\subg}[1]{\|{#1}\|_{\psi_2}}
\newcommand{\Lp}[2]{\big\|{#1}\big\|_{L^{#2}}}

\newenvironment{solution}
{\begin{proof}[Solution]}
{\end{proof}}

\voffset=-3cm
\hoffset=-2.25cm
\textheight=24cm
\textwidth=17.25cm
\addtolength{\jot}{8pt}
\linespread{1.3}

\begin{document}
\noindent{\em Liam Hardiman\hfill{April 21, 2020} }
\begin{center}
{\bf \Large 270C - Homework 1}
\vspace{0.2cm}
\hrule
\end{center}

\noindent\textbf{2.2.7}
Let $X_1, \ldots, X_N$ be independent random variables. Assume that $X_i\in [m_i, M_i]$ for every $i$. Then for any $t>0$ we have
\[
\Pr\left[\sum_{i=1}^N(X_i- E[X_i])\geq t \right] \leq \exp\left(-\frac{2t^2}{\sum_{i=1}^N(M_i-m_i)^2} \right).
\]
\begin{proof}
	[Is there an easy argument I'm missing here?] Let $Y_i = X_i - E[X_i]$. Like in all of these Hoeffding-like proofs, we multiply by $\lambda$, exponentiate, and apply Markov to obtain
	\begin{equation}\label{markov}
	\Pr\left[\sum_{i=1}^NY_i \geq t\right]  \leq e^{-\lambda t}\prod_{i=1}^NE\left[e^{\lambda Y_i} \right].
	\end{equation}
	Let's bound that MGF. [I Googled around for this and found Hoeffding's lemma.] Since $Y_i \in [m_i, M_i]$ and $t\geq 0$, the convexity of $t\mapsto e^{\lambda t}$ gives
	\[
	e^{\lambda Y_i}\leq \frac{Y_i-m_i}{M_i-m_i}e^{\lambda M_i} + \frac{M_i-Y_i}{M_i-m_i}e^{\lambda m_i}.
	\]
	Taking the expectation and using the fact that the $Y_i$'s are centered gives
	\[
	Ee^{\lambda Y_i}\leq \frac{M_i}{M_i-m_i}e^{\lambda m_i} - \frac{m_i}{M_i-m_i}e^{\lambda M_i}.
	\]
	Set $\alpha = \frac{M_i}{M_i - m_i}$. Taking the logarithm of the above expression gives
	\begin{align*}
		\log(\alpha e^{\lambda M} + (1-\alpha)e^{\lambda m}) = \lambda m + \log(\alpha + (1-\alpha)e^{\lambda(M-m)}).
	\end{align*}
	Regard this expression as a function $\varphi$ of $u = \lambda(M-m)$. Some (tedious) calculus shows that this function is zero at zero, has zero derivative at zero, and has bounded (say by $K>0$ independent of $M_i$ and $m_i$) second derivative. By Taylor's theorem, there is some $\xi\in (0, u)$ such that
	\[
	\varphi(u) = \varphi(0) + \varphi'(0)u + \frac{\varphi''(\xi)}{2}u^2 \leq \frac{1}{2}K\lambda^2(M_i-m_i)^2.
	\]
	We then have that
	\[
	Ee^{\lambda Y_i}\leq e^{K\lambda^2(M_i-m_i)^2/2}.
	\]
	Substituting this bound into (\ref{markov}) gives
	\[
	\Pr\left[\sum_{i=1}^NY_i \geq t\right]  \leq \exp\left(-\lambda t + \frac{K\lambda^2}{2}\sum_{i=1}^N(M_i-m_i)^2 \right).
	\]
	Choosing $\lambda= \frac{t}{K\sum_{i=1}^N(M_i-m_i)^2}$ minimizes the above expression, which gives the desired conclusion, with some absolute constant $K$ in place of 2. 

\end{proof}










\noindent\textbf{2.2.8}
Imagine we have an algorithm for solving some decision problem. Suppose the algorithm makes a decision at random and returns the correct answer with probability $\frac{1}{2}+\delta$< for some $\delta>0$. To improve the performance, we run the algorithm $N$ times and take the majority vote. Show that for any $\epsilon \in (0, 1)$, the answer is correct with probability at least $1-\epsilon$, as long as
\[
N\geq \frac{1}{2\delta^2}\log\frac{1}{\epsilon}.
\]
\begin{proof}
	If $X_1, X_2, \ldots, X_N$ are the indicators of wrong outputs, then we're wrong exactly when $\sum_{i=1}^NX_i > \frac{1}{2}N$. The expectation of this sum is  $(\frac{1}{2}-\delta)N$, so by the bounded version of Hoeffding proved in exercise 2.2.7 we have
	\begin{align*}
		\Pr\left[\sum_{i=1}^NX_i > \frac{1}{2}N\right] &= \Pr\left[\sum_{i=1}^NX_i - \left(\frac{1}{2}-\delta\right)N> N\delta \right]\\
		&\leq e^{-2(N\delta)^2/N}\\
		&= e^{2N\delta^2}.
	\end{align*}
	This quantity is less than $\epsilon$ when $N>\frac{1}{2\delta^2}\log\frac{1}{\epsilon}$.
\end{proof}










\noindent\textbf{2.2.9}
Suppose we want to estimate the mean $\mu$ of a random variable $X$ (assumed to take values in $[a,b]$ a.s.) from a sample $X_1, \ldots, X_N$ drawn independently from the distribution of $X$. We want an $\epsilon$-accurate estimate.
\begin{enumerate}[(a)]
	\item Show that a sample of size $N = O(\sigma^2/\epsilon^2)$ is sufficient to compute an $\epsilon$-accurate estimate with probability at least 3/4, where $\sigma^2 = \Var[X]$.
	\begin{proof}
		The sample mean $\hat{\mu} = \frac{1}{N}\sum_{i=1}^NX_i$ has mean $\mu$. By Hoeffding we have
		\[
		\Pr[|\hat{\mu}-\mu|>\epsilon] \leq 2e^{-2N\epsilon^2/(b-a)^2}.
		\]
		Setting this less than $1/4$ and solving for $N$ gives
		\[
		N\geq \frac{(b-a)^2\log 8}{\epsilon^2} \geq \frac{4\log 8\sigma^2}{\epsilon^2} = O(\sigma^2/\epsilon^2).
		\]

	\end{proof}

	\item Show that a sample of size $N = O(\log(1/\delta)\sigma^2/\epsilon^2)$ is sufficient to compute an $\epsilon$-accurate estimate with probability at least $1-\delta$.
	\begin{proof}
		Like in part (a) we have
		\[
		\Pr[|\hat{mu}-\mu|>\epsilon]\leq 2e^{-2N\epsilon^2/(b-a)^2}.
		\]
		Setting this less than $\delta$ and solving for $N$ gives
		\[
		N > \frac{1}{\epsilon^2}(b-a)^2\log\frac{2}{\delta} \geq \frac{4\sigma^2}{\epsilon^2}\log\frac{2}{\delta} = O\left(\frac{\sigma^2}{\epsilon^2}\log\frac{2}{\delta}\right).
		\]
	\end{proof}
\end{enumerate}










\noindent\textbf{2.2.10}
Let $X_1, \ldots, X_N$ be non-negative independent random variables with continuous distributions. Assume that the densities of $X_i$ are uniformly bounded by 1.
\begin{enumerate}[(a)]
	\item Show that the MGF of $X_i$ satisfies
	\[
	E\exp(-tX_i)\leq \frac{1}{t}\quad\text{for all }t>0.
	\]
	\begin{proof}
		Let $f_i$ be the density of $X_i$. Since $f_i(x)\leq 1$ for all $x$, we have
		\[
		E\exp(-tX_i) = \int_0^\infty e^{-tx}f_i(x)\ dx \leq \int_0^\infty e^{-tx}\ dx = \frac{1}{t}.
		\]
	\end{proof}

	\item Deduce that for any $\epsilon>0$ we have
	\[
	\Pr\left[\sum_{i=1}^NX_i\leq \epsilon N\right]\leq (e\epsilon)^N.
	\]
	\begin{proof}
		We have
		\begin{align*}
			\Pr\left[\sum_{i=1}^NX_i\leq \epsilon N\right] &= \Pr\left[-t\sum_{i=1}^N(X_i/\epsilon)\geq -tN\right]\\
			&= \Pr\left[\exp\left(-t\sum_{i=1}^N(X_i/\epsilon)\right)\geq e^{-tN}\right].
		\end{align*}
		Applying Markov's inequality gives
		\begin{align*}
			\Pr\left[\sum_{i=1}^NX_i\leq \epsilon N\right] &\leq e^{tN}E\left[\exp\left(-t\sum_{i=1}^N(X_i/\epsilon)\right)\right]\\
			&= e^{tN}\prod_{i=1}^NE\exp(-(t/\epsilon)X_i)\\
			&\leq \frac{e^{tN}\epsilon^N}{t^N}.
		\end{align*}
		The right-hand side attains a minimum at $t=1$, which finally gives
		\[
		\Pr\left[\sum_{i=1}^NX_i\leq \epsilon N\right]\leq (e\epsilon)^N.
		\]
	\end{proof}
\end{enumerate}










\noindent\textbf{2.3.2}
Let $X_i$ be independent Bernoulli random variables with parameters $p_i$. Consider the sum $S_N = \sum_{i=1}^NX_i$ and denote its mean by $\mu = E[S_N]$. Then, for any $t<\mu$, we have
\[
\Pr[S_N\leq t]\leq e^{-\mu}\left(\frac{e\mu}{t}\right)^t.
\]
\begin{proof}
	We compute.
	\begin{align*}
		\Pr[S_n \leq t] &= \Pr[-S_n \geq -t]\\
		&= \Pr[\exp(-\lambda S_n) \geq e^{-\lambda t}]\\
		&\leq e^{\lambda t}\prod_{i=1}^NE[e^{-\lambda X_i}]\\
		&= e^{\lambda t}\prod_{i=1}^N[e^{-\lambda}p_i + (1-p_i)]\\
		&\leq e^{\lambda t}\exp\left[(e^{-\lambda}-1)\sum_{i=1}^Np_i \right]\\
		&= e^{-\lambda t}\exp\left[\mu\left(\frac{t}{\mu}-1\right)\right].
	\end{align*}
	This holds for all $\lambda \geq 0$. Since $t<\mu$, setting $\lambda = \log(\mu/t)$ gives
	\begin{align*}
		\Pr[S_n \leq t] &\leq \left(\frac{\mu}{t}\right)^te^{t-\mu} = e^{-\mu}\left(\frac{e\mu}{t}\right)^t.
	\end{align*}
\end{proof}










\noindent\textbf{2.3.5}
Show that in the setting of the previous problem, for $\delta \in (0, 1]$ we have
\[
\Pr[|S_N-\mu|\geq \delta \mu] \leq 2e^{-c\mu\delta^2},
\]
where $c>0$ is an absolute constant.
\begin{proof}
	By the previous exercise we have
	\begin{align*}
		\Pr[|S_N-\mu| \geq \delta \mu] &= \Pr[S_N\geq (1+\delta)\mu] + \Pr[S_N\leq (1-\delta)\mu]\\
		&\leq e^{-\mu}\left(\frac{e\mu}{(1+\delta)\mu}\right)^{(1+\delta)\mu} + e^{-\mu}\left(\frac{e\mu}{(1-\delta)\mu}\right)^{(1-\delta)\mu}\\
		&= \left(\frac{e^\delta}{(1+\delta)^{1+\delta}}\right)^\mu + \left(\frac{e^{-\delta}}{(1-\delta)^{1-\delta}}\right)^\mu
	\end{align*}
	First, we claim that
	\begin{equation}\label{2.3.5: plus}
	\left(\frac{e^\delta}{(1+\delta)^{1+\delta}}\right)^\mu\leq e^{-\delta^2\mu/3}.
	\end{equation}
	To see this, we take the logarithm of the left-hand side:
	\[
	\mu[\delta - (1+\delta)\log(1+\delta)].
	\]
	Now an elementary calculus argument shows that $\log(1+\delta)\geq \frac{2\delta}{2+\delta}$ for $\delta\in (0, 1]$ (the difference is zero for $\delta = 0$ and its derivative is nonnegative). From this we deduce
	\[
	\mu[\delta-(1+\delta)\log(1+\delta)] \leq \mu\left[\delta-\frac{2\delta(1+\delta)}{2+\delta}\right] = -\frac{\mu\delta^2}{2+\delta},
	\]
	which establishes (\ref{2.3.5: plus}) after considering $\delta\in (0,1]$. Next, we claim that
	\begin{equation}\label{2.3.5: minus}
	\left(\frac{e^{-\delta}}{(1-\delta)^{1-\delta}}\right)^\mu \leq e^{-\delta^2\mu/2}.
	\end{equation}
	Like in the preceding inequality, we take the logarithm of the left-hand side:
	\[
	\mu[-\delta-(1-\delta)\log(1-\delta)].
	\]
	By looking at the Taylor expansion, we see that $(1-\delta)\log(1-\delta) \geq -\delta+\frac{\delta^2}{2}$. Substituting this into the above quantity establishes (\ref{2.3.5: minus}). Inequalities (\ref{2.3.5: plus}) and (\ref{2.3.5: minus}) establish the desired claim.
\end{proof}










\noindent\textbf{2.3.6}
Let $X\sim Pois(\lambda)$. Show that for $t\in (0, \lambda]$ we have
\[
\Pr[|X-\lambda|\geq t] \leq 2\exp\left(-\frac{ct^2}{\lambda}\right).
\]
\begin{proof}
	Let $X_{N, i}$ be a sequence of independent Bernoulli random variables with parameters $p_{N,i}$. Let $S_N = \sum_{i=1}^NX_{N,i}$. Furthermore, suppose that as $N\to \infty$
	\[
	\max_{i\leq N}p_{N,i} \to 0,\quad\text{and}\quad E[S_N]\to \lambda.
	\]
	Then by the Poisson limit theorem, $S_N\to X$ in distribution. Setting $\mu_N = E[S_N]$, we have
	\begin{align*}
		\Pr[|X-\lambda|\geq t]&= \lim_{N\to \infty}\Pr[|S_N-\mu_N|\geq t]\\
		&\leq \lim_{N\to \infty}2e^{-ct^2/\mu_N}\\
		&= 2e^{-ct^2/\lambda}.
	\end{align*}
\end{proof}










\noindent\textbf{2.4.2}
Consider a random graph $G\sim G(n, p)$ with expected degree $d = O(\log n)$. Show that with high probability, all vertices have degrees $O(\log n)$.
\begin{proof}
	Fix $c>0$ to be determined later and fix $\epsilon>0$. Chernoff tells us that for any vertex $i$ we have
	\[
	\Pr[d_i \geq cd] \leq e^{-d}\left(\frac{ed}{cd}\right)^{cd}= \left(\frac{e^{c-1}}{c^c}\right)^d.
	\]
	A union bound over all vertices gives
	\[
	\Pr[d_i\geq c_d\text{ for some }i\leq n] \leq n\left(\frac{e^{c-1}}{c^c}\right).
	\]
	We take the logarithm and do some algebra.
	\begin{align*}
		n\left(\frac{e^{c-1}}{c^c}\right)^d<\epsilon &\iff \log n + d\log\frac{e^{c-1}}{c^c}<\log \epsilon\\
		&\iff d< \frac{\log(n/\epsilon)}{\log(c^c/e^{c-1})}.
	\end{align*}
	For $c$ sufficiently large (say 10), we have that $d = O(\log n)$. We have then shown that when $d = O(\log n)$, the probability that any vertex has degree larger than $cd$ is less than $\epsilon$. Consequently, with probability $1-\epsilon$, each degree is $O(d) = O(\log n)$.
\end{proof}










\noindent\textbf{2.4.3}
Consider a random graph $G\sim G(n, p)$ with expected degree $d = O(1)$. Show that with high probability, say 0.9, all vertices of $G$ have degrees $O(\frac{\log n}{\log\log n})$.
\begin{proof}
	Fix $c>0$ to be determined later and fix $\epsilon>0$. For ease of notation, let $f(n) = \frac{\log n}{\log\log n}$. Since $d = O(1)$, we can say $d_i \leq M$ for some fixed large $M$ and for all $i$. Like in the previous exercise, Chernoff and a union bound give us
	\[
	\Pr[d_i >cf(n)\text{ for some }i\leq n] \leq ne^{-d}\left(\frac{ed}{cf(n)}\right)^{cf(n)}\leq n\left(\frac{eM}{cf(n)}\right)^{cf(n)}.
	\]
	Setting this less than $\epsilon$ and taking logarithms gives
	\begin{align*}
		n\left(\frac{eM}{cf(n)}\right)^{cf(n)} <\epsilon &\iff \log n + cf(n)[1+\log M - \log(cf(n))]<\log \epsilon\\
		&\iff cf(n)[\log(cf(n))-1-\log M] > \log\frac{n}{\epsilon}.
	\end{align*}
	Now the leading term of the left-hand side of the final inequality is
	\[
	cf(n)\log(cf(n)) = c\frac{\log n}{\log\log n}\left(\log(c\log n) - \log\log\log n \right),
	\]
	which is $\Omega(\log n)$ for an appropriate choice of $c$. For such $c$, we have that $\Pr[d_i > cf(n)\text{ for some }i\leq n] <\epsilon$, so with probability at least $1-\epsilon$, $d_i = O(\frac{\log n}{\log\log n})$.
\end{proof}










\noindent\textbf{2.4.4}
Consider a random graph $G\sim G(n,p)$ with expected degree $d = o(\log n)$. Show that with high probability, say 0.9, $G$ has a vertex with degree $10d$.
\begin{proof}
	Since $d = o(\log n)$, we must have $p = o(\frac{\log n}{n})$. If the degrees $d_i$ were independent, we could write
	\[
	\Pr[d_i\neq 10d\text{ for all }i\leq n] = \prod_{i=1}^n\Pr[d_i \neq 10d],
	\]
	and then use Chernoff to bound the right-hand side. Since life isn't so simple, we have to be a bit more clever.\\

	Maybe we can take an arbitrary subset and consider the outdegrees of its vertices: these will be independent. 
\end{proof}










\noindent\textbf{2.5.5}
\begin{enumerate}[(a)]
	\item Show that if $X\sim \mcal{N}(0, 1)$, the function $\lambda\mapsto E\exp(\lambda^2X^2)$ is only finite in some bounded neighborhood of zero.
	\begin{proof}
		Let's compute that expectation.
		\[
		E\exp(\lambda^2X^2) =\frac{1}{\sqrt{2\pi}}\int_\reals e^{(\lambda^2-1/2)x^2}\ dx.
		\]
		This integral is finite if and only if $\lambda\in(-1/\sqrt{2}, 1/\sqrt{2})$.
	\end{proof}

	\item Suppose that some random variable $X$ satisfies $E\exp(\lambda^2X^2)\leq \exp(K\lambda^2)$ for all $\lambda\in \reals$ and some constant $K$. Show that $X$ is a bounded random variable, i.e. $\|X\|_\infty<\infty$.
	\begin{proof}
		Suppose $X$ were not bounded. Then $\Pr[|X|\geq M]>0$ for every $M$. We can bound that expectation below:
		\[
		E\exp(\lambda^2X^2) \geq E[\exp(\lambda^2X^2)\cdot \ind_{\{X\geq M\}}] \geq \exp(\lambda^2M^2)\cdot \Pr[|X|\geq M].
		\]

	\end{proof}
\end{enumerate}










\noindent\textbf{2.5.7}
Define the norm $\subg{\cdot}$ on the space of sub-gaussian random variables:
\[
\subg{X} = \inf\{t>0: E\exp(X^2/t^2)\leq 2\}.
\]
Show that $\subg{\cdot}$ is indeed a norm.
\begin{proof}
	Clearly $\subg{0} = 0$. Conversely, if $\|X\|_\infty > 0$, then for some $\epsilon>0$ and $\delta>0$, $\Pr[|X|>\epsilon] > \delta$. We then have
	\[
	E\exp(X^2/t^2) \geq \Pr[|X|>\epsilon]\exp(\epsilon^2/t^2)\geq \delta \exp(\epsilon^2/t^2).
	\]
	This quantity can be made arbitrarily large for $t$ arbitrarily small, so it cannot be less than 2 $t$ arbitrarily small. We conclude that $\subg{X} = 0$ if and only if $X = 0$.\\

	Let $c$ be any real number and suppose $\subg{X} = r$. Then
	\[
	\subg{cX} = \inf\{t>0: E\exp(c^2X^2/t^2)\leq 2\} = \inf\{t>0: E\exp(X^2/(t/c)^2)\leq 2\} = |c|r,
	\]
	so $\subg{\cdot}$ is homogeneous.\\

	For ease of notation, let $f(x) = e^{x^2}$. It suffices to show that for any two sub-gaussian random variables $X$ and $Y$,
	\begin{equation}\label{2.5.7: tri}
		Ef\left( \frac{|X+Y|}{\subg{X}+\subg{Y}}\right)\leq 2.
	\end{equation}
	Since $f$ is convex and increasing, we have for any $s$ and $t$,
	\[
	f\left(\frac{|X+Y|}{s+t}\right)\leq f\left(\frac{|X|+|Y|}{s+t}\right) \leq \frac{s}{s+t}f\left(\frac{|X|}{s}\right) + \frac{t}{s+t}f\left(\frac{|Y|}{t}\right).
	\]
	Taking the expectation of both sides and setting $s = \subg{X}$ and $t= \subg{Y}$ gives
	\[
	Ef\left(\frac{|X+Y|}{\subg{X}+\subg{Y}}\right) \leq \frac{\subg{X}}{\subg{X}+\subg{Y}}\cdot 2 + \frac{\subg{Y}}{\subg{X}+\subg{Y}}\cdot 2 = 2.
	\]
	Thus, $\subg{\cdot}$ is a norm.
\end{proof}










\noindent\textbf{2.5.10}
Let $X_1, X_2, \ldots$ be a sequence of sub-gaussian random variables, which are not necessarily independent. Show that
\[
E\max_i \frac{|X_i|}{\sqrt{1+\log i}}\leq CK,
\]
where $K = \max_i\subg{X_i}$. Deduce that for every $N\geq 2$ we have
\[
E\max_{i\leq N}|X_i|\leq CK\sqrt{\log N}.
\]
\begin{proof}
	We use the fact that for a nonnegative random variable $Y$, $E[Y] = \int_0^\infty \Pr[Y\geq t]\ dt$. We split this integral
	\begin{align*}
		E\max_{i\leq N} \frac{|X_i|}{\sqrt{1+\log i}} &= \int_0^\infty\Pr\left[\max_{i\leq N}\frac{|X_i|}{\sqrt{1+\log i}} \geq t \right]\ dt\\
		&= \int_0^\alpha\Pr\left[\max_{i\leq N}\frac{|X_i|}{\sqrt{1+\log i}} \geq t \right]\ dt + \int_\alpha^\infty\Pr\left[\max_{i\leq N}\frac{|X_i|}{\sqrt{1+\log i}} \geq t \right]\ dt.
	\end{align*}
	The first integral is clearly bounded by $\alpha$. For the second, we use a union bound.
	\begin{align*}
		E\max_{i\leq N} \frac{|X_i|}{\sqrt{1+\log i}} &\leq \alpha + \int_\alpha^\infty \sum_{i=1}^N\Pr\left[\frac{|X_i|}{\sqrt{1+\log i}} \geq t \right]\ dt\\
		&= \alpha + \sum_{i=1}^N\int_\alpha^\infty \Pr[|X_i|\geq t\sqrt{1+\log i}]\ dt.
	\end{align*}
	Since the $X_i$'s are sub-Gaussian, we can bound these tail probabilities.
	\begin{align*}
		E\max_{i\leq N} \frac{|X_i|}{\sqrt{1+\log i}} &\leq\alpha + \sum_{i=1}^N\int_\alpha^\infty 2\exp\left[-t^2(1+\log i)/K^2 \right]\ dt\\
		&= \alpha + \sum_{i=1}^N\int_\alpha^\infty 2\exp\left[-t^2/K^2 \right]i^{-t^2}\ dt\\
		&\leq  \alpha + \sum_{i=1}^N\int_\alpha^\infty 2\exp\left[-t^2/K^2 \right]i^{-\alpha^2}\ dt\\
		&\leq \alpha + CK\sum_{i=1}^\infty i^{-\alpha^2}.
	\end{align*}
	Setting $\alpha= 2$ makes the sum converge and establishes the first claim.\\
\end{proof}










\noindent\textbf{2.5.11}
Show that the bound exercise 2.5.10 is sharp. let $X_1, \ldots, X_N$ be independent $\mcal{N}(0, 1)$ random variables. Prove that
\[
E\max_{i\leq N}X_i \geq c\sqrt{\log N}.
\]
\begin{proof}
	We have
	\[
	E[\max_i X_i] = E[\max_i X_i\cdot \ind_{\max_i X_i \geq 0}] + E[\max_i X_i\cdot \ind_{\max_i X_i <0}].
	\]
	The probability that $\max_i X_i$ is negative is $2^{-N}$ and $\max_i X_i$ is monotone increasing in $N$, so by dominated convergence, the second term is $o(1)$. As for the first term, by independence we have
	\begin{align*}
	E[\max_i X_i\cdot \ind_{\max_i X_i \geq 0}]&= \int_0^\infty \Pr[X_i\cdot \ind_{\max_i X_i \geq 0} \geq t]\ dt\\
	&= 	\int_0^\infty 1 - \Pr[X_i <t,\ i\leq N]\ dt\\
	&= \int_0^\infty 1- \Phi(t)^N\ dt\\
	&\geq \int_0^{\sqrt{c\log N}}1- \Phi(t)^N\ dt.
	\end{align*}
	where $\Phi(t)$ is the CDF of the standard normal. I want an \textit{upper} bound for $\Phi(t)$ here, but our usual tail bounds only give us \textit{lower} bounds on it. Maybe a Mill's ratio thing?
\end{proof}










\noindent\textbf{2.6.5}
Let $X_1, \ldots, X_N$ be in independent sub-Gaussian random variables with zero means and unit variances, and let $a = (a_1, \ldots, a_N)\in \reals^N$. Prove that for every $p\in [2, \infty)$ we have
\[
\left(\sum_{i=1}^Na_i^2\right)^{1/2}\leq \Lp{\sum_{i=1}^Na_iX_i}{p} \leq CK\sqrt{p}\left(\sum_{i=1}^Na_i^2\right)^{1/2}
\]
where $K = \max_i\subg{X_i}$ and $C$ is an absolute constant.
\begin{proof}
	First the lower bound. Since the $X_i$ are independent, centered, and have unit variance, we have
	\begin{align*}
		\left(\sum_{i=1}^Na_i^2\right)^{1/2} &= \left(E\left[\left(\sum_{i=1}^Na_iX_i \right)^2 \right]\right)^{1/2}\\
		&= \Lp{\sum_{i=1}^Na_iX_i}{2}\\
		&\leq \Lp{\sum_{i=1}^Na_iX_i}{p},
	\end{align*}
	for all $p\geq 2$. As for the upper bound, we've established that the sum of sub-Gaussians is sub-Gaussian, so $\sum_{i=1}^Na_iX_i$ is sub-Gaussian. By independence we also have
	\begin{align*}
		\Lp{\sum_{i=1}^Na_iX_i}{p}&\leq C_1\sqrt{p}\cdot \subg{\sum_{i=1}^Na_iX_i}\\
		&\leq C_2\sqrt{p}\left(\sum_{i=1}^N\subg{a_iX_i}^2\right)^{1/2}\\
		&\leq C_2K\sqrt{p}\left(\sum_{i=1}^Na_i^2\right)^{1/2}.
	\end{align*}
\end{proof}










\noindent\textbf{2.6.6}
Show that in the setting of exercise 2.6.5, we have
\[
c(K)\left(\sum_{i=1}^Na_i^2\right)^{1/2} \leq \Lp{\sum_{i=1}^Na_iX_i}{1} \leq \left(\sum_{i=1}^Na_i^2\right)^{1/2}.
\]
Here $K = \max_i\subg{X_i}$ and $c(K) > 0$ is a quantity which may depend only on $K$.
\begin{proof}
	First the lower bound. Let $Z = \sum_i a_i X_i$. First we claim that $\|Z\|_2 \leq \|Z\|_1^{1/4}\|Z\|_3^{3/4}$. To see this, let $a,b>0$ be such that $a+b= 1$ and let $p,q>0$ be such that $\frac{1}{p} + \frac{1}{q} = 1$. We apply H\"older's inequality.
	\begin{align*}
		\Lp{Z}{2} &= \Lp{Z^{a+b}}{2}\\
		&\leq \left(\Lp{Z^{2a}}{p}\Lp{Z^{2b}}{q} \right)^{1/2}\\
		&=\Lp{Z}{2ap}^{a}\Lp{Z}{2bq}^{b}
	\end{align*}
	Setting $a = 1/4$, $b = 3/4$, $p=q=2$ establishes the claim. By independence, $\Lp{Z}{2}^2 = \sum_i a_i^2$. Furthermore, by the previous exercise we have
	\begin{align*}
	\left(\sum_{i=1}^Na_i^2\right)^{1/2} &\leq \Lp{\sum_{i=1}^Na_iX_i}{1}^{1/4}\cdot \Lp{\sum_{i=1}^Na_iX_i}{3}^{3/4}\\
	&\leq \Lp{\sum_{i=1}^Na_iX_i}{1}^{1/4}\cdot (CK\sqrt{3})^{3/4}\left(\sum_{i=1}^Na_i^2\right)^{3/8}.
	\end{align*}
	Dividing through by the sum and raising each term to the power 4 establishes the desired lower bound.
	\[
	(CK\sqrt{3})^{-3}\left(\sum_{i=1}^Na_i^2\right)^{1/2}\leq \Lp{\sum_{i=1}^Na_iX_i}{1}.
	\]
	The upper bound trivially follows by Cauchy-Schwarz: $\Lp{Z}{1} \leq \Pr[\Omega]^{1/2}\cdot \Lp{Z}{2} = \Lp{Z}{2}$ for all measurable functions $Z:\Omega\to \reals$.
\end{proof}










\noindent\textbf{2.6.7}
State and prove a version of Khintchine's inequality for $p\in (0, 2)$.
\begin{solution}
	Set $Z = \sum_i a_iX_i$. The same argument used in 2.6.5 gives the upper bound for $p\in [1, 2)$. For $0<p<1$, we use Jensen's inequality and Cauchy-Schwarz:
	\[
	\Lp{Z}{p} = E[Z^p]^{1/p}\leq E[|Z|]\leq \Pr[\Omega]\cdot \Lp{Z}{2} = \Lp{Z}{2}.
	\]
	We generalize the argument from the last exercise to get the lower bound. Let $a,b>0$ be such that $a+b = 1$ and $q,r > 1$ be such that $\frac{1}{q} + \frac{1}{r} = 1$. Note that $\Lp{Z}{2} = (\sum_i a_i^2)^{1/2}$ by independence. By the same H\"older inequality argument used in the previous exercise, we have
	\[
	\Lp{Z}{2} \leq \Lp{Z}{2aq}^a\Lp{Z}{2br}^b.
	\]
	Applying Khintchine from 2.6.5 gives
	\[
	\Lp{Z}{2} \leq \Lp{Z}{2aq}^a\cdot \left(CK\sqrt{2br}\right)^b\Lp{Z}{2}^b.
	\]
	Rearranging and using the fact that $1-b = a$ gives
	\[
	\left(CK\sqrt{2br}\right)^{-b}\Lp{Z}{2}^a\leq \Lp{Z}{2aq}^a\implies \left(CK\sqrt{2br}\right)^{-b/a}\Lp{Z}{2}\leq \Lp{Z}{2aq}.
	\]
	Applying the relations $a+b= 1$ and $\frac{1}{q} + \frac{1}{r}= 1$ gives
	\[
	\left(CK\sqrt{2(1-a)\frac{q}{q-1}}\right)^{1-1/a}\Lp{Z}{2} \leq \Lp{Z}{2aq}.
	\]
	We need $2aq = p$, which gives
	\[	
	\left(CK\sqrt{2(1-a)\frac{p}{p-2a}}\right)^{1-1/a}\Lp{Z}{2} \leq \Lp{Z}{2p}.
	\]
	This holds for all $a\in (0,1)$ such that $p-2a>0$, so simply (or not so simply) choose such an $a$ that maximizes the constant on the left.
\end{solution}










\noindent\textbf{2.8.5}
Let $X$ be a mean-zero random variable such that $|X|\leq K$. Prove the following bound on the MGF of $X$:
\[
E\exp(\lambda X)\leq \exp(g(\lambda)E[X^2])\quad\text{where}\quad g(\lambda) = \frac{\lambda^2/2}{1-|\lambda|K/3},
\]
provided that $|\lambda|<3/K$.
\begin{proof}
	First, we claim that for $|z|<3$ we have the inequality
	\[
	e^z \leq 1 + z + \frac{z^2/2}{1 - |z|/3}.
	\]
	To see this, we look at the Taylor series for $e^z$.
	\begin{align*}
		e^z - 1 + z  &= \sum_{n=2}^\infty \frac{z^n}{n!}\\
		&= \frac{z^2}{2}\cdot \sum_{n=0}^\infty \frac{2}{(n+2)!}z^n.
	\end{align*}
	Now $\frac{2}{(n+2)!}\leq \frac{1}{3^n}$ for all $n\geq 0$, so we can bound the above sum by a geometric series for $|z|<3$:
	\[
	e^z - 1 + z \leq \frac{z^2}{2}\cdot \sum_{n=0}^\infty (z/3)^n = \frac{z^2/2}{1- |z|/3}.
	\]
	Now we substitute $\lambda X$ in for $z$ and use the fact that $X$ is centered and that $|\lambda|<3/K$.
	\begin{align*}
		E[\exp(\lambda X)] & \leq E\left[1 + X + \frac{(\lambda X)^2/2}{1-|\lambda X|/3} \right]\\
		&\leq 1 + g(\lambda)E[X^2]\\
		&\leq e^{g(\lambda)E[X^2]}.
	\end{align*}
\end{proof}










\noindent\textbf{2.8.6}
Use the result of exercise 2.8.5 to prove the following theorem. Let $X_1, \ldots, X_N$ be independent, mean zero random variables, such that $|X_i|\leq K$ for all $i$. Then, for every $t\geq 0$, we have
\[
\Pr\left[ \bigg|\sum_{i=1}^NX_i\bigg| \geq t\right] \leq 2\exp\left( -\frac{t^2/2}{\sigma^2 + Kt/3}\right),
\]
where $\sigma^2 = \sum_{i=1}^NE[X_i^2]$.
\begin{proof}
	We do the usual multiplying by $\lambda$ and applying Markov trick. Applying the previous exercise gives
	\[
	\Pr\left[ \sum_{i=1}^NX_i \geq t\right] \leq \exp\left( \sigma^2g(\lambda) - \lambda t\right).
	\]
	Now we minimize that exponent.
	\[
	\sigma^2g(\lambda) - \lambda t \leq \frac{\sigma^2}{2}\lambda^2 - t\lambda.
	\]
	This quadratic is minimized at $\lambda = t/\sigma^2$ and substituting this in gives the desired bound.
\end{proof}



\end{document}