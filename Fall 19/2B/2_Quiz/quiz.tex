\documentclass[12pt]{article}  
%%Read the manual for other options. 

\pagestyle{empty} %%Eliminates page numbers
%%\input rmb_macros
%%Collect your favorite macros in a 
%%separate file

%\input amssym.def
%\input amssym
%\input mssymb
%%Defines additional symbols



\usepackage{graphics}
\usepackage{amsmath,amssymb,amsthm, multicol}
\usepackage[pdftex]{graphicx}
\usepackage{epsf}
%%Use to include pictures. 

%\newcommand{\comment}[1]{}
%\newcommand{\sobolev}[2]{W^{#1,#2}}
%\newcommand{\sobolev}[2]{L^#2_#1}
%%Some examples of macros or new commands.

\addtolength{\oddsidemargin}{-.75in}
\addtolength{\evensidemargin}{-.75in}
\addtolength{\textwidth}{1.5in}
\addtolength{\topmargin}{-1in}
\addtolength{\textheight}{2.25in}
%%Set margins, defaults are ok. 

\begin{document}
\begin{flushleft} 
%%Paragraphs will not be indented in this 
%%environment
\centerline{\LARGE{Quiz 2}} 
\vspace{5 mm}
{Form A}\hfill  
%%\hfill forces following text 
%%to right margin
{Name \rule {2 in}{0.01in}}\\
Math 2B, 12 PM
\\
%%gives a line of length 2in and 
%%thickness 0.01in
{Please justify all your answers}\hfill {October 10, 2019}
\\
{Please also write your full name on the back} 

\medskip
\end{flushleft}

\begin{enumerate}
	\item Estimate $\int_0^\pi x\cos x\ dx$ using a left-handed Riemann sum with four rectangles of equal width.
	\vfill

	\item Compute $F'(x)$
	\[
	F(x) = \int_1^{\sin x}\frac{1}{1+t^2}\ dt.
	\]

	\vfill\null
\end{enumerate}
\pagebreak


\begin{flushleft} 
%%Paragraphs will not be indented in this 
%%environment
\centerline{\LARGE{Quiz 2}} 
\vspace{5 mm}
{Form B}\hfill  
%%\hfill forces following text 
%%to right margin
{Name \rule {2 in}{0.01in}}\\
Math 2B, 12 PM
\\
%%gives a line of length 2in and 
%%thickness 0.01in
{Please justify all your answers}\hfill {October 10, 2019}
\\
{Please also write your full name on the back} 

\medskip
\end{flushleft}

\begin{enumerate}
	\item Estimate $\int_0^\pi x\sin x\ dx$ using a right-handed Riemann sum with four rectangles of equal width.
	\vfill

	\item Compute $\frac{dR}{dy}$
	\[
	R(y) = \int_1^{e^y}\ln t\ dt.
	\]

	\vfill\null
\end{enumerate}

%\vfill will divide page evenly
%use \begin{enumerate} environment for ordered lists
\end{document}