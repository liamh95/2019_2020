\documentclass[12pt]{article}  
%%Read the manual for other options. 

\pagestyle{empty} %%Eliminates page numbers
%%\input rmb_macros
%%Collect your favorite macros in a 
%%separate file

%\input amssym.def
%\input amssym
%\input mssymb
%%Defines additional symbols



\usepackage{graphics}
\usepackage{amsmath,amssymb,amsthm, multicol}
\usepackage[pdftex]{graphicx}
\usepackage{epsf}
%%Use to include pictures. 

%\newcommand{\comment}[1]{}
%\newcommand{\sobolev}[2]{W^{#1,#2}}
%\newcommand{\sobolev}[2]{L^#2_#1}
%%Some examples of macros or new commands.

\addtolength{\oddsidemargin}{-.75in}
\addtolength{\evensidemargin}{-.75in}
\addtolength{\textwidth}{1.5in}
\addtolength{\topmargin}{-1in}
\addtolength{\textheight}{2.25in}
%%Set margins, defaults are ok. 

\begin{document}
\begin{center}
{\bf \Large MATH 2B - Integration By Parts}
\vspace{0.2cm}
\hrule
\end{center}

\begin{multicols*}{2}
	\begin{enumerate}
		\item Evaluate the integrals by parts.
		\begin{enumerate}
			\item \[
			\int t^2 \sin (4t)\ dt
			\]
			\vfill
			\item \[
			\int ye^{9y}dy
			\]
			\vfill
			\item \[
			\int \tan^{-1}(2y)\ dy
			\]
		\end{enumerate}
		\vfill
		\item Make a substitution and then integrate by parts.
		\begin{enumerate}
			\item \[
			\int e^{\sqrt x}dx
			\]
			\vfill
			\item \[
			\int e^s\cos(1+s)\ ds
			\]
			\vfill
			\item \[
			\int x\sin x\cos x\ dx
			\]
			\vfill
		\end{enumerate}
		\item Use integration by parts to prove the reduction formula
		\[
		\int (\ln x)^n\ dx = x(\ln x)^n - n\int(\ln x)^{n-1}dx.
		\]
		Find a similar formula for $\int x^ne^x\ dx$.
		\vfill\null\columnbreak
		\item Evaluate the integral (not just integration by parts).
		\begin{enumerate}
			\item \[
			\int \frac{1}{\sqrt{e^{2x}-1}}dx
			\]
			\vfill
			\item \[
			\int \cos(\log x) dx
			\]
			\vfill
			\item \[
			\int \frac{\arctan x}{1+x^2}dx
			\]
			\vfill
			\item \[
			\int \log \sqrt{1+x^2}\ dx
			\]
			\vfill
			\item \[
			\int \arctan x\ dx
			\]
			\vfill
			\item \[
			\int x\arctan x\ dx
			\]
			\vfill
		\end{enumerate}
		\item This result (the Riemann-Lebesgue lemma) is really important in a branch of mathematics called harmonic analysis. Assuming $f'$ is continuous on $[a,b]$, use integration by parts to prove that
		\[
		\lim_{n\to \infty}\int_a^bf(t)\sin(nt)\ dt = 0.
		\]
		\vfill
	\end{enumerate}	
\end{multicols*}


\end{document}