\documentclass[11pt,letterpaper]{report}
\usepackage{amssymb,amsfonts,color,graphicx,amsmath,enumerate}
\usepackage{amsthm}

\newcommand{\naturals}{\mathbb{N}}
\newcommand{\integers}{\mathbb{Z}}
\newcommand{\complex}{\mathbb{C}}
\newcommand{\reals}{\mathbb{R}}
\newcommand{\mcal}[1]{\mathcal{#1}}
\newcommand{\rationals}{\mathbb{Q}}
\newcommand{\field}{\mathbb{F}}
\newcommand{\Var}{\text{Var}}
\newcommand{\ind}{\mathbbm{1}}
\newcommand{\Cov}{\text{Cov}}

\newenvironment{solution}
{\begin{proof}[Solution]}
{\end{proof}}

\voffset=-3cm
\hoffset=-2.25cm
\textheight=24cm
\textwidth=17.25cm
\addtolength{\jot}{8pt}
\linespread{1.3}

\begin{document}
\begin{center}
{\bf \Large Math 130B - Moment Generating Functions and the Central Limit Theorem}
\vspace{0.2cm}
\hrule
\end{center}

\begin{enumerate}
	%mitz 97
	\item Use moment generating functions to prove that the sum of finitely many independent Poisson random variables is also a Poisson random variable.

	\vfill

	\item Here we use moment generating functions to derive a neat tail bound. Let $X$ be a random variable. Suppose the MGF of $X^2$ satisfies
	\[
	M_{X^2}(\lambda^2)\leq e^{C_1^2\lambda^2}\quad\text{for all $\lambda$ such that }|\lambda|\leq \frac{1}{C_1},
	\]
	where $C_1$ is some positive constant. Deduce that the MGF of $X^2$ is bounded at some point, namely that
	\[
	E[\exp(X^2/C_2^2)]\leq 2
	\]
	for some positive constant $C_2$. Use this to prove the following tail bound for all $t\geq 0$
	\[
	\Pr\big[|X|\geq t\big]\leq 2\exp\big(-t^2/C_3^2\big),
	\]
	where $C_3$ is some positive constant. \textit{Hint: look back at the proof of Chernoff.}

	\vfill

	\item Fifty numbers are rounded off to the nearest integer then summed. If the individual round-off errors are uniformly distributed over $(-.5, .5)$, approximate the probability that the resultant sum differs from the exact sum by more than 3.

	\vfill

	\item Let $X_1, X_2, \ldots$ be a sequence of iid random variables having finite mean and variance. While the central limit theorem says that $\sum_{i=1}^nX_i$ approaches a normal distribution as $n\to \infty$, it doesn't say how fast, i.e. it doesn't say how large $n$ needs to be in order for the normal approximation to be any good.\\

	It happens that the normal approximation was pretty good in the last example. In lots of applications, the normal approximation is good enough when $n\geq 20$ or even smaller. Come up with a distribution for the $X_i$'s so that $\sum_{i=1}^{100}X_i$ isn't close to a normal distribution. \textit{Hint: problem 1.}
	\vfill
\end{enumerate}


\end{document}