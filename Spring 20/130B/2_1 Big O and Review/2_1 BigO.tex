\documentclass[11pt,letterpaper]{report}
\usepackage{amssymb,amsfonts,color,graphicx,amsmath,enumerate}
\usepackage{tikz}
\usepackage{amsthm}

\newcommand{\naturals}{\mathbb{N}}
\newcommand{\integers}{\mathbb{Z}}
\newcommand{\complex}{\mathbb{C}}
\newcommand{\reals}{\mathbb{R}}
\newcommand{\mcal}[1]{\mathcal{#1}}
\newcommand{\rationals}{\mathbb{Q}}
\newcommand{\Lp}[2]{\left\|{#1}\right\|_{L^{#2}}}
\newcommand{\field}{\mathbb{F}}
\newcommand{\affine}{\mathbb{A}}
\newcommand{\E}{\mathbb{E}}
\newcommand{\Prob}{\mathbb{P}}
\newcommand{\Var}{\text{Var}}
\newcommand{\ind}{\mathbbm{1}}
\newcommand{\Cov}{\text{Cov}}

\theoremstyle{definition}
\newtheorem{definition}{Definition}
\newtheorem{exercise}{Exercise}
\newtheorem{theorem}{Theorem}

\newenvironment{solution}
{\begin{proof}[Solution]}
{\end{proof}}

\voffset=-3cm
\hoffset=-2.25cm
\textheight=24cm
\textwidth=17.25cm
\addtolength{\jot}{8pt}
\linespread{1.3}

\begin{document}
\begin{center}
{\bf \Large Math 130B - Big $O$ Notation}
\vspace{0.2cm}
\hrule
\end{center}


The following definition gives us a rigorous way of saying one function is ``larger'' than another.
\begin{definition}
	Let $f,g:\reals\to \reals$ be functions.
	\begin{enumerate}[(a)]
		\item We write $f(x) = O(g(x))$ if there exist positive constants $c_1$ and $c_2$ such that $|f(x)| \leq c_1|g(x)|$ for all $x \geq c_2$. That is, $f$ is \textit{eventually} at most a constant multiple of $g$.
		\item $f(x) = \Omega(g(x))$ if there exist positive constants $c_1$ and $c_2$ such that $|f(x)| \geq c_1|g(x)|$ for all $x\geq c_2$. That is, $f$ is eventually at least a constant multiple of $g$.
		\item $f(x) = \Theta(g(x))$ if $f(x) = O(g(x))$ and $f(x) = \Omega(g(x))$. 
	\end{enumerate}
\end{definition}

% \begin{exercise}
% 	Show that if $f(x) = O(g(x))$ and $g(x) = O(h(x))$ then $f(x) = O(h(x))$ and that the same is true if we replace $O$ with $\Omega$.
% \end{exercise}

\begin{exercise}
	\begin{enumerate}[(a)]
		\item Let $f(x) = ax+b$ and $g(x) = cx+d$ where $a,c\neq 0$. Show that $f(x) = O(g(x))$.
		\item Let $f(x) = ax^2+bx+c$, $a > 0$. Show that $f(x) = \Omega(x^2)$.
		\item Show that $\sin x = \Theta(1)$.
	\end{enumerate}
\end{exercise}

These definitions can be a bit cumbersome to work with sometimes. The following is sometimes easier to check. Try proving it!

\begin{theorem}
	\begin{enumerate}[(a)]
		\item If $\lim_{x\to \infty}\frac{|f(x)|}{|g(x)|}<\infty$ then $f(x) = O(g(x))$.
		\item If $\lim_{x\to \infty}\frac{|f(x)|}{|g(x)|} > 0$ then $f(x) = \Omega(g(x))$.
	\end{enumerate}
\end{theorem}

\begin{exercise}
	Prove the following.
	\begin{enumerate}[(a)]
		\item $x^2 + \sqrt{x} = O(x^2)$.
		\item If $\lim_{x\to \infty}f(x) = \infty$, then $f(x) + \sin x = \Theta(f(x))$.
		\item $k^22^k = O(e^{2k})$.
		\item $N^{10}2^N = O(e^N)$.n
	\end{enumerate}
\end{exercise}

We also have notation to express the idea of one function being \textit{strictly} less or greater than another.

\begin{definition}
	Let $f,g:\reals\to \reals$ be functions.
	\begin{enumerate}[(a)]
		\item We write $f(x) = o(g(x))$ if for all $c_1>0$ there exists a $c_2>0$ so that $|f(x)|\leq c_1|g(x)|$ for all $x \geq c_2$. That is, $f$ is eventually smaller than \textit{any} constant multiple of $g$.
		\item We write $f(x)= \omega(g(x))$ if for all $c_1>0$ there exists a $c_2>0$ so that $|f(x)|\geq c_1|g(x)|$ for all $x\geq c_2$. That is, $g$ is eventually greater than any multiple of $g$.
	\end{enumerate}
\end{definition}

Just like with $O$ and $\Omega$, we can take limits to show $f(x) = o(g(x))$ or $\omega(g(x))$.

\begin{theorem}
	\begin{enumerate}[(a)]
		\item $f(x) = o(g(x))$ if and only if $\lim_{x\to \infty}\frac{f(x)}{g(x)} = 0$.
		\item $f(x) = \omega(g(x))$ if and only if $\lim_{x\to \infty}\frac{|f(x)|}{|g(x)|} = \infty$.
	\end{enumerate}
\end{theorem}

\begin{exercise}
	Prove the following.
	\begin{enumerate}[(a)]
		\item We often say that a sequence of events $E_n$ happens ``with high probability'' if $\Pr[E_n] = 1-o(1)$. Why does this make sense?
		\item If $f(x) = o(g(x))$ then $f(x) = O(g(x))$. If $f(x) = \omega(g(x))$ then $f(x) = \Omega(g(x))$. Give examples to show that the converses to these statements are false.
		\item $k^{300} = o(2^k)$.
		\item $k^{0.001} = \omega((\log k)^{375})$.
	\end{enumerate}
\end{exercise}


\end{document}