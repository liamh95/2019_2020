\documentclass[11pt,letterpaper]{report}
\usepackage{amssymb,amsfonts,color,graphicx,amsmath,enumerate}
\usepackage{amsthm}

\newcommand{\naturals}{\mathbb{N}}
\newcommand{\integers}{\mathbb{Z}}
\newcommand{\complex}{\mathbb{C}}
\newcommand{\reals}{\mathbb{R}}
\newcommand{\mcal}[1]{\mathcal{#1}}
\newcommand{\rationals}{\mathbb{Q}}
\newcommand{\field}{\mathbb{F}}
\newcommand{\Var}{\text{Var}}
\newcommand{\ind}{\mathbbm{1}}
\newcommand{\Cov}{\text{Cov}}

\newenvironment{solution}
{\begin{proof}[Solution]}
{\end{proof}}

\voffset=-3cm
\hoffset=-2.25cm
\textheight=24cm
\textwidth=17.25cm
\addtolength{\jot}{8pt}
\linespread{1.3}

\begin{document}
\begin{center}
{\bf \Large Math 130B - Homework 4}
\vspace{0.2cm}
\hrule
\end{center}


\centerline{\textbf{You must type up your solutions to this assigment in \LaTeX.}}
\begin{enumerate}
	\item Suppose that $n$ points are independently chosen at random on the circumference of a circle, and we want the probability that they all lie in a semicircle. That is, we want the probability that there is a line passing through the center of the circle such that all the points are on one side of that line.\\

	\noindent Let $P_1, \ldots, P_n$ denote the $n$ points. Let $A^{(n)}$ denote the event that all the points are contained in some semicircle, and let $A^{(n)}_i$ be the event that all the points lie in the semicircle beginning at the point $P_i$ and going clockwise for $180^\circ$, $i = 1, \ldots, n$.
	\begin{enumerate}
		\item Express $A^{(n)}$ in terms of the $A^{(n)}_i$.
		\item Find $\Pr[A^{(n)}]$ and show that it is $o(1)$.
	\end{enumerate}


	\item The joint density function of $X$ and $Y$ is given by
	\[
	f(x,y) = xe^{-x(y+1)},\quad x>0,\ y>0.
	\]
	\begin{enumerate}
		\item Find the conditional density of $X$, given $Y = y$, and that of $Y$, given $X = x$.
		\item Find the density function of $Z = XY$.
	\end{enumerate}

	\item Suppose that $A$, $B$, $C$ are independent random variables, each being uniformly distributed over $(0, 1)$.
	\begin{enumerate}
		\item What is the joint cumulative distribution function of $A$, $B$, $C$?
		\item What is the probability that all of the roots of the equation $Ax^2 + Bx + C = 0$ are real?
	\end{enumerate}

	\item Show that the jointly continuous (discrete) random variables $X_1, \ldots, X_n$ are independent if and only if their joint probability density (mass) function $f(x_1, \ldots, x_n)$ can be written as
	\[
	f(x_1, \ldots, x_n) = \prod_{i=1}^ng_i(x_i)
	\]
	for nonnegative functions $g_i(x)$, $i= 1, \ldots, n$.


	\item Suppose that a random real number $x$ has been chosen according to some distribution and you have two sealed envelopes in front of you. One contains the number $x$ and the other contains the number $2x$.
	\begin{enumerate}
		\item If you need to guess which of the envelopes contains the number $x$, what would be your probability of guessing correctly?

		\item Now suppose that you are allowed to pick one envelope and look at the number inside of it before you make a decision. Can you come up with a strategy that gives you a better success rate than the one in part (a)?
	\end{enumerate}
\end{enumerate}

\end{document}