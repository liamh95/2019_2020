\documentclass[11pt,letterpaper]{report}
\usepackage{amssymb,amsfonts,color,graphicx,amsmath,enumerate}
\usepackage{tikz}
\usepackage{amsthm}
\usepackage{bbm}
\usepackage{hyperref}

\newcommand{\naturals}{\mathbb{N}}
\newcommand{\integers}{\mathbb{Z}}
\newcommand{\complex}{\mathbb{C}}
\newcommand{\reals}{\mathbb{R}}
\newcommand{\mcal}[1]{\mathcal{#1}}
\newcommand{\rationals}{\mathbb{Q}}
\newcommand{\Lp}[2]{\left\|{#1}\right\|_{L^{#2}}}
\newcommand{\field}{\mathbb{F}}
\newcommand{\affine}{\mathbb{A}}
\newcommand{\E}{\mathbb{E}}
\newcommand{\Prob}{\mathbb{P}}
\newcommand{\Var}{\text{Var}}
\newcommand{\ind}{\mathbbm{1}}
\newcommand{\Cov}{\text{Cov}}

\newtheorem{theorem}{Theorem}[section]
\newtheorem{corollary}{Corollary}[theorem]
\newtheorem{lemma}[theorem]{Lemma}
\newtheorem*{claim*}{Claim}

\newenvironment{solution}
{\begin{proof}[Solution]}
{\end{proof}}

\voffset=-3cm
\hoffset=-2.25cm
\textheight=24cm
\textwidth=17.25cm
\addtolength{\jot}{8pt}
\linespread{1.3}

\begin{document}
%\noindent{\em Liam Hardiman\hfill{Date} }
\begin{center}
{\bf \Large Math 175 - Week 2, Discussion 1}
\vspace{0.2cm}
\hrule
\end{center}

\begin{enumerate}

	\item What's wrong with this proof?
	\begin{claim*}
		$n(n+1)$ is an odd number for every natural $n$.
	\end{claim*}
	\begin{proof}
		Suppose this is true for $n$; we prove it for $n+1$. Some quick algebra shows that
		\[
		(n+1)(n+2) = n(n+1) + 2(n+1).
		\]
		By the induction hypothesis, $n(n+1)$ is odd. Since $2(n+1)$ is clearly even, we must have that $(n+1)(n+2)$ is odd.
	\end{proof}
	\vfill

	\item What's wrong with this proof?
	\begin{claim*}
		If we have $n$ lines in the plane, no two of which are parallel, then they all go through one point.
	\end{claim*}
	\begin{proof}
		This is clearly true for one or two lines by definition. Suppose that it is true for any set of $n$ lines and let $S = \{\ell_1, \ell_2, \ell_3, \ell_4,  \ldots, \ell_{n+1}\}$ be a set of $n+1$ lines in the plane, no two of which are parallel. Delete the line $\ell_3$ to obtain a set $S'$ of $n$ lines, no two of which are parallel. By the induction hypothesis, the lines in $S'$ must all pass through some point $P$. In particular, $\ell_1$ and $\ell_2$ pass through $P$.\\

		\noindent Now put $\ell_3$ back and delete $\ell_4$ instead to get another set $S''$ of $n$ lines, no two of which are parallel. Again by the induction hypothesis, they must all pass through some point $Q$. In particular, $\ell_1$ and $\ell_2$ pass through $Q$. But $\ell_1$ and $\ell_2$ pass through $P$. Since two lines can pass through at most one point, we must have $P = Q$. But then $\ell_3$ goes through $P$, so all the lines in $S$ go through $P$.
	\end{proof}

	\vfill

	\item What is the following sum? Prove your guess by induction.
	\[
	\frac{1}{1\cdot 2} + \frac{1}{2\cdot 3} + \frac{1}{3\cdot 4} + \cdots + \frac{1}{(n-1)n}.
	\]

	\vfill

	\item What is the following sum? Give an inductive proof and a combinatorial proof.
	\[
	0\cdot \binom{n}{0} + 1\cdot \binom{n}{1} + 2\cdot \binom{n}{2} + \cdots + (n-1)\cdot\binom{n}{n-1} + n\cdot \binom{n}{n}
	\]
	\vfill\pagebreak

	\item Prove the following identity. Try to give a combinatorial proof.
	\[
	\binom{n}{k} = \binom{n-1}{k} + \binom{n-1}{k-1}.
	\]

	\vfill

	\item Prove the following identity.
	\[
	\binom{2}{2} + \binom{3}{2} + \cdots + \binom{n}{2} = \binom{n+1}{3}.
	\]

	\vfill

	\item We select 38 even positive integers, all less than 1000. Prove that there will be two of them whose difference is at most 26.

	\vfill

	\item Prove that in a group of $n$ people, at least two will have the same number of friends in that group.

	\vfill

	\item Let $S$ be a finite set. Prove that $f: S\to S$ is injective (one-to-one) if and only if it is surjective (onto).\\

	\vfill

	\item (If you've seen 120B) Prove that any finite integral domain is a field.

	\vfill

	\item (Hard) Every element of the infinite grid $\integers\times \integers$ is colored red, green, or blue. Prove that there is a rectangle whose corners are all the same color.
	\vfill

\end{enumerate}


\end{document}