\documentclass[11pt,letterpaper]{report}
\usepackage{amssymb,amsfonts,color,graphicx,amsmath,enumerate}
\usepackage{tikz}
\usepackage{amsthm}
\usepackage{bbm}

\newcommand{\naturals}{\mathbb{N}}
\newcommand{\integers}{\mathbb{Z}}
\newcommand{\complex}{\mathbb{C}}
\newcommand{\reals}{\mathbb{R}}
\newcommand{\exreals}{\overline{\mathbb{R}}}
\newcommand{\mcal}[1]{\mathcal{#1}}
\newcommand{\mable}{measurable}
\newcommand{\quats}{\mathbb{H}}
\newcommand{\rationals}{\mathbb{Q}}
\newcommand{\norm}{\trianglelefteq}
\newcommand{\Aut}{\text{Aut}}
\newcommand{\disk}{\mathbb{D}}
\newcommand{\halfplane}{\mathbb{H}}
\newcommand{\Lp}[2]{\left\|{#1}\right\|_{L^{#2}}}
\newcommand{\supp}[1]{\text{supp}({#1})}
\newcommand{\Hom}[2]{\text{Hom}_{{#1}}({#2})}
\newcommand{\tr}{\text{tr}}
\newcommand{\field}[1]{\mathbb{F}_{{#1}}}
\newcommand{\Gal}[1]{\text{Gal}\left({#1}\right)}
\newcommand{\esssup}{\text{ess sup }}
\newcommand{\essinf}{\text{ess inf }}
\newcommand{\affine}{\mathbb{A}}
\newcommand{\E}{\mathbb{E}}
\newcommand{\Prob}{\mathbb{P}}
\newcommand{\ind}{\mathbbm{1}}

\newenvironment{solution}
{\begin{proof}[Solution]}
{\end{proof}}

\voffset=-3cm
\hoffset=-2.25cm
\textheight=24cm
\textwidth=17.25cm
\addtolength{\jot}{8pt}
\linespread{1.3}

\begin{document}
\noindent{\em Liam Hardiman\hfill{October 28, 2019} }
\begin{center}
{\bf \Large 270A - Homework 1}
\vspace{0.2cm}
\hrule
\end{center}

\begin{enumerate}
	\item \begin{enumerate}
		\item Let $\mcal{F}$ be the family of all finite subsets of $\Omega$ and their complements. Is $\mcal{F}$ a $\sigma$-algebra?
		\begin{solution}
			If $\Omega$ is finite then $\mcal{F}$ is simply the power set of $\Omega$, which is definitely a $\sigma$-algebra. However, if $\mcal{F}$ is infinite, then $\mcal{F}$ is never a $\sigma$-algebra. To see this, let $(x_n)$ be a countable sequence of distinct elements in $\Omega$ and consider the set of even-indexed terms
			\[
				F = \{x_n: n = 2k,\ k\in \naturals\}.
			\]
			This set is a countable union of singletons and all singletons belong to $\mcal{F}$. $F$ is clearly infinite, but so is its complement, which contains the (infinite) set of odd-indexed terms. We conclude that $F$ is neither finite nor co-finite, so $\mcal{F}$ is not closed under countable unions when $\Omega$ is an infinite set.
		\end{solution}

		\item Let $\mcal{F}$ be the family of all countable subsets of $\Omega$ and their complements. Is $\mcal{F}$ a $\sigma$-algebra?
		\begin{solution}
			$\mcal{F}$ is indeed a $\sigma$-algebra. The empty set is clearly countable, and $\Omega^C = \emptyset$. Let $F_n$ be a countable collection of sets in $\mcal{F}$ and consider their union, $F = \cup_{n=1}^\infty F_n$. If each $F_n$ is countable, then $F$ is just a countable union of countable sets: countable. If one of the $F_n$'s, say $F_k$, were co-countable, then $F^C \subseteq F_k^C$, which is countable, so $F$ is co-countable. Since $\mcal{F}$ contains the empty set and $\Omega$ and is closed under countable unions and complements, it is a $\sigma$-algebra.
		\end{solution}

		\item Let $\mcal{F}$ and $\mcal{G}$ be two $\sigma$-algebras of subsets of $\Omega$. Is $\mcal{F}\cap \mcal{G}$ always a $\sigma$-algebra?
		\begin{solution}
			$\mcal{F}$ is a $\sigma$-algebra. Since $\mcal{F}$ and $\mcal{G}$ both contain $\emptyset$ and $\Omega$, so does their intersection. Let $E_n$ be a countable collection of sets in $\mcal{F}\cap \mcal{G}$. Since $\mcal{F}$ and $\mcal{G}$ are both $\sigma$-algebras, the union $E = \cup_{n=1}^\infty E_n$ is in both $\mcal{F}$ and $\mcal{G}$ and each $E_n^C$ is in both $\mcal{F}$ and $\mcal{G}$ as well.
		\end{solution}

		\item Let $\mcal{F}$ and $\mcal{G}$ be two $\sigma$-algebras of subsets of $\Omega$. Is $\mcal{F}\cup \mcal{G}$ always a $\sigma$-algebra?
		\begin{solution}
			The union need not be a $\sigma$-algebra. Let $\Omega = \{1, 2, 3, 4\}$, $\mcal{F} = \{\emptyset, \Omega, \{1\}, \{2, 3, 4\}\}$, and $\mcal{G} = \{\emptyset, \Omega, \{2\}, \{1, 3, 4\}\}$. $\mcal{F}$ and $\mcal{G}$ are $\sigma$-algebras, but the set $\{1\}\cup \{2\} = \{1,2\}$ is not in their union.
		\end{solution}
	\end{enumerate}

	\item A subset $A\subset \naturals$ is said to have asymptotic density if
	\[
	\lim_{n\to \infty}\frac{|A\cap \{1, \ldots, n\}|}{n}
	\]
	exists. Let $\mcal{F}$ be the collection of subsets of $\naturals$ for which the asymptotic density exists. Is $\mcal{F}$ a $\sigma$-algebra?
	\begin{solution}
		$\mcal{F}$ is not a $\sigma$-algebra. First let's construct a set not in $\mcal{F}$. The idea is to build a set that has long gaps followed by even longer ``runs''. Let $F_0 = \{1\}$ and $F_i = \{2^i, \ldots, 2^{i+1}-1\}$. Define the set $A$ by $A = \cup_{j=0}^\infty F_{2j}$. $A$ consists of a run of length $2^{2j}$ followed by a gap of length $2^{2j+1}$ for each $j = 0, 1, \ldots$. Our set $A$ does not have asymptotic density since
		\begin{align*}
			\frac{|A\cap [2^{2k}]|}{2^{2k}} &= \frac{\sum_{j=0}^{k-1}2^{2j}+1}{2^{2k}}\\
			&= \frac{1}{3}
		\end{align*}
		while on the other hand,
		\begin{align*}
			\frac{|A\cap [2^{2k+1}]|}{2^{2k+1}} &= \frac{\sum_{j=0}^{k}2^{2j}}{2^{2k+1}}\\
			&= \frac{1}{3}\left(2 - \frac{1}{2^{2k+1}}\right)\\
			&\to \frac{2}{3}.
		\end{align*}
		Hence, $A$ is not in $\mcal{F}$. Since $A$ is a countable union of singletons, which clearly have asymptotic density zero, we conclude that $\mcal{F}$ is not a $\sigma$-algebra.\\
	\end{solution}

	\item Let $X$ and $Y$ be two random variables on the same probability space $(\Omega, \mcal{F}, \Prob)$, and let $E\in \mcal{F}$ be an event. Define
	\[
	Z = \begin{cases}
		X&\text{if }E\text{ occurs}\\
		Y&\text{otherwise.}
	\end{cases}
	\]
	Prove that $Z$ is a random variable.
	\begin{proof}
		We can write $Z = X\cdot \ind_E + Y\cdot \ind_{E^c}$. Since $E$ is an event, the indicator functions $\ind_E$ and $\ind_{E^c}$ are measurable. Since $X$ and $Y$ are measurable and products and sums of measurable functions are measurable, we have that $Z$ is measurable, and hence a random variable.
	\end{proof}

	\item Let $X$ be a random variable with density $f$. Compute the density of $X^2$.
	\begin{solution}
		First let's compute the distribution of $X^2$. Let $t\geq 0$.
		\begin{align*}
			\Prob[X^2<t] &= \Prob[-\sqrt{t}<X<\sqrt{t}]\\
			&= \int_{-\sqrt{t}}^{\sqrt{t}}f(s)\ ds.
		\end{align*}
		By the Lebesgue differentiation theorem, the above integral is an almost everywhere differentiable function of $t$ and we can apply the fundamental theorem of calculus. If we let $g$ be the density of $X^2$ then
		\begin{align*}
			g(t) &= \frac{d}{dt}\int_{-\sqrt{t}}^{\sqrt{t}}f(s)\ ds\\
			&= \frac{1}{2\sqrt{t}}[f(\sqrt{t}) + f(-\sqrt{t})],
		\end{align*}
		for $t\geq 0$. Since $X^2$ is clearly nonnegative, we then have
		\[
		g(t) = \begin{cases}
			\frac{1}{2\sqrt{t}}[f(\sqrt{t})+f(-\sqrt{t})],&t>0\\
			0,&t\leq 0.
		\end{cases}
		\]
	\end{solution}

	\item Let $X$ be a nonnegative random variable. Show that
	\[
	\E[X] = \int_0^\infty \Prob[X>t]\ dt.
	\]
	\begin{proof}
		Since $X$ is nonnegative (this is important -- the Lebesgue integral is orientation-independent, unlike the Riemann integral!),
		\begin{align*}
			X &= \int_0^Xdt.
		\end{align*}
		We can then take the expectation of both sides and apply Fubini's theorem.
		\begin{align*}
			\E[X] &= \int_\Omega\int_0^Xdt\ d\Prob\\
			&= \int_\Omega\int_0^\infty \ind_{X>t}(t)\ dt\ d\Prob\\
	 		&= \int_0^\infty \int_\Omega\ind_{X>t}(x)\ d\Prob\ dt\\
			&= \int_0^\infty \Prob[X>t]\ dt.
		\end{align*}
	\end{proof}

	\item Let $\varphi:\reals\to \reals$ be a strictly convex function. Let $X$ be a random variable such that $\E[|X|]<\infty$ and $\E[|\varphi(X)|]\leq \infty$. Show that
	\[
	\varphi(\E[X]) = \E[\varphi(X)] \implies X = \E[X]\text{ a.s.}
	\]
	\begin{proof}
		Since $\varphi$ is strictly convex, for every $t\in \reals$ there exists an affine linear function $F_t(x)$ such that $F_t(t) = \varphi(t)$ and $F_t(x)<\varphi(x)$ for all $x\neq t$. We can set $t = \E[X]$ compose with $X$ to obtain $F_t(X)\leq\varphi(X)$, with equality if and only if $X = \E[X]$. Note that since $F_t$ is affine linear we have that $\E[F_t(X)] = F_t(\E[X])$.\\

		\noindent Suppose that $\varphi(\E[X]) = \E[\varphi(X)]$. When $t = \E[X]$, $F_t$ and $\varphi$ agree at $\E[X]$, so $\varphi(\E[X]) = F_t(\E[X])$, yielding
		\begin{align*}
			\E[\varphi(X)] &= \varphi(\E[X])\\
			&= F_t(\E[X])\\
			&= \E[F_t(X)].
		\end{align*}
		By the linearity of expectation we then have $\E[\varphi(X) - F_t(X)] = 0$. By convexity, $\varphi(X) - F_t(X)\geq 0$, so since this expectation is zero, we must have that $\varphi(X) = F_t(X)$ almost surely. By strict convexity, this implies that $X = t = \E[X]$ almost surely.
	\end{proof}

	\item Suppose $0\leq p_n\leq 1$ and put $\alpha_n = \min(p_n, 1-p_n)$. Show that if $\sum_n\alpha_n$ diverges, then no discrete probability space can contain independent events $A_1, A_2, \ldots$ such that $\Prob[A_n] = p_n$.
\end{enumerate}

\end{document}