\documentclass[11pt,letterpaper]{report}
\usepackage{amssymb,amsfonts,color,graphicx,amsmath,enumerate}
\usepackage{amsthm}

\newcommand{\naturals}{\mathbb{N}}
\newcommand{\integers}{\mathbb{Z}}
\newcommand{\complex}{\mathbb{C}}
\newcommand{\reals}{\mathbb{R}}
\newcommand{\mcal}[1]{\mathcal{#1}}
\newcommand{\rationals}{\mathbb{Q}}
\newcommand{\field}{\mathbb{F}}
\newcommand{\Var}{\text{Var}}
\newcommand{\ind}{\mathbbm{1}}
\newcommand{\Cov}{\text{Cov}}

\newenvironment{solution}
{\begin{proof}[Solution]}
{\end{proof}}

\voffset=-3cm
\hoffset=-2.25cm
\textheight=24cm
\textwidth=17.25cm
\addtolength{\jot}{8pt}
\linespread{1.3}

\begin{document}
\begin{center}
{\bf \Large Math 130B - Homework 2}
\vspace{0.2cm}
\hrule
\end{center}

\begin{enumerate}
	\item A biased coin with head probability $p$ is flipped $n$ times. Let $X$ be the number of heads and $Y$ the number of tails. Clearly, $X$ and $Y$ are not independent since $Y = n-X$ (in particular, knowing $X$ lets us compute $Y$). Suppose instead that the coin is tossed a random number of times $N$, where $N\sim Poisson(\lambda)$. Show that now $X$ and $Y$ are independent.

	\item Let $X$ be a random variable with density $f$. Compute the density of $X^2$. \textit{Hint: first compute the distribution function of $X^2$, then differentiate.}

	\item Suppose that $n$ prisoners are playing the following game: there are $n$ boxes, and each box contains a card with a name of exactly one prisoner and every name appears exactly once. The game is played in $n$ rounds, where in round $i$, prisoner $i$ goes into the room with the boxes and can open up to $n/2$ boxes. If one of the opened boxes contains their name, they close all the other boxes and immediately get out of jail, get \$1,000,000 and start a new life (without seeing/talking to their prisoner friends).  If they fail in finding their name in one of these boxes, then they and all the other remaining prisoners are immediately executed.\\

	The rules are that no communication between the prisoners is allowed after the game starts (they can talk before the game starts and agree on some strategy). Moreover, the boxes are labelled and the cards are distributed uniformly at random among the boxes.\\

	Come up with a good strategy that with probability at least $0.3$ ensures that all prisoners will be able to start a new life, regardless of the number of prisoners.

	\item Suppose that a couple has two children and that the sexes and birthdays of the children are independent.
	\begin{enumerate}
		\item Suppose that the older child is a boy. What is the probability that the younger one is a boy as well?
		\item Suppose we are told at least one of the children is a boy. What is the probability that both of them are boys?
		\item Say we're told that at least one of the children is a boy born on a Monday. What's the probability that both children are boys?
		\item Say we're told that at least one child is born on a Monday between 6PM and 7PM. What's the probability that both are boys?
		\item What happens to this probability as we give more information on one of the boys?
	\end{enumerate}

	\item Let $\{E_i\}_{i\in A}$ be a set of events. Recall that these events are pairwise independent if $\Pr[E_i\cap E_j] = \Pr[E_i]\Pr[E_j]$ for all $i\neq j$ and mutually independent if $\Pr[\cap_{i\in I}E_i] = \prod_{i\in I}E_i$ for all finite subsets $I\subset A$.
	\begin{enumerate}[(a)]
		\item Prove that if the events are mutually independent, then they are also pairwise independent.
		\item Is the converse to part (a) true? Prove it or give a counterexample.
	\end{enumerate}

	\item You are trying to send your friend a single bit, either a 0 or a 1. When you transmit the bit, it goes through a series of $n$ relays before it gets to them. Each relay flips the bit independently with probability $p$.
	\begin{enumerate}[(a)]
		\item Show that the probability you receive the correct bit is
		\[
		\sum_{k=0}^{\lfloor n/2\rfloor}\binom{n}{2k}p^{2k}(1-p)^{2k}.
		\]

		\item Let's look at another way to compute this probability. We say the relay has \textit{bias} $q$ if the probability it flips the bit is $(1-q)/2$. Prove that sending a bit through two relays with bias $q_1$ and $q_2$ is equivalent to sending a bit through a single relay with bias $q_1q_2$.

		\item Prove that the probability you receive the correct bit when it passes through $n$ relays as described before (a) is
		\[
		\frac{1+(2p-1)^n}{2}.
		\]
	\end{enumerate}

% cycle prisoners

% littlewood offard: sum ai*xi, a_i positive, x_i 0/1. given any m, P[sum a_i x_i <=]...
% then do it +/- 1

% boy or girl increasing specificity

% pick a random number. if the number is larger than your choice guess that it's the bigger. if both smaller it's 1/2.
% 	a) without looking
%	b) looking inside one
\end{enumerate}

\end{document}