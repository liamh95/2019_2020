\documentclass[12pt,fleqn]{article}  
\usepackage{amsmath}
\usepackage{amssymb}
\usepackage{graphics}
\usepackage{pgfplots}

\pagestyle{empty}

\unitlength 1cm
\textheight 22cm
\textwidth 17cm
\oddsidemargin -0.5cm
\evensidemargin -0.5cm
\topmargin -1.5cm
\topskip 0cm
\headheight 0.5cm
\headsep 1cm
\marginparwidth 1.2cm

\begin{document}

\begin{center}
	\textbf{Math 2B Worksheet: 5.2 The Definite Integral}
\end{center}

\emph{Write your names and Student ID numbers at the top of the page}


\begin{enumerate}
	\item If $\int_2^8f(x)\;dx=7.3$ and $\int_2^4f(x)\;dx=5.9$, find $\int_4^8f(x)\;dx$.\\[150pt]


	\item Write the expression below as a single integral in the form $\int_a^bf(x)\;dx$.
	\[\int_{-2}^2f(x)\;dx+\int_2^5f(x)\;dx-\int_{-2}^1f(x)\;dx\]
	
	\vspace{150pt}

	\item If $\int_0^9f(x)\;dx=37$ and $\int_0^9g(x)\;dx=16$, find
	\[\int_0^9[2f(x)+3g(x)]\;dx\]
	
	\newpage
	

	\item Consider the function $h(s)=\sqrt{s}$.  
	\begin{enumerate}
		\item Write an expression in sigma (summation) notation for estimating the area under the curve of $h(s)$ on the interval $[1,5]$ using 8 rectangles and right endpoints.\\[200pt]

	\item Write an expression in sigma notation for finding the \textbf{\textit{exact}} area under the curve of $h(s)$ from $[1,5]$.  
\end{enumerate}

\end{enumerate}



\end{document}

